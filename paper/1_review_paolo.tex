\section{A Brief Review of the ddPCM Strategy}\label{sec:review}

\subsection{The Polarizable Continuum Model}
The foundation of Polarizable Continuum Solvation Models (PCSM's) is the assumption that in a solute/solvent system the solvent can be treated as either a dielectric, or a conducting continuum medium on the outside of the molecular cavity $\Omega$ occupied by the solute. We follow the customary approach of taking the cavity to be the so-called Van der Waals cavity\cite{ReviewPCM_2005}, i.e., the union of spheres centered at each atom with radii coinciding with the van der Waals radii.
Within this approach, the topologically similar Solvent Accessible Surface (SAS) cavity can be treated as well. 
Models based on the Solvent Excluded Surface (SES) have recently been proposed~\cite{Harbrecht2011,quan2017polarizable,C5CP03410H,JCC:JCC21431} but are not considered here.

The electrostatic part of the solute/solvent interaction is given by
\[
E_s = \tfrac{1}{2}\, f(\varepsilon)\,\int_\Omega \rho(x) W(x) \, dx
\]
where $f(\varepsilon)$ {\color{red}(shouldn't this be $\varepsilon_s$ rather than $\varepsilon$?)~}is an empirical scaling that depends on the dielectric constant of the solvent (and which is only applied in the case of the COSMO {\color{red}(needs to be clarified)}), $\rho$ is the charge density of the solute, and $W$ is the polarization potential of the solvent. The quantities $W$ and $E_s$ are usually referred to, respectively, as the reaction potential and the electrostatic contribution to the solvation energy. 

The reaction potential is defined as $W = \varphi - \Phi$, where $\varphi$ is the total electrostatic potential of the solute/solvent system and $\Phi$ is the potential of the solute \emph{in vacuo}. In the case of the PCM, the total potential $\varphi$ satisfies a (generalized) Poisson equation with suitable interface conditions\cite{Mennucci_JCP_IEF1,Mennucci_JMC_IEF2}. Indeed, if $\varepsilon_s$ is the macroscopic, zero-frequency relative dielectric permittivity of the solvent, and define $\varepsilon(x) = 1$ when $x \in \Omega$ and $\varepsilon(x) = \varepsilon_s$ otherwise, the reaction potential fulfills 
\begin{equation} 
\label{eq:pcmpde}
\left \{ 
\begin{alignedat}{4}
\Delta  W &= 0  &&\mbox{in } \sR^3\setminus \Gamma  \\
 \ [\![W]\!] &= 0  &&\mbox{on } \Gamma\\
\  [\![\varepsilon \, \partial_\nu W]\!] &= (\varepsilon_s-1)\, \partial_\nu \Phi &\qquad& \mbox{on } \Gamma
\end{alignedat} 
\right.
\end{equation}
Here $\Gamma=\partial\Omega$ is the boundary of the cavity, $\partial_\nu$ is the normal derivative on $\Gamma$, and $[\![\,\cdot\,]\!]$ is the jump operator (inside minus outside) on $\Gamma$.

Recalling potential theory, see, e.g., \cite{sauter2010boundary}, $W$ can be represented as $W(x) = (\tilde{\mathcal{S}}\sigma)(x)$ when $x \in \sR^3 \setminus \Gamma$, or $W(s) = (\mathcal{S}\sigma)(s)$ when $s \in \Gamma$. 
The surface density $\sigma$ defined on $\Gamma$ is the so-called apparent surface charge,  $\tilde{\cS}$ is the single layer potential, and $\cS$ is the single layer operator,  which is invertible \cite{Calderon}. Both $\tilde{\cS}$ and $\cS$ are relative to $\Gamma$.
It can be shown that $\sigma$ satisfies the equation $\sigma = 1/4\pi \, [\![ \partial_\nu W]\!]$, so that it is possible to recast the PCM problem \eqref{eq:pcmpde} as a single integral equation for $\sigma$. In fact, if we define the operators 
\begin{equation}
 \label{eq:Reps}
 \cR_\varepsilon = 2\pi \, \frac{\varepsilon+1}{\varepsilon-1} \, \cI - \cD \qquad, \qquad \cR_\infty = 2\pi \, \cI - \cD
\end{equation}
{\color{red} (shouldn't this be $\varepsilon_s$ rather then $\varepsilon$ in the rhs?)~}{\color{red} (wouldn't it be cleaner to introduce a factor $f_{\varepsilon_s}$ rather than carrying the fraction along?)~}where $\cI$ is the identity and $\cD$ is the double layer operator, also relative to $\Gamma$, it can be shown\cite{ReviewPCM_2005} that the apparent surface charge satisfies
\begin{equation}
\label{eq:IEFPCM}
\cR_\varepsilon \, \cS \, \sigma = - \cR_\infty \, \Phi \qquad \text{on }\Gamma
\end{equation}
This is known as the IEF-PCM equation, and it involves operators $\cR_\infty$ and $\cR_\varepsilon$, which are both invertible. When the dielectric constant $\varepsilon_s$ approaches infinity, the IEF-PCM equation simplifies to $\cS \, \sigma = - \Phi$ on $\Gamma$, which is the Integral Equation Formulation of the Conductor-like Screening Model (COSMO)\cite{Lipparini_JCP_VPCM}.


\subsection{The ddPCM-method}


We recall how to solve the IEF-PCM boundary integral equation \eqref{eq:IEFPCM} within the domain-decomposition paradigm. As a preliminary step, we set $\Phi_\varepsilon = \cS \, \sigma$ and write \eqref{eq:IEFPCM} as a succession of two integral equations, the latter of which is equivalent to the COSMO equation\cite{Cances_Librone_PCM}, namely
\begin{alignat}{3}
\cR_\varepsilon \, \Phi_\varepsilon & = \cR_\infty \, \Phi \qquad && \text{on }\Gamma  \label{eq:ddPCM-1} \\
\cS \, \sigma & = -\Phi_\varepsilon  && \text{on }\Gamma \label{eq:ddPCM-2} 
\end{alignat}
Indeed, \eqref{eq:ddPCM-2} is a COSMO equation with the modified potential $\Phi_\varepsilon$ in place of the potential $\Phi$. This allows to develop the ddPCM strategy as an extension of the ddCOSMO approach. First, equation \eqref{eq:ddPCM-1} is solved in order to compute the right-hand-side $-\Phi_\varepsilon$ of equation \eqref{eq:ddPCM-2}; secondly, ddCOSMO is employed to solve equation \eqref{eq:ddPCM-2} with the modified potential $-\Phi_\varepsilon$, and compute the solvation energy $E_s$.


In order to discuss the domain-decomposition approach employed for both steps, let us introduce some notation. As anticipated, we take the cavity $\Omega$ be the union of $M$ spheres $\Omega_j = B(x_j, r_j)$ with boundaries $\Gamma_j$. We define $\Gamma_j^\text{ext}:= \Gamma_j \cap \Gamma$ and $\Gamma_j^\text{int} := \Gamma_j \cap \Omega$, and let $U_j: \Gamma_j \to \mathbb{R}$ be the characteristic function of $\Gamma_j^\text{ext}$. Next, we define $\Phi_j : \Gamma_j \to \mathbb{R}$ and $\Phi_{\varepsilon,j} : \Gamma_j \to \mathbb{R}$ as the trivial   extension of $\Phi$ resp. $\Phi_\varepsilon$ that vanish on $\Gamma_j^\text{int}$.
Then, it is immediate to conclude that
\begin{equation}\label{eq:16}
(\mathcal{D} \, \Phi ) (s) = ( \mathcal{D}_j \, \Phi_j )(s) + \sum_{k \ne j} \,(\tilde{\mathcal{D}}_k \, \Phi_k )(s) \qquad ; \qquad s \in \Gamma_j^\text{ext} \quad, \quad  j = 1 , \ldots , M
\end{equation}
where $\cD_j$ and $\tilde{\cD}_j$ are, respectively, the local double layer operator and the local double layer potential relative to the local spheres $\Gamma_j$. An analogous result holds for $\Phi_\varepsilon$ and its local extensions $\Phi_{\varepsilon,j}$. We refer to \cite{Stamm_JCP_DDPCM} for further details. %Since the right-hand-side is indeed well-defined for every $s\in \Gamma_j$, we can define the extensions $\widetilde{\mathcal{D} \, \Phi} : \Gamma_j \to \mathbb{R}$ and $\widetilde{\mathcal{D} \, \Phi}_\varepsilon : \Gamma_j \to \mathbb{R}$.

{\bf Step 1.} 
We start to localize the integral equation \eqref{eq:ddPCM-1} to $\Gamma_j$ through the characteristic function $U_j$ as
\begin{equation}\label{eq:20}
2 \pi \, \frac{\varepsilon + 1}{\varepsilon + 1} \, U_j \, \Phi_{\varepsilon} - U_j \, {\mathcal{D} \, \Phi}_\varepsilon = 2 \pi \, U_j \, \Phi - U_j \, {\mathcal{D} \, \Phi} \qquad \text{on }\Gamma_j
\end{equation}
for each $j=1,\ldots,M$. 
%We define $\Phi_j : \Gamma_j \to \mathbb{R}$ as $\Phi_j = U_j\,\Phi$ and, w
Without loss of generality, we trade $U_j \, \Phi_\varepsilon$ for $U_j \, \Phi_{\varepsilon,j}$ in the left-hand-side of~\eqref{eq:20}.
%, where $\Phi_{\varepsilon,j} : \Gamma_j \to \mathbb{R}$ is an extension of the unknown $\Phi_\varepsilon$. 
We ensure that $\Phi_{\varepsilon,j}$ vanishes indeed on $\Gamma_j^\text{int}$ by imposing the additional constraint
\begin{equation}\label{eq:19}
(1 - U_j) \, \Phi_{\varepsilon,j}  = 0\qquad \text{on }\Gamma_j 
\end{equation}
In order to obtain a single equation, we multiply \eqref{eq:19} by the factor $2\pi \, (\varepsilon+1)/(\varepsilon-1)$, and add it sidewise to \eqref{eq:20}, so that 
\begin{equation}\label{eq:21}
2 \pi \, \frac{\varepsilon + 1}{\varepsilon + 1} \, \Phi_{\varepsilon,j} - U_j \, {\mathcal{D} \, \Phi}_\varepsilon = 2 \pi \, \Phi_j - U_j \, {\mathcal{D} \, \Phi} \qquad \text{on }\Gamma_j
\end{equation}
When $s \in \Gamma_j^\text{int}$, we recover $\Phi_{\varepsilon,j}(s) = 0$, so that we have effectively built the constraint \eqref{eq:19} into the previous equation. Thus, recalling that $\Phi_j$ vanishes on $\Gamma_j^\text{int}$ by definition, we proceed to apply the decomposition \eqref{eq:16} to both sides of \eqref{eq:21}, and obtain
%As a next step, we employ the decomposition \eqref{eq:16}, under the hypotheses that $\Phi_{\varepsilon,j}$ and $\Phi_j$ vanish on $\Gamma_j^\text{int}$, which we impose as additional explicit constraints. We obtain the integral equation 
%\begin{multline}\label{eq:17}
%2\pi \, \frac{\varepsilon + 1}{\varepsilon - 1} \, U_j \, \Phi_{\varepsilon,j} - U_j \bigg( \cD_j \, \Phi_{\varepsilon,j} + \sum_{k \ne j} \tilde{\cD}_{k} \, \Phi_{\varepsilon,k}  \bigg) = \\ 2 \pi \, U_j \, \Phi_j - U_j \bigg( \cD_j \, \Phi_j + \sum_{k \ne j} \tilde{\cD}_{k} \, \Phi_{k}  \bigg) \qquad \text{on }\Gamma_j
%\end{multline}
%alogn with the constraints
%\begin{alignat}{2}
%\Phi_j & = U_j \, \widetilde{\Phi} \qquad & \text{on }\Gamma_j \label{eq:18}\\
%(1 - U_j) \, \Phi_{\varepsilon,j} & = 0 &\text{on }\Gamma_j \label{eq:19}
%\end{alignat}
%In order to obtain a single equation, we insert \eqref{eq:18} into the right-hand-side of \eqref{eq:17}, multiply \eqref{eq:19} by the factor $2\pi \, (\varepsilon+1)/(\varepsilon-1)$ and add it sidewise to \eqref{eq:17}.
 %This yields
\begin{multline}\label{eq:1}
2\pi \, \frac{\varepsilon + 1}{\varepsilon - 1} \, \Phi_{\varepsilon,j} - U_j \bigg( {\mathcal{D}}_j \, \Phi_{\varepsilon,j} + \sum_{k \ne j} \, \tilde{\mathcal{D}}_{k} \, \Phi_{\varepsilon,k}  \bigg) = \\ 2 \pi \, {\Phi_j} - U_j \bigg( {\mathcal{D}}_j \,\Phi_{j} + \sum_{k \ne j} \, \tilde{\mathcal{D}}_{k} \, \Phi_{k}  \bigg) \qquad \text{on }\Gamma_j
\end{multline}
which constitutes our domain-decomposition strategy for equation \eqref{eq:ddPCM-1}. The summation implies that every subdomain $\Omega_j$ interacts with all other subdomains. We anticipate that this contrasts with the ddCOSMO strategy for equation \eqref{eq:ddPCM-2}, which only involves neighbor-to-neighbor interactions.

{\bf Step 2.} 
We note that $W = \tilde{\mathcal{S}}\sigma$, where $\sigma$ solves the integral equation \eqref{eq:ddPCM-2}, solves the equation $\Delta W = 0$ in $\Omega$ with boundary condition $W=-\Phi_\varepsilon$ on $\Gamma=\partial \Omega$. 
The restriction $W_j := W |_{\overline{\Omega}_j}$ is harmonic over the subdomain $\Omega_j$, thus it can be represented  locally as 
\begin{equation}\label{eq:COSMOloc}
W_j(x) = (\tilde{\mathcal{S}}_j \,  \sigma_j) (x) \quad , \quad x \in \Omega_j \qquad ; \qquad
W_j(s) = (\mathcal{S}_j \,  \sigma_j) (s) \quad , \quad s \in \Gamma_j
\end{equation}
where $\sigma_j$ is an unknown surface density on $\Gamma_j$, and $\cS_j$ and $\tilde{\cS}_j$ are, respectively, the single layer potential and the single layer operator on $\Gamma_j$. The local problems \eqref{eq:COSMOloc} are coupled together by imposing on $\Gamma_j$ the coupling condition
%decomposing $W_j$ as
\begin{equation}\label{eq:5}
W_j(s) = - \Phi_{\varepsilon,j}(s) +  n_j(s) \, \sum_{k \in N_j} \,{W}_k(s) \qquad ; \qquad s \in \Gamma_j \quad , \quad j = 1, \ldots , M
\end{equation}
where $N_j$ is the set of all neighboring subdomains of $\Omega_j$, $W_k$ is understood as its trivial extension to $\Omega$, and $n_j$ is a normalization factor. If $s$ does not belong to any neighbor of $\Omega_j$, then $n_j(s)$ vanishes. Otherwise, $n_j(s)$ is the reciprocal of the number of neighbors. The decomposition \eqref{eq:5} also employs the fact that $\Phi_{\varepsilon,j}$ vanishes on $\Gamma_j^\text{int}$. When we substitute the local problems \eqref{eq:COSMOloc} into the decomposition \eqref{eq:5}, and define $(\tilde{\cS}_{jk} \, \sigma_k) (s) = n_j(s) \, (\tilde{\cS}_k \, \sigma_k)(s)$, we obtain
\begin{equation}\label{eq:2}
\mathcal{S}_j \, \sigma_j  -  \sum_{k \in N_j} \, \tilde{\mathcal{S}}_{jk} \, \sigma_k = -  \Phi_{\varepsilon,j} \qquad \text{on } \Gamma_j
\end{equation}
As opposed to the local problem \eqref{eq:1} which features a global interaction of all subdomains, the ddCOSMO step \eqref{eq:2} is characterized by the interaction of subdomain $\Omega_j$ with only its neighbors. This results in a sparse, rather than dense, discrete operator. We anticipate that this is the key feature that allows to compute the ddCOSMO forces within linear complexity, with respect to the number of atoms $M$. More details are provided in Section \ref{sec:forces}.

\subsection{Numerical Discretization}

The first step to discretize equation \eqref{eq:1} and \eqref{eq:2} is to expand $\Phi_j$, $\Phi_{\varepsilon,j}$ and $\sigma_j$ as truncated series of spherical harmonics. Let $Y_\ell^m$ be the spherical harmonic of degree $\ell$ and order $m$ on the unit sphere $\mathbb{S}$, and let $y$ be the variable on $\mathbb{S}$. For a prescribed integer parameter ${L_\text{max}}$,  we approximate the surface charge $\sigma_j$ as the truncated expansion
\[
\sigma_j(s) = \sigma_j(x_j + r_j y) = \frac{1}{r_j} \, \sum_{\ell=0}^{L_\text{max}} \sum_{m = -\ell}^\ell \, [X_j]_\ell^m \, Y_\ell^m(y)
\]
for some unknown coefficients $X = [X_j]_\ell^m$. The scaling factor has been introduced for convenience, see the derivation in Appendix \ref{app:cosmo}. {\color{red} (comment why this ansatz is legit)} We approximate $\Phi_{\varepsilon,j}$ and $\Phi_j$ in the same fashion, namely
\[
\Phi_{\varepsilon,j}(s) = - \sum_{\ell=0}^{L_\text{max}} \sum_{m = -\ell}^\ell \, [G_j]_\ell^m \, Y_\ell^m(y) \qquad , \qquad \Phi_j(s) = -\sum_{\ell=0}^{L_\text{max}} \sum_{m = -\ell}^\ell \, [F_j]_\ell^m \, Y_\ell^m(y)
\]
where $G = [G_j]_\ell^m$ and $F = [F_j]_\ell^m$ are the coefficients of the expansions. The reason for introducing the minus signs is to eliminate the negative sign in the discretization of the right-hand-side of the ddCOSMO step \eqref{eq:2}. 


We interpret each local problem (\ref{eq:1}) and (\ref{eq:2}) in a variational setting that uses spherical harmonics as test functions, and employ orthogonality conditions of the spherical harmonics, along with Lebedev grids to perform numerical quadrature, to derive discretizations of the global problems \eqref{eq:ddPCM-1} and \eqref{eq:ddPCM-2}. Full details of the derivations are provided in Appendix \ref{app:pcm} and \ref{app:cosmo}. Respectively, we obtain the linear systems
\begin{equation}\label{eq:6}
A_\varepsilon \, G = A_\infty \, F \qquad , \qquad  L \, X = G
\end{equation}
where the entries of $A_\varepsilon$ and $A_\infty$ are given in closed form in \eqref{eq:ajj} and \eqref{eq:ajk}, and those of $L$ are given in \eqref{eq:61} and \eqref{eq:62}.

The operators $A_\varepsilon$, $A_\infty$, and $L$ clearly depend upon the nuclear positions. Although more subtle, the load vector $F$ depends on the nuclear positions as well. Indeed, as discussed in details in Appendix \ref{app:pcm}, the coefficients $[F_j]_\ell^m$ are related to the potential as
\begin{equation}\label{eq:25}
[F_j]_\ell^m = - \int_{\mathbb{S}} \Phi_j(s(y)) \, Y_\ell^m(y) \,dy = - \int_{\mathbb{S}} U_j(s(y)) \, \Phi(s(y)) \, Y_\ell^m(y) \,dy
\end{equation}
and are evaluated through numerical quadrature. Since the characteristic function $U_j$ depends upon $x_j$ and $x_i$ such that $i \in N_j$, so does $F_j = [F_j]_\ell^m$. Those dependencies upon the nuclear positions are need in the computation of the ddPCM-forces, discussed in the following Section.


%As customary, in order to obtain the ddPCM and ddCOSMO discretizations we interpret each local problem (\ref{eq:1}) and (\ref{eq:2}) in a variational setting that employs expansions of $\Phi_j$, $\Phi_{\varepsilon,j}$ and $\sigma_j$ as truncated series of spherical harmonics, as well as spherical harmonics as test functions. We briefly describe the discretization of the ddCOSMO problem \eqref{eq:2}, and refer to Appendix \ref{app:mats} for the more involved discretization of the ddPCM problem \eqref{eq:1}.
%
%
%
%The PCM equation \eqref{eq:1} is discretized in a similar fashion, by expanding $\hat{\Phi}_j$ and $\hat{\Phi}_{\varepsilon,j}$ as truncated series of spherical harmonics
%\[
%\hat{\Phi}_{\varepsilon,j}(y) = - \sum_{\ell'=0}^{L_\text{max}} \sum_{m' = -\ell'}^{\ell'} [G_j]_{\ell'}^{m'} \, Y_{\ell'}^{m'}(y) \qquad , \qquad \hat{\Phi}_j(y) = -\sum_{\ell'=0}^{L_\text{max}} \sum_{m' = -\ell'}^{\ell'} [F_j]_{\ell'}^{m'} \, Y_{\ell'}^{m'}(y)
%\]
%where the minus signs have been introduced for convenience. We refer to Appendix \ref{app:mats} for the full details of the numerical discretization.
%The final discretizations of the global problems \eqref{eq:ddPCM-1} and \eqref{eq:ddPCM-2} are, respectively
%\begin{equation}\label{eq:6}
%A_\varepsilon \, G = A_\infty \, F \qquad , \qquad  L \, X = G
%\end{equation}
%where $G = [G_j]_{\ell'}^{m'}$,  $F = [F_j]_{\ell'}^{m'}$ and the expressions for the entries of the discrete operator $A_\varepsilon$ are given in \eqref{eq:ajj} and \eqref{eq:ajk}.
