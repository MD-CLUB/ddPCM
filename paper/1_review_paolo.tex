\section{A Brief Review of the ddPCM Strategy}\label{sec:review}

\subsection{The Polarizable Continuum Model}
The foundation of Polarizable Continuum Solvation Models (PCSM's) is the assumption that the solvent in a solute-solvent system can be treated as either a dielectric, or a conducting continuum medium on the outside of the molecular cavity $\Omega$ of the solute. We follow the customary approach of taking the cavity to be the so-called Van der Waals cavity\cite{ReviewPCM_2005}, i.e., the union of spheres centered at each atom with radii coinciding with the van der Waals radii.
Within this approach, the topologically similar Solvent Accessible Surface (SAS) cavity can be treated as well. 
Models based on the Solvent Excluded Surface (SES) have recently been proposed~\cite{Harbrecht2011,quan2017polarizable,C5CP03410H,JCC:JCC21431} but are not considered here.

The electrostatic part of the solute-solvent interaction is given by $E_s = \tfrac{1}{2}\, f(\varepsilon)\,\int_\Omega \rho(x) W(x) \, dx$, where $f(\varepsilon)$ is an empirical scaling that depends on the dielectric constant of the solvent (and which is only applied in the case of the COSMO), $\rho$ is the charge density of the solute, and $W$ is the polarization potential of the solvent. The quantities $W$ and $E_s$ are usually referred to, respectively, as the reaction potential and the electrostatic contribution to the solvation energy. 

The reaction potential is defined as $W = \varphi - \Phi$, where $\varphi$ is the total electrostatic potential of the solute-solvent system and $\Phi$ is the potential of the solute \emph{in vacuo}. In the case of the PCM, the total potential $\varphi$ satisfies a (generalized) Poisson equation with suitable interface conditions\cite{Mennucci_JCP_IEF1,Mennucci_JMC_IEF2}. Indeed, if $\varepsilon_s$ is the macroscopic, zero-frequency relative dielectric permittivity of the solvent, and define $\varepsilon(x) = 1$ when $x \in \Omega$ and $\varepsilon(x) = \varepsilon_s$ otherwise, the reaction potential fulfills 
\begin{equation} 
\label{eq:pcmpde}
\left \{ 
\begin{alignedat}{4}
\Delta  W &= 0  &&\mbox{in } \sR^3\setminus \Gamma  \\
 \ [W] &= 0  &&\mbox{on } \Gamma\\
\  [\varepsilon \, \partial_\nu W] &= (\varepsilon_s-1) \partial_\nu \Phi &\qquad& \mbox{on } \Gamma
\end{alignedat} 
\right.
\end{equation}
Here $\Gamma=\partial\Omega$ is the boundary of the cavity, $\partial_\nu$ is the normal derivative on $\Gamma$, and $[\,\cdot\,]$ is the jump operator (inside minus outside) on $\Gamma$.

Recalling potential theory, see for example\cite{sauter2010boundary}, $W$ can be represented as $W(x) = (\tilde{\mathcal{S}}\sigma)(x)$ when $x \in \sR^3 \setminus \Gamma$, or $W(s) = (\mathcal{S}\sigma)(s)$ when $s \in \Gamma$. 
The surface density $\sigma$ defined on $\Gamma$ is the so-called apparent surface charge,  $\tilde{\cS}$ is the single layer potential and $\cS$ is the single layer operator,  which is invertible \cite{Calderon}. 
Note that both $\tilde{\cS}$ and $\cS$ are based on the surface $\Gamma$.
It can be shown that $\sigma$ satisfies the equation $\sigma = 1/4\pi \, [ \partial_\nu W]$, so that it is possible to recast the PCM problem \eqref{eq:pcmpde} as an integral equation for $\sigma$. In fact, if we define the operators 
\begin{equation}
 \label{eq:Reps}
 \cR_\varepsilon = 2\pi \frac{\varepsilon+1}{\varepsilon-1} \, \cI - \cD \qquad, \qquad \cR_\infty = 2\pi \, \cI - \cD
\end{equation}
where $\cI$ is the identity and $\cD$ is the double layer boundary operator (also based on $\Gamma$).
It can be shown\cite{ReviewPCM_2005} that the apparent surface charge satisfies
\begin{equation}
\label{eq:IEFPCM}
\cR_\varepsilon \, \cS \, \sigma = - \cR_\infty \, \Phi \qquad \text{on }\Gamma
\end{equation}
which is known as the IEF-PCM equation. It involves operators $\cR_\infty$ and $\cR_\varepsilon$, which are both invertible. Furthermore, when the dielectric constant $\varepsilon_s$ approaches infinity, the IEF-PCM equation simplifies to $\cS \, \sigma = - \Phi$ on $\Gamma$, which is the Integral Equation Formulation of the Conductor-like Screening Model (COSMO)\cite{Lipparini_JCP_VPCM}.


\subsection{The ddPCM-method}


Let us recall how to solve equation \eqref{eq:IEFPCM} within the domain-decomposition paradigm. The first step is to write the IEF-PCM integral equation \eqref{eq:IEFPCM} as a succession of two integral equations, one of which is equivalent to the COSMO equation\cite{Cances_Librone_PCM}. Indeed, if we define $\Phi_\varepsilon = \cS \, \sigma$, equation \eqref{eq:IEFPCM} becomes
\begin{alignat}{3}
\cR_\varepsilon \, \Phi_\varepsilon & = \cR_\infty \, \Phi \qquad && \text{on }\Gamma  \label{eq:ddPCM-1} \\
\cS \, \sigma & = -\Phi_\varepsilon  && \text{on }\Gamma \label{eq:ddPCM-2} 
\end{alignat}
The ddPCM strategy is an extension of ddCOSMO in the following sense: first, equation \eqref{eq:ddPCM-1} is solved in order to compute the right-hand side $-\Phi_\varepsilon$ of equation \eqref{eq:ddPCM-2}; secondly, ddCOSMO is employed to solve equation \eqref{eq:ddPCM-2} with the modified potential $-\Phi_\varepsilon$, and compute the solvation energy $E_s$.


In order to discuss the domain-decomposition approach employed for both steps, let us introduce some notation. As anticipated, we take the cavity $\Omega$ be the union of $M$ spheres $\Omega_j = B(x_j, r_j)$ with boundaries $\Gamma_j$. We define $\Gamma_j^\text{ext}:= \Gamma_j \cap \Gamma$ and $\Gamma_j^\text{int} := \Gamma_j \cap \Omega$, and let $U_j: \Gamma_j \to \mathbb{R}$ be the characteristic function of $\Gamma_j^\text{ext}$. 

Next, we define $\Phi_j : \Gamma_j \to \mathbb{R}$ and $\Phi_{\varepsilon,j} : \Gamma_j \to \mathbb{R}$ as the trivial   extension of $\Phi$ resp. $\Phi_\varepsilon$ that vanish on $\Gamma_j^\text{int}$.
Then, it is immediate to conclude that
\begin{equation}\label{eq:16}
(\mathcal{D} \, \Phi ) (s) = ( \mathcal{D}_j \, \Phi_j )(s) + \sum_{k \ne j} (\tilde{\mathcal{D}}_k \, \Phi_k )(s) \qquad ; \qquad s \in \Gamma_j^\text{ext} \quad, \quad  j = 1 , \ldots , M
\end{equation}
where $\cD_j$ and $\tilde{\cD}_j$ denote, respectively, the local double layer operator and the local double layer potential over the local spheres $\Gamma_j$. An analogous result holds for $\Phi_\varepsilon$ and its local extensions $\Phi_{\varepsilon,j}$. 
Details can be found in\cite{Stamm_JCP_DDPCM}.
%Since the right-hand-side is indeed well-defined for every $s\in \Gamma_j$, we can define the extensions $\widetilde{\mathcal{D} \, \Phi} : \Gamma_j \to \mathbb{R}$ and $\widetilde{\mathcal{D} \, \Phi}_\varepsilon : \Gamma_j \to \mathbb{R}$.

{\bf Step 1.} 
We start to localize the integral equation \eqref{eq:ddPCM-1} to $\Gamma_j$ through the characteristic function $U_j$ as
\begin{equation}\label{eq:20}
2 \pi \, \frac{\varepsilon + 1}{\varepsilon + 1} \, U_j \, \Phi_{\varepsilon} - U_j \, {\mathcal{D} \, \Phi}_\varepsilon = 2 \pi \, U_j \, \Phi - U_j \, {\mathcal{D} \, \Phi} \qquad \text{on }\Gamma_j
\end{equation}
for each $j=1,\ldots,M$. 
%We define $\Phi_j : \Gamma_j \to \mathbb{R}$ as $\Phi_j = U_j\,\Phi$ and, w
Without loss of generality, we trade $U_j \, \Phi_\varepsilon$ for $U_j \, \Phi_{\varepsilon,j}$ in the left-hand-side of~\eqref{eq:20}.
%, where $\Phi_{\varepsilon,j} : \Gamma_j \to \mathbb{R}$ is an extension of the unknown $\Phi_\varepsilon$. 
We ensure that $\Phi_{\varepsilon,j}$ vanishes indeed on $\Gamma_j^\text{int}$ by imposing the additional constraint
\begin{equation}\label{eq:19}
(1 - U_j) \, \Phi_{\varepsilon,j}  = 0\qquad \text{on }\Gamma_j 
\end{equation}
In order to obtain a single equation, we multiply \eqref{eq:19} by the factor $2\pi \, (\varepsilon+1)/(\varepsilon-1)$, and add it sidewise to \eqref{eq:20}, so that 
\begin{equation}\label{eq:21}
2 \pi \, \frac{\varepsilon + 1}{\varepsilon + 1} \, \Phi_{\varepsilon,j} - U_j \, {\mathcal{D} \, \Phi}_\varepsilon = 2 \pi \, \Phi_j - U_j \, {\mathcal{D} \, \Phi} \qquad \text{on }\Gamma_j
\end{equation}
When $s \in \Gamma_j^\text{int}$, we recover $\Phi_{\varepsilon,j}(s) = 0$, so that we have effectively built the constraint \eqref{eq:19} into the previous equation. Thus, recalling that $\Phi_j$ vanishes on $\Gamma_j^\text{int}$ by definition, we proceed to apply the decomposition \eqref{eq:16} to both sides of \eqref{eq:21}, and obtain
%As a next step, we employ the decomposition \eqref{eq:16}, under the hypotheses that $\Phi_{\varepsilon,j}$ and $\Phi_j$ vanish on $\Gamma_j^\text{int}$, which we impose as additional explicit constraints. We obtain the integral equation 
%\begin{multline}\label{eq:17}
%2\pi \, \frac{\varepsilon + 1}{\varepsilon - 1} \, U_j \, \Phi_{\varepsilon,j} - U_j \bigg( \cD_j \, \Phi_{\varepsilon,j} + \sum_{k \ne j} \tilde{\cD}_{k} \, \Phi_{\varepsilon,k}  \bigg) = \\ 2 \pi \, U_j \, \Phi_j - U_j \bigg( \cD_j \, \Phi_j + \sum_{k \ne j} \tilde{\cD}_{k} \, \Phi_{k}  \bigg) \qquad \text{on }\Gamma_j
%\end{multline}
%alogn with the constraints
%\begin{alignat}{2}
%\Phi_j & = U_j \, \widetilde{\Phi} \qquad & \text{on }\Gamma_j \label{eq:18}\\
%(1 - U_j) \, \Phi_{\varepsilon,j} & = 0 &\text{on }\Gamma_j \label{eq:19}
%\end{alignat}
%In order to obtain a single equation, we insert \eqref{eq:18} into the right-hand-side of \eqref{eq:17}, multiply \eqref{eq:19} by the factor $2\pi \, (\varepsilon+1)/(\varepsilon-1)$ and add it sidewise to \eqref{eq:17}.
 %This yields
\begin{multline}\label{eq:1}
2\pi \, \frac{\varepsilon + 1}{\varepsilon - 1} \, \Phi_{\varepsilon,j} - U_j \bigg( {\mathcal{D}}_j \, \Phi_{\varepsilon,j} + \sum_{k \ne j} \tilde{\mathcal{D}}_{k} \, \Phi_{\varepsilon,k}  \bigg) = \\ 2 \pi \, {\Phi_j} - U_j \bigg( {\mathcal{D}}_j \,\Phi_{j} + \sum_{k \ne j} \tilde{\mathcal{D}}_{k} \, \Phi_{k}  \bigg) \qquad \text{on }\Gamma_j
\end{multline}
which constitutes our domain-decomposition strategy for equation \eqref{eq:ddPCM-1}. It is important to remark that, because of the summation, every subdomain $\Omega_j$ interacts with all other subdomains. We anticipate that this contrasts with the ddCOSMO strategy for equation \eqref{eq:ddPCM-2}.

{\bf Step 2.} 
We note that $W = \tilde{\mathcal{S}}\sigma$, where $\sigma$ solves the integral equation \eqref{eq:ddPCM-2}, solves the equation $\Delta W = 0$ in $\Omega$ with boundary condition $W=-\Phi_\varepsilon$ on $\Gamma=\partial \Omega$. 
The restriction $W_j := W |_{\overline{\Omega}_j}$ is harmonic over the subdomain $\Omega_j$, thus it can be represented  locally as 
\begin{equation}\label{eq:COSMOloc}
W_j(x) = (\tilde{\mathcal{S}}_j \,  \sigma_j) (x) \quad , \quad x \in \Omega_j \qquad ; \qquad
W_j(s) = (\mathcal{S}_j \,  \sigma_j) (s) \quad , \quad s \in \Gamma_j
\end{equation}
where $\sigma_j$ is an unknown surface density on $\Gamma_j$, and $\cS_j$ and $\tilde{\cS}_j$ are, respectively, the single layer potential and the single layer operator on $\Gamma_j$. The local problems \eqref{eq:COSMOloc} are coupled together by imposing on $\Gamma_j$ the coupling condition
%decomposing $W_j$ as
\begin{equation}\label{eq:5}
W_j(s) = - \Phi_{\varepsilon,j}(s) +  n_j(s) \, \sum_{k \in N_j} {W}_k(s) \qquad ; \qquad s \in \Gamma_j \quad , \quad j = 1, \ldots , M
\end{equation}
where $N_j$ is the set of all neighboring subdomains of $\Omega_j$, $W_k$ is understood as its trivial extension to $\Omega$, and $n_j$ is a normalization factor. If $s$ does not belong to any neighbor of $\Omega_j$, then $n_j(s)$ vanishes. Otherwise, $n_j(s)$ is the reciprocal of the number of neighbors. The decomposition \eqref{eq:5} also employs the fact that $\Phi_{\varepsilon,j}$ vanishes on $\Gamma_j^\text{int}$. When we substitute the local problems \eqref{eq:COSMOloc} into the decomposition \eqref{eq:5}, and define $(\tilde{\cS}_{jk} \, \sigma_k) (s) = n_j(s) \, (\tilde{\cS}_k \, \sigma_k)(s)$, we obtain
\begin{equation}\label{eq:2}
\mathcal{S}_j \, \sigma_j  -  \sum_{k \in N_j} \tilde{\mathcal{S}}_{jk} \, \sigma_k = -  \Phi_{\varepsilon,j} \qquad \text{on } \Gamma_j
\end{equation}
As opposed to the local problem \eqref{eq:1} which features a global interaction of all subdomains, the ddCOSMO step \eqref{eq:2} is characterized by the interaction of subdomain $\Omega_j$ with only its neighbors. This results in a sparse, rather than dense, discrete operator.

\subsection{Numerical Discretization}

We discretize equation \eqref{eq:1} and \eqref{eq:2} by expanding $\Phi_j$, $\Phi_{\varepsilon,j}$ and $\sigma_j$ as truncated series of spherical harmonics. If $Y_\ell^m$ indicates the spherical harmonic of degree $\ell$ and order $m$ on the unit sphere $\mathbb{S}$, we approximate the surface charge $\sigma_j$ as
\[
\sigma_j(s) = \sigma_j(x_j + r_j y) = \frac{1}{r_j}\sum_{\ell=0}^{L_\text{max}} \sum_{m = -\ell}^\ell [X_j]_\ell^m \, Y_\ell^m(y)
\]
for some unknown coefficients $X = [X_j]_\ell^m$ and a prescribed integer parameter ${L_\text{max}}$. Here $y$ is the variable on $\mathbb{S}$. %{\bf (Discuss why this ansatz is legit.)} 
We approximate $\Phi_{\varepsilon,j}$ and $\Phi_j$ in a similar fashion, namely
\[
\Phi_{\varepsilon,j} = - \sum_{\ell=0}^{L_\text{max}} \sum_{m = -\ell}^\ell [G_j]_\ell^m \, Y_\ell^m \qquad , \qquad \Phi_j = -\sum_{\ell=0}^{L_\text{max}} \sum_{m = -\ell}^\ell [F_j]_\ell^m \, Y_\ell^m
\]
where $G = [G_j]_\ell^m$ and $F = [F_j]_\ell^m$ are the coefficients of the expansions, and the minus signs have been introduced for convenience. In the following, we shall use the condensed notation $\sum_{\ell ,m}$ to indicate the double sum. The coefficients $[F_j]_\ell^m$ are related to the potential as
\begin{equation}\label{eq:25}
[F_j]_\ell^m = - \int_{\Gamma_j} \Phi_j(s) \, Y_\ell^m(y_j(s)) \,ds = - \int_{\Gamma_j} U_j(s) \, \Phi(s) \, Y_\ell^m(y_j(s)) \,ds,
\end{equation}
with $y_j(s)=\tfrac{s-x_j}{r_j}$, 
and are evaluated through numerical quadrature. We remark that $F$ does depend on the characteristic functions and, thus, upon the nuclear position of the neighboring spheres. We interpret each local problem (\ref{eq:1}) and (\ref{eq:2}) in a variational setting that uses spherical harmonics as test functions, see Appendix \ref{app:mats}. We employ orthogonality conditions of the spherical harmonics, along with Lebedev grids to perform numerical quadrature to derive discretizations of the global problems \eqref{eq:ddPCM-1} and \eqref{eq:ddPCM-2}. Respectively, we obtain
\begin{equation}\label{eq:6}
A_\varepsilon \, G = A_\infty \, F \qquad , \qquad  L \, X = G
\end{equation}
and the expressions for the entries of the discrete operator $A_\varepsilon$ are given in \eqref{eq:ajj} and \eqref{eq:ajk}.


%As customary, in order to obtain the ddPCM and ddCOSMO discretizations we interpret each local problem (\ref{eq:1}) and (\ref{eq:2}) in a variational setting that employs expansions of $\Phi_j$, $\Phi_{\varepsilon,j}$ and $\sigma_j$ as truncated series of spherical harmonics, as well as spherical harmonics as test functions. We briefly describe the discretization of the ddCOSMO problem \eqref{eq:2}, and refer to Appendix \ref{app:mats} for the more involved discretization of the ddPCM problem \eqref{eq:1}.
%
%
%
%The PCM equation \eqref{eq:1} is discretized in a similar fashion, by expanding $\hat{\Phi}_j$ and $\hat{\Phi}_{\varepsilon,j}$ as truncated series of spherical harmonics
%\[
%\hat{\Phi}_{\varepsilon,j}(y) = - \sum_{\ell'=0}^{L_\text{max}} \sum_{m' = -\ell'}^{\ell'} [G_j]_{\ell'}^{m'} \, Y_{\ell'}^{m'}(y) \qquad , \qquad \hat{\Phi}_j(y) = -\sum_{\ell'=0}^{L_\text{max}} \sum_{m' = -\ell'}^{\ell'} [F_j]_{\ell'}^{m'} \, Y_{\ell'}^{m'}(y)
%\]
%where the minus signs have been introduced for convenience. We refer to Appendix \ref{app:mats} for the full details of the numerical discretization.
%The final discretizations of the global problems \eqref{eq:ddPCM-1} and \eqref{eq:ddPCM-2} are, respectively
%\begin{equation}\label{eq:6}
%A_\varepsilon \, G = A_\infty \, F \qquad , \qquad  L \, X = G
%\end{equation}
%where $G = [G_j]_{\ell'}^{m'}$,  $F = [F_j]_{\ell'}^{m'}$ and the expressions for the entries of the discrete operator $A_\varepsilon$ are given in \eqref{eq:ajj} and \eqref{eq:ajk}.
