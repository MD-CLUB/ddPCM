\section{Computation of Forces}\label{sec:forces}
%In the case of a classical solute, the form of the charge density $\rho$ allows to compute 
The solvation energy can be written as a sum of subdomain contributions, which perfectly fits the ddCOSMO paradigm. 
Indeed, the spherical harmonics addition theorem implies that
\[
W_j(x) = (\tilde{\cS}_j \, \sigma_j)(x) = \sum_{\ell,m} [X_j]_\ell^m (\tilde{\cS}_j \, Y_\ell^m)(x) = \sum_{\ell,m} [X_j]_\ell^m \, \frac{4\pi}{2\ell+1} \,\frac{r^\ell}{r_j^{\ell+1}} \, Y_\ell^m(y)
\]
where $x = x_j + r\, y$, so that the solvation energy can be determined as
\[
E_s = \tfrac{1}{2} 
%\,f(\varepsilon) 
\, \sum_{j=1}^M \int_{\Omega_j} \rho(x) W_j(x) \, dx = \tfrac{1}{2}
%\, f(\varepsilon) 
\, \sum_{j=1}^M \sum_{\ell,m} [X_j]_\ell^m \, \frac{4\pi}{2\ell+1} \,\frac{1}{r_j^{\ell+1}} 
\int_{\Omega_j} \rho(x) \, r^\ell \, Y_\ell^m(y) \, dx
\]
If we define
\[
[\Psi_j]_\ell^m = \frac{4\pi}{2\ell+1}\, \frac{1}{r_j^{\ell+1}}\int_{\Omega_j} \rho(x) \, r^\ell \, Y_\ell^m(y) \, dx
\]
we can compactly write the energy as
\[
E_s = \tfrac{1}{2}
% \,f(\varepsilon)
 \, \sum_j \sum_{\ell,m} [\Psi_j]_\ell^m [X_j]_\ell^m
  =: \tfrac{1}{2} 
  %\,f(\varepsilon) 
  \,\langle \Psi, X \rangle
\]
where the angular brackets indicate the double scalar product over $j$ and $\ell,m$.

The force acting on the $i$-th particle can be computed as
\[
\mathcal{F}_i = -\nabla E_s = - \tfrac{1}{2} \,f(\varepsilon) \,  \langle \Psi, \nabla X \rangle % = - \langle \Psi, \nabla_j\sigma \rangle
\]
where the gradient is understood with respect to $x_i$. Here we used the fact that $\Psi$ is independent of the atomic positions. On the other hand, since both the COSMO operator and right-hand side depend on $x_1 , \ldots, x_M$, so does the unknown $X$. The idea is consider the adjoint problem $(A_\varepsilon \, L)^* s = \Psi$ and compute the quantity $\langle \Psi, \nabla X \rangle = \langle s ,  A_\varepsilon \, L \, \nabla X\rangle$ through an integration-by-parts-like approach.

If combine equations \eqref{eq:6}, the fully discretized problem becomes $A_\varepsilon \, L \, X = A_\infty \, F$, and Leibnitz differentiation rule allows to move derivatives from $X$ onto the other terms, namely
\[
A_\varepsilon \, L \, \nabla X = \nabla A_\infty \, F +  A_\infty \, \nabla F - \nabla A_\varepsilon \, L \, X -  A_\varepsilon \, \nabla L \, X=: h
\]
Thus, once the solution $s$ of the adjoint problem and vector $h$ have been determined, the forces can be computed as
\[
\mathcal{F}_i =  - \tfrac{1}{2} \,f(\varepsilon) \,  \langle s, h \rangle
\]
The derivatives $\nabla L $ of the ddCOSMO discretization were discussed in \cite{}. The quantity $\nabla F$ is \emph{a priori} nonzero since $F_j$ is the discretization of $\Phi_j = U_j \, \tilde{\Phi}$. In remainder of this section we discuss the derivatives of the ddPCM matrix.

Let $\{ s_n\}$ be the $N_\text{grid}$ Lebedev integration points and define the following quantities
\[
t_n^{jk} = \frac{|x_j + r_j s_n -x_k|}{r_k} \quad , \quad s_n^{jk} = \frac{x_j + r_j s_n -x_k}{|x_j + r_j s_n -x_k|}\quad , \quad U_j^n = U_j(x_j + r_j s_n)
\]
The blocks $A_{jk}^\varepsilon$ of the ddPCM matrix $A_\varepsilon$, see \eqref{eq:ajj} and \eqref{eq:ajk}, have the form
\begin{alignat*}{3}
{[A_{jj}^\varepsilon]}_{\ell \ell'}^{mm'}& = 2\pi \, \frac{\varepsilon + 1}{\varepsilon - 1}\, \delta_{\ell \ell'} \delta_{m m'}&& + \frac{2\pi}{2 \ell' + 1} \,\sum_{n= 1}^{N_\text{grid}} w_n \, U_j^n  \,Y_\ell^m(s_n) \,  Y_{\ell'}^{m'}(s_n) \\
{[A_{jk}^\varepsilon]}_{\ell \ell'}^{mm'}& =&& -  \frac{4 \pi \ell'}{2 \ell'+1} \, \sum_{n= 1}^{N_\text{grid}} w_n\, U_j^n  \, Y_\ell^m(s_n) \, \big( t_n^{jk}\big)^{-(\ell'+1)} \, Y_{\ell'}^{m'} (s_n^{jk})
\end{alignat*}
where $\{ w_n\}$ are the weights associated to the integration points. Since the derivatives are independent of $\varepsilon$, we drop the $\varepsilon$-dependency for ease of notation.

The case of the diagonal blocks yields
\[
{[\nabla A_{jj}]}_{\ell \ell'}^{mm'} = \frac{2\pi}{2 \ell' + 1} \,\sum_{n} w_n \, \nabla U_j^n  \,Y_\ell^m(s_n) \,  Y_{\ell'}^{m'}(s_n)
\]
so that it only requires the derivatives of the characteristic function. The function $U_j$ is, in practice, a smoothed version of the (discontinuous!)~characteristic function, and is defined as 
\[
U_j(x_j + r_j y) =
\begin{cases}
1 - f_j(y) 	&\quad f_j(y) \le 1\\
0		&\quad \text{otherwise}
\end{cases}
\qquad , \qquad 
f_j(y) = \sum_{k \in N_j} \chi \bigg(\frac{|x_j + r_j y - x_k|}{r_k}\bigg)
\]
where $y$ varies on $\mathbb{S}^2$ and $\chi$ is a regularized characteristic function of $[0,1]$. We conclude that $\nabla U_j$ and, consequently, $\nabla A_{jj}$ are \emph{a priori} nonzero only when $i \in N_j$ or $i = j$.

The case of the off-diagonal blocks, i.e., $j \not=k$, is more involved since it includes the gradient of the product of three functions, namely
\begin{equation}\label{eq:7}
{[\nabla A_{jk}]}_{\ell \ell'}^{mm'} = -  \frac{4 \pi \ell'}{2 \ell'+1} \, \sum_{n} w_n\, Y_\ell^m(s_n) \,\nabla \Big[ U_j^n  \,  \big( t_n^{jk}\big)^{-(\ell'+1)} \, Y_{\ell'}^{m'} (s_n^{jk}) \Big]
\end{equation}
%\cdots\Big[ \nabla U_j^n  \,  \big( t_n^{jk}\big)^{-(\ell'+1)} \, Y_{\ell'}^{m'} (s_n^{jk}) + U_j^n  \, \nabla \Big( \big( t_n^{jk}\big)^{-(\ell'+1)} \Big)\, Y_{\ell'}^{m'} (s_n^{jk}) + U_j^n  \,  \big( t_n^{jk}\big)^{-(\ell'+1)} \, \nabla Y_{\ell'}^{m'} (s_n^{jk}) \Big]
%\end{multline*}
However, since $t_n^{jk}$ and $s_n^{jk}$ depend only upon $x_j$ and $x_k$, if we assume $i \not=j$ and $i\not=k$, we obtain
\begin{equation}\label{eq:8}
{[\nabla A_{jk}]}_{\ell \ell'}^{mm'} = -  \frac{4 \pi \ell'}{2 \ell'+1} \, \sum_{n} w_n\, Y_\ell^m(s_n) \,\nabla U_j^n  \,  \big( t_n^{jk}\big)^{-(\ell'+1)} \, Y_{\ell'}^{m'} (s_n^{jk})
\end{equation}
%Such relation holds, in particular, for $i \in N_j$ and $i \not= k$.
Thus, since $U_j$ depends only upon $x_i$ such that $i \in N_j$, we conclude that $\nabla A_{jk}$ vanishes whenever $i \not= j$ and $i \not=k$ and $i \not\in N_j$. In order to discuss the opposite case, i.e., $i = j$ or $i =k$ or $i \in N_j$, notice that the events $(i = j)$ and $(i = k)$ are mutually exclusive, as are $(i = j)$ and $(i \in N_j)$. We obtain the subcases $i = j$, and $i = k$, and $i \in N_j \, , \, i \not= k$, which we address individually.

Standard differentiation implies that
\begin{multline}\label{eq:9}
\nabla \Big[ U_j^n  \,  \big( t_n^{jk}\big)^{-(\ell'+1)} \, Y_{\ell'}^{m'} (s_n^{jk}) \Big] = \nabla U_j^n  \,  \big( t_n^{jk}\big)^{-(\ell'+1)} \, Y_{\ell'}^{m'} (s_n^{jk}) - \\
+ U_j^n  \, (\ell' + 1)  \big( t_n^{jk}\big)^{-(\ell'+2)} \, \nabla t_n^{jk} \, Y_{\ell'}^{m'} (s_n^{jk}) + U_j^n  \,  \big( t_n^{jk}\big)^{-(\ell'+1)} \, \big( D\, s_n^{ji} \big)^T\, \nabla Y_{\ell'}^{m'} (s_n^{jk})
\end{multline}
where $D$ emphasizes that the gradient of the vector quantity $s_j^{jk}$ is indeed its Jacobian matrix. We proceed to evaluate $\nabla t_n^{jk}$ and $ D \, s_n^{jk}$. When $i = j$, differentiation implies
\[
\nabla t_n^{jk} = \frac{s_n^{jk}}{r_k} \qquad , \qquad D \, s_n^{jk} = \frac{I - s_n^{jk} \otimes s_n^{jk} }{|x_j + r_j s_n - x_k|^3}
\]
where $I$ is the identity matrix and $\otimes$ indicates the outer product. We remark that the Jacobian matrix $D \, s_n^{jk}$ is symmetric, so that the transpose in \eqref{eq:9} is redundant. Analogously, the case $i = k$ yields $\nabla_{\! j} \, t_n^{jk} = - \nabla_{\! k} \, t_n^{jk}$ and $D_j \, s_n^{jk} = - D_k \, s_n^{jk}$, where the extra subscripts refer to the variables with respect to which differentiation is taken. Those relationships imply
\begin{multline*}
\nabla_{\! j} \Big[ U_j^n  \,  \big( t_n^{jk}\big)^{-(\ell'+1)} \, Y_{\ell'}^{m'} (s_n^{jk}) \Big] + \nabla_{\! k} \Big[ U_j^n  \,  \big( t_n^{jk}\big)^{-(\ell'+1)} \, Y_{\ell'}^{m'} (s_n^{jk}) \Big] = \\
  \Big[ \nabla_{\! j} \, U_j^n + \nabla_{\! k} \, U_j^n  \Big]  \big( t_n^{jk}\big)^{-(\ell'+1)} \, Y_{\ell'}^{m'} (s_n^{jk})
\end{multline*}
which provide a convenient way of evaluating $[\nabla_{\! k} \, A_{jk}]_{\ell \ell'}^{m m'}$ from $[\nabla_{\! j} \, A_{jk}]_{\ell \ell'}^{m m'}$. In fact, we obtain the quasi-skew-symmetric relation
\[
[ \nabla_{\! j} \, A_{jk}]_{\ell \ell'}^{m m'} + [\nabla_{\! k} \, A_{jk}]_{\ell \ell'}^{m m'} = -  \frac{4 \pi \ell'}{2 \ell'+1} \, \sum_{n} w_n\, Y_\ell^m(s_n) \Big[ \nabla_{\! j} \, U_j^n + \nabla_{\! k} \, U_j^n  \Big]  \big( t_n^{jk}\big)^{-(\ell'+1)} \, Y_{\ell'}^{m'} (s_n^{jk})
\]
for $\nabla A_{jk}$. Finally, the case $i \in N_j \, , \, i \not= k$ reduces to \eqref{eq:8}.