\section{A Brief Review of the ddPCM Strategy}\label{sec:review}

\subsection{The Polarizable Continuum Model}
In PCSM's, the solute is accommodated in a hollow molecular cavity $\Omega$ surrounded by the continuum, which can be treated either as a conductor, as in COSMO, or as a dielectric, as in PCM. We follow the customary approach of taking the cavity to be the so-called Van der Waals cavity\cite{ReviewPCM_2005}, i.e., the union of spheres centered at each atom with radii coinciding with (scaled) van der Waals radii. Note that the similar Solvent Accessible Surface (SAS) cavity can be treated through this approach as well. Models based on the Solvent Excluded Surface (SES) have recently been proposed~\cite{Harbrecht2011,quan2017polarizable,C5CP03410H,JCC:JCC21431}, however are not considered here.

The electrostatic part of the solute/solvent interaction is given by the electrostatic interaction energy between the solute's density of charge $\rho$ and the polarization potential of the solvent $W$, namely
\[
E_s = \tfrac{1}{2}\, f(\varepsilon_s)\,\int_\Omega \rho(x) W(x) \, dx,
\]
Here $f(\varepsilon_s)$ is an empirical scaling that depends on the dielectric constant $\varepsilon_s$ of the solvent, e.g., $f(\varepsilon_s) = (\varepsilon_s - 1)/\varepsilon_s$ for COSMO and $f(\varepsilon_s) = 1$ for PCM, $\rho$ is the solute's charge density, and $W$ is the polarization potential of the solvent. The quantities $W$ and $E_s$ are usually referred to as the reaction potential and the electrostatic contribution to the solvation energy, respectively. 

The reaction potential is defined as $W = \varphi - \Phi$, where $\varphi$ is the total electrostatic potential of the solute/solvent system and $\Phi$ is the potential of the solute \emph{in vacuo}. In the case of the PCM, the total potential $\varphi$ satisfies the generalized Poisson equation\cite{Mennucci_JCP_IEF1,Mennucci_JMC_IEF2}
\begin{equation}
 \label{eq:genpoisson}
\text{div} \, (\varepsilon \, \nabla \varphi) = -4\pi \rho,
\end{equation}
where the coefficient $\varepsilon$ is defined as $\varepsilon(x) = 1$ when $x \in \Omega$, and $\varepsilon(x) = \varepsilon_s$ otherwise. As a consequence, the reaction potential fulfills
\begin{equation} 
\label{eq:pcmpde}
\left \{ 
\begin{alignedat}{4}
\Delta \,  W &= 0  &&\mbox{in } \sR^3\setminus \Gamma  \\
 \ [\![W]\!] &= 0  &&\mbox{on } \Gamma\\
\  [\![\varepsilon \, \partial_n W]\!] &= (\varepsilon_s-1)\, \partial_n \Phi &\qquad& \mbox{on } \Gamma.
\end{alignedat} 
\right.
\end{equation}
Here $\Gamma=\partial\Omega$ is the boundary of the cavity, $\partial_n$ is the normal derivative at $\Gamma$, and $[\![\,\cdot\,]\!]$ is the jump operator (inside minus outside) at $\Gamma$.

The reaction potential $W$ is a \emph{single-layer potential}, and can be represented\cite{sauter2010boundary} as $W(x) = (\tilde{\mathcal{S}}\sigma)(x)$ when $x \in \sR^3 \setminus \Gamma$, or $W(s) = (\mathcal{S}\sigma)(s)$ when $s \in \Gamma$, where we introduced a surface density of charge $\sigma$ usually referred to as \emph{apparent surface charge}. The integral operators
\begin{equation*}
 %\label{eq:Stilde}
 (\tilde{\cS}\sigma)(x) = \int_{\Gamma} \frac{\sigma(t)}{|x-t|} \, dt \qquad  x \in \sR^3 \setminus \Gamma \qquad , \qquad  ({\cS}\sigma)(s) = \int_{\Gamma} \frac{\sigma(t)}{|s-t|} \, dt \qquad  s \in \Gamma
\end{equation*}
are known, respectively, as the single layer potential and the single layer operator. Furthermore, the latter one is invertible\cite{Calderon}. It can be shown that $\sigma$ satisfies the equation $\sigma = 1/4\pi \, [\![ \partial_n W]\!]$, so that it is possible to recast the PCM problem \eqref{eq:pcmpde} as a single integral equation for $\sigma$. Let $\cD$ be the double layer operator
\begin{equation}\label{eq:DOper}
 ({\cD}\sigma)(s) = \int_{\Gamma} \sigma(t) \, \partial_{n(t)} \bigg( \frac{1}{|s-t|}\bigg) \,dt \qquad s \in \Gamma
\end{equation}
and define the operators 
\begin{equation}
 \label{eq:Reps}
 \cR_\varepsilon = 2\pi \, g(\varepsilon_s) \, \cI - \cD \qquad, \qquad \cR_\infty = 2\pi \, \cI - \cD
\end{equation}
where $g(\varepsilon_s) = (\varepsilon_s+1)/(\varepsilon_s-1)$ and $\cI$ is the identity. It can be shown\cite{ReviewPCM_2005} that the apparent surface charge satisfies
\begin{equation}
\label{eq:IEFPCM}
\cR_\varepsilon \, \cS \, \sigma = - \cR_\infty \, \Phi \qquad \text{on }\Gamma
\end{equation}
This is known as the IEF-PCM equation, and it involves the operators $\cR_\infty$ and $\cR_\varepsilon$, which are both invertible. When the dielectric constant $\varepsilon_s$ approaches infinity, the IEF-PCM equation simplifies to $\cS \, \sigma = - \Phi$ on $\Gamma$, which is the Integral Equation Formulation of the Conductor-like Screening Model (COSMO)\cite{Lipparini_JCP_VPCM}.


\subsection{The ddPCM-method}


We recall how to solve the IEF-PCM boundary integral equation \eqref{eq:IEFPCM} within the domain-decomposition paradigm. As a preliminary step, we set $\Phi_\varepsilon = \cS \, \sigma$ and write \eqref{eq:IEFPCM} as a succession of two integral equations, the latter of which is equivalent to the COSMO equation\cite{Cances_Librone_PCM}, namely
\begin{alignat}{3}
\cR_\varepsilon \, \Phi_\varepsilon & = \cR_\infty \, \Phi \qquad && \text{on }\Gamma  \label{eq:ddPCM-1} \\
\cS \, \sigma & = -\Phi_\varepsilon  && \text{on }\Gamma \label{eq:ddPCM-2} 
\end{alignat}
Indeed, \eqref{eq:ddPCM-2} is a COSMO equation with the modified potential $\Phi_\varepsilon$ in place of the potential $\Phi$. This allows to develop the ddPCM strategy as an extension of the ddCOSMO approach. First, equation \eqref{eq:ddPCM-1} is solved in order to compute the right-hand-side $-\Phi_\varepsilon$ of equation \eqref{eq:ddPCM-2}; secondly, ddCOSMO is employed to solve equation \eqref{eq:ddPCM-2} with the modified potential $-\Phi_\varepsilon$, and compute the solvation energy $E_s$.


In order to discuss the domain-decomposition approach employed for both steps, let us introduce some notation. As anticipated, we take the cavity $\Omega$ be the union of $M$ spheres $\Omega_j = B(x_j, r_j)$ with boundaries $\Gamma_j$. We define $\Gamma_j^\text{ext}:= \Gamma_j \cap \Gamma$ and $\Gamma_j^\text{int} := \Gamma_j \cap \Omega$, as the portion of the surface $\Gamma_j$ which is exposed to the solvent or buried into the cavity, respectively, and let $U_j: \Gamma_j \to \mathbb{R}$ be the characteristic function of $\Gamma_j^\text{ext}$. Let $\Phi_j : \Gamma_j \to \mathbb{R}$ and $\Phi_{\varepsilon,j} : \Gamma_j \to \mathbb{R}$ be, respectively, the local trivial extensions of $\Phi$ and $\Phi_\varepsilon$ to $\Gamma_j$ defined as
\begin{equation}
 \label{eq:trivial_ext}
 \Phi_j(s) = 
 \begin{cases}
  \Phi(s) \phantom{0} \quad s \in \Gamma_j^\text{ext}\\
  0 \phantom{\Phi(s)} \quad s \in \Gamma_j^\text{int}
 \end{cases}
\qquad
\Phi_{\varepsilon,j}(s) = 
 \begin{cases}
  \Phi_{\varepsilon}(s) \phantom{0} \quad s \in \Gamma_j^\text{ext}\\
  0 \phantom{\Phi(s)_{\varepsilon}} \quad s \in \Gamma_j^\text{int}
 \end{cases}
\end{equation}

Then, through simple algebra, we obtain
\begin{equation}\label{eq:16}
(\mathcal{D} \, \Phi ) (s) = ( \mathcal{D}_j \, \Phi_j )(s) + \sum_{k \ne j} \,(\tilde{\mathcal{D}}_k \, \Phi_k )(s) \qquad ; \qquad s \in \Gamma_j^\text{ext} \quad, \quad  j = 1 , \ldots , M,
\end{equation}
where $\cD_j$ and $\tilde{\cD}_j$ are, respectively, the local double layer operator and the local double layer potential relative to the local sphere $\Gamma_j$, i.e., they are of the same form as~\eqref{eq:DOper} with the integration over $\Gamma_j$ instead of $\Gamma$ and $s\in\Gamma_j$ or $s\not\in\Gamma_j$ for $\cD_j$ resp. $\tilde{\cD}_j$. 
An analogous result holds for $\Phi_\varepsilon$ and its local extensions $\Phi_{\varepsilon,j}$.  %We refer to \cite{Stamm_JCP_DDPCM} for further details. %Since the right-hand-side is indeed well-defined for every $s\in \Gamma_j$, we can define the extensions $\widetilde{\mathcal{D} \, \Phi} : \Gamma_j \to \mathbb{R}$ and $\widetilde{\mathcal{D} \, \Phi}_\varepsilon : \Gamma_j \to \mathbb{R}$.
We proceed as follows.

{\bf Step 1.} 
We localize the integral equation \eqref{eq:ddPCM-1} to $\Gamma_j$ through the characteristic function $U_j$ as
\begin{equation}\label{eq:20}
2 \pi \, g(\varepsilon_s) \, U_j \, \Phi_{\varepsilon} - U_j \, {\mathcal{D} \, \Phi}_\varepsilon = 2 \pi \, U_j \, \Phi - U_j \, {\mathcal{D} \, \Phi} \qquad \text{on }\Gamma_j
\end{equation}
for each $j=1,\ldots,M$. 
%We define $\Phi_j : \Gamma_j \to \mathbb{R}$ as $\Phi_j = U_j\,\Phi$ and, w
Without loss of generality, we trade $U_j \, \Phi_\varepsilon$ for $U_j \, \Phi_{\varepsilon,j}$ in the left-hand-side of~\eqref{eq:20}.
%, where $\Phi_{\varepsilon,j} : \Gamma_j \to \mathbb{R}$ is an extension of the unknown $\Phi_\varepsilon$. 
We ensure that $\Phi_{\varepsilon,j}$ indeed vanishes on $\Gamma_j^\text{int}$ by imposing the additional constraint
\begin{equation}\label{eq:19}
(1 - U_j) \, \Phi_{\varepsilon,j}  = 0\qquad \text{on }\Gamma_j 
\end{equation}
In order to obtain a single equation, we multiply \eqref{eq:19} by the factor $g(\varepsilon_s)$, and add it sidewise to \eqref{eq:20}, so that
\begin{equation}\label{eq:21}
2 \pi \, g(\varepsilon_s) \, \Phi_{\varepsilon,j} - U_j \, {\mathcal{D} \, \Phi}_\varepsilon = 2 \pi \, \Phi_j - U_j \, {\mathcal{D} \, \Phi} \qquad \text{on }\Gamma_j
\end{equation}
When $s$ belongs to $\Gamma_j^\text{int}$, we recover $\Phi_{\varepsilon,j}(s) = 0$, so that we have effectively built the constraint \eqref{eq:19} into the previous equation. Thus, recalling that $\Phi_j$ vanishes on $\Gamma_j^\text{int}$ by definition, we proceed to apply the decomposition \eqref{eq:16} to both sides of \eqref{eq:21}, and obtain
%As a next step, we employ the decomposition \eqref{eq:16}, under the hypotheses that $\Phi_{\varepsilon,j}$ and $\Phi_j$ vanish on $\Gamma_j^\text{int}$, which we impose as additional explicit constraints. We obtain the integral equation 
%\begin{multline}\label{eq:17}
%2\pi \, \frac{\varepsilon + 1}{\varepsilon - 1} \, U_j \, \Phi_{\varepsilon,j} - U_j \bigg( \cD_j \, \Phi_{\varepsilon,j} + \sum_{k \ne j} \tilde{\cD}_{k} \, \Phi_{\varepsilon,k}  \bigg) = \\ 2 \pi \, U_j \, \Phi_j - U_j \bigg( \cD_j \, \Phi_j + \sum_{k \ne j} \tilde{\cD}_{k} \, \Phi_{k}  \bigg) \qquad \text{on }\Gamma_j
%\end{multline}
%alogn with the constraints
%\begin{alignat}{2}
%\Phi_j & = U_j \, \widetilde{\Phi} \qquad & \text{on }\Gamma_j \label{eq:18}\\
%(1 - U_j) \, \Phi_{\varepsilon,j} & = 0 &\text{on }\Gamma_j \label{eq:19}
%\end{alignat}
%In order to obtain a single equation, we insert \eqref{eq:18} into the right-hand-side of \eqref{eq:17}, multiply \eqref{eq:19} by the factor $2\pi \, (\varepsilon+1)/(\varepsilon-1)$ and add it sidewise to \eqref{eq:17}.
 %This yields
\begin{multline}\label{eq:1}
2\pi \, g(\varepsilon_s) \, \Phi_{\varepsilon,j} - U_j \bigg( {\mathcal{D}}_j \, \Phi_{\varepsilon,j} + \sum_{k \ne j} \, \tilde{\mathcal{D}}_{k} \, \Phi_{\varepsilon,k}  \bigg) = \\ 2 \pi \, {\Phi_j} - U_j \bigg( {\mathcal{D}}_j \,\Phi_{j} + \sum_{k \ne j} \, \tilde{\mathcal{D}}_{k} \, \Phi_{k}  \bigg) \qquad \text{on }\Gamma_j
\end{multline}
which constitutes our domain-decomposition strategy for equation \eqref{eq:ddPCM-1}. The summation implies that every subdomain $\Omega_j$ interacts with all other subdomains. We anticipate that this contrasts with the ddCOSMO strategy for equation \eqref{eq:ddPCM-2}, which only involves neighbor-to-neighbor interactions.

{\bf Step 2.} 
We use ddCOSMO to solve equation \eqref{eq:ddPCM-2}. A complete derivation of the ddCOSMO method and details on its implementation can be found elsewhere \cite{Cances_JCP_ddCOSMO,Lipparini_JCTC_ddCOSMO}. 

%Here, we recall that ddCOSMO recasts the COSMO partial differential equation for the reaction potential as a set of coupled local equations, one for each sphere, which are in turn written as local integral equations

%ddCOSMO reaction potential $W = \tilde{\mathcal{S}}\,\sigma$, where $\sigma$ solves the integral equation \eqref{eq:ddPCM-2}, solves the equation $\Delta W = 0$ in $\Omega$ with boundary condition $W=-\Phi_\varepsilon$ on $\Gamma=\partial \Omega$. 
%The restriction $W_j := W |_{\overline{\Omega}_j}$ is harmonic over the subdomain $\Omega_j$, thus it can be represented  locally as 
% \begin{equation}\label{eq:COSMOloc}
% W_j(x) = (\tilde{\mathcal{S}}_j \,  \sigma_j) (x) \quad , \quad x \in \Omega_j \qquad ; \qquad
% W_j(s) = (\mathcal{S}_j \,  \sigma_j) (s) \quad , \quad s \in \Gamma_j
% \end{equation}
% where $\sigma_j$ is an unknown surface density on $\Gamma_j$, and $\cS_j$ and $\tilde{\cS}_j$ are, respectively, the single layer potential and the single layer operator on $\Gamma_j$. The local problems \eqref{eq:COSMOloc} are coupled together by imposing on $\Gamma_j$ the coupling condition
%decomposing $W_j$ as
% \begin{equation}\label{eq:5}
% W_j(s) = - \Phi_{\varepsilon,j}(s) +  n_j(s) \, \sum_{k \in N_j} \,{W}_k(s) \qquad ; \qquad s \in \Gamma_j \quad , \quad j = 1, \ldots , M
% \end{equation}
% where $N_j$ is the set of all neighboring subdomains of $\Omega_j$, $W_k$ is understood as its trivial extension to $\Omega$, and $n_j$ is a normalization factor defined as follows. If $s$ does not belong to any neighbor of $\Omega_j$, then $n_j(s)$ vanishes. Otherwise, $n_j(s)$ is the reciprocal of the number of neighbors. The decomposition \eqref{eq:5} also employs the fact that $\Phi_{\varepsilon,j}$ vanishes on $\Gamma_j^\text{int}$. When we substitute the local problems \eqref{eq:COSMOloc} into the decomposition \eqref{eq:5}, and define $\tilde{\cS}_{jk} \, \sigma_k = n_j \, \tilde{\cS}_k \, \sigma_k$, we obtain

%\begin{equation}
%\label{eq:2}
%\mathcal{S}_j \, \sigma_j  -  \sum_{k \in N_j} \, \tilde{\mathcal{S}}_{jk} \, \sigma_k = -  \Phi_{\varepsilon,j} \qquad \text{on } \Gamma_j,
%\end{equation}
%where $N_j$ is the set of all \emph{neighbors}, i.e., the spheres that intersect $\Omega_j$, and $n_j$ is a normalization factor defined as follows. If $s$ does not belong to any neighbor of $\Omega_j$, then $n_j(s)$ vanishes. Otherwise, $n_j(s)$ is the reciprocal of the number of neighbors. %The decomposition \eqref{eq:5} also employs the fact that $\Phi_{\varepsilon,j}$ vanishes on $\Gamma_j^\text{int}$. 
%As opposed to the local problem \eqref{eq:1} which features a global interaction of all subdomains, the ddCOSMO step \eqref{eq:2} is characterized by the interaction of subdomain $\Omega_j$ with only its neighbors. This results in a sparse, rather than dense, discrete operator. We anticipate that this is the key feature that allows to compute the ddCOSMO forces within linear complexity, with respect to the number of atoms $M$. More details are provided in Section \ref{sec:forces}.

We note that $W = \tilde{\mathcal{S}}\,\sigma$, where $\sigma$ it the solution to the integral equation \eqref{eq:ddPCM-2}, solves the equation $\Delta W = 0$ in $\Omega$ with boundary condition $W=-\Phi_\varepsilon$ on $\Gamma=\partial \Omega$. 
The restriction $W_j := W |_{\overline{\Omega}_j}$ is harmonic over the subdomain $\Omega_j$, thus it can be represented locally as 
\begin{equation}\label{eq:COSMOloc}
W_j(x) = (\tilde{\mathcal{S}}_j \,  \sigma_j) (x) \quad , \quad x \in \Omega_j \qquad ; \qquad
W_j(s) = (\mathcal{S}_j \,  \sigma_j) (s) \quad , \quad s \in \Gamma_j
\end{equation}
where $\sigma_j$ is an unknown surface density on $\Gamma_j$, and $\cS_j$ and $\tilde{\cS}_j$ are, respectively, the single layer potential and the single layer operator on $\Gamma_j$. The local problems \eqref{eq:COSMOloc} are coupled together by imposing on $\Gamma_j$ the coupling condition
%decomposing $W_j$ as
\begin{equation}\label{eq:5}
W_j(s) = - \Phi_{\varepsilon,j}(s) +  n_j(s) \, \sum_{k \in N_j} \,{W}_k(s) \qquad ; \qquad s \in \Gamma_j \quad , \quad j = 1, \ldots , M
\end{equation}
where $N_j$ is the set of all neighboring subdomains of $\Omega_j$, $W_k$ is understood as its trivial (zero) extension to $\Omega$, and $n_j$ is a normalization factor defined as follows. If $s$ does not belong to any neighbor of $\Omega_j$, then $n_j(s)$ vanishes. Otherwise, $n_j(s)$ is the reciprocal of the number of neighbors. The decomposition \eqref{eq:5} also employs the fact that $\Phi_{\varepsilon,j}$ vanishes on $\Gamma_j^\text{int}$. When we substitute the local problems \eqref{eq:COSMOloc} into the decomposition \eqref{eq:5}, and define $\tilde{\cS}_{jk} \, \sigma_k = n_j \, \tilde{\cS}_k \, \sigma_k$, we obtain
\begin{equation}\label{eq:2}
\mathcal{S}_j \, \sigma_j  -  \sum_{k \in N_j} \, \tilde{\mathcal{S}}_{jk} \, \sigma_k = -  \Phi_{\varepsilon,j} \qquad \text{on } \Gamma_j
\end{equation}
As opposed to the local problem \eqref{eq:1} which features a global interaction of all subdomains, the ddCOSMO step \eqref{eq:2} is characterized by the interaction of subdomain $\Omega_j$ with only its neighbors. This results in a sparse, rather than dense, discrete operator. We anticipate that this is the key feature that allows to compute the ddCOSMO forces within linear complexity, with respect to the number of atoms $M$. More details are provided in Section \ref{sec:forces}.
\black 

\subsection{Numerical Discretization}

The first step to discretize equations \eqref{eq:1} and \eqref{eq:2} is to expand $\Phi_j$, $\Phi_{\varepsilon,j}$ and $\sigma_j$ as truncated series of spherical harmonics. Let $Y_\ell^m$ be the spherical harmonic of degree $\ell$ and order $m$ on the unit sphere $\mathbb{S}$, and let $y$ be the variable on $\mathbb{S}$. For a prescribed integer parameter ${L_\text{max}}$,  we approximate the surface charge $\sigma_j$ as the truncated expansion
\[
\sigma_j(s) = \sigma_j(x_j + r_j y) = \frac{1}{r_j} \, \sum_{\ell=0}^{L_\text{max}} \sum_{m = -\ell}^\ell \, [X_j]_\ell^m \, Y_\ell^m(y)
\]
for some unknown coefficients $X = [X_j]_\ell^m$. %The choice of this ansatz implies a spectral-like numerical method. 
The scaling factor has been introduced for convenience, as seen from the derivation in Appendix \ref{app:cosmo}. We approximate $\Phi_{\varepsilon,j}$ and $\Phi_j$ in the same fashion, namely
\begin{equation}\label{eq:71}
\Phi_{\varepsilon,j}(s) = - \sum_{\ell=0}^{L_\text{max}} \sum_{m = -\ell}^\ell \, [G_j]_\ell^m \, Y_\ell^m(y) \qquad , \qquad \Phi_j(s) = -\sum_{\ell=0}^{L_\text{max}} \sum_{m = -\ell}^\ell \, [F_j]_\ell^m \, Y_\ell^m(y)
\end{equation}
where $G = [G_j]_\ell^m$ and $F = [F_j]_\ell^m$ are the coefficients of the expansions. The minus signs have been introduced so that there is no negative sign in the discretization of the right-hand-side of the ddCOSMO step \eqref{eq:2}. 


We interpret the local problems (\ref{eq:1}) and (\ref{eq:2}) in a variational setting that uses spherical harmonics as test functions. Numerical discretizations are obtained by employing the orthogonality of the spherical harmonics, along with Lebedev grids to perform quadrature, see Appendix \ref{app:pcm} and \ref{app:cosmo}. In the spirit of domain-decomposition, we combine together the discretizations of the local problems to obtain those of the global problems \eqref{eq:ddPCM-1} and \eqref{eq:ddPCM-2}. Respectively, we obtain the linear systems
\begin{equation}\label{eq:6}
A_\varepsilon \, G = A_\infty \, F \qquad , \qquad  L \, X = G
\end{equation}
where the discrete operators $A_\varepsilon$ and $L$ are known in closed form, see \eqref{eq:ajj}, \eqref{eq:ajk}, and \eqref{eq:61}, \eqref{eq:62}. We remark that $A_\varepsilon$ is a dense matrix, while $L$ is a sparse matrix. In fact, for a fixed index $j$, the block $L_{jk}$ is \emph{a priori} nonzero only when $k = j$ or $k \in N_j$. Since $\Omega_j$ is a neighbor of $\Omega_k$ if and only if $\Omega_k$ is a neighbor of $\Omega_j$, then $L_{jk}$ is \emph{a priori} nonzero if and only if $L_{kj}$ is. We conclude that $L$ has a symmetric block-structure, although, in general, $L$ is non symmetric.

The dependency of operators $A_\varepsilon$ and $L$ upon the nuclear positions is evident. The right-hand side vector $F$ depends on the nuclear positions as well, not only due to the obvious dependence on the nuclear positions of the solute's potential but also in a more subtle way. Indeed, the second one of the expansions \eqref{eq:71} and the fact that $\Phi_j$ is the trivial extension of $\Phi$, i.e., $\Phi_j = U_j \, \Phi$ imply
\begin{equation}\label{eq:25}
[F_j]_\ell^m = - \int_{\mathbb{S}} \Phi_j(s(y)) \, Y_\ell^m(y) \,dy = - \int_{\mathbb{S}} U_j(s(y)) \, \Phi(s(y)) \, Y_\ell^m(y) \,dy
\end{equation}
Since the characteristic function $U_j$ depends upon $x_j$ and $x_i$ such that $i \in N_j$, see Appendix \ref{app:pcm_der}, so does $[F_j]_\ell^m$. 
%{\color{red}(Paolo: should also comment on $\nabla \Phi = - \boldsymbol{E} $...)} 
Those dependencies upon the nuclear positions are needed in the computation of the ddPCM-forces, discussed in the following Section.


%As customary, in order to obtain the ddPCM and ddCOSMO discretizations we interpret each local problem (\ref{eq:1}) and (\ref{eq:2}) in a variational setting that employs expansions of $\Phi_j$, $\Phi_{\varepsilon,j}$ and $\sigma_j$ as truncated series of spherical harmonics, as well as spherical harmonics as test functions. We briefly describe the discretization of the ddCOSMO problem \eqref{eq:2}, and refer to Appendix \ref{app:mats} for the more involved discretization of the ddPCM problem \eqref{eq:1}.
%
%
%
%The PCM equation \eqref{eq:1} is discretized in a similar fashion, by expanding $\hat{\Phi}_j$ and $\hat{\Phi}_{\varepsilon,j}$ as truncated series of spherical harmonics
%\[
%\hat{\Phi}_{\varepsilon,j}(y) = - \sum_{\ell'=0}^{L_\text{max}} \sum_{m' = -\ell'}^{\ell'} [G_j]_{\ell'}^{m'} \, Y_{\ell'}^{m'}(y) \qquad , \qquad \hat{\Phi}_j(y) = -\sum_{\ell'=0}^{L_\text{max}} \sum_{m' = -\ell'}^{\ell'} [F_j]_{\ell'}^{m'} \, Y_{\ell'}^{m'}(y)
%\]
%where the minus signs have been introduced for convenience. We refer to Appendix \ref{app:mats} for the full details of the numerical discretization.
%The final discretizations of the global problems \eqref{eq:ddPCM-1} and \eqref{eq:ddPCM-2} are, respectively
%\begin{equation}\label{eq:6}
%A_\varepsilon \, G = A_\infty \, F \qquad , \qquad  L \, X = G
%\end{equation}
%where $G = [G_j]_{\ell'}^{m'}$,  $F = [F_j]_{\ell'}^{m'}$ and the expressions for the entries of the discrete operator $A_\varepsilon$ are given in \eqref{eq:ajj} and \eqref{eq:ajk}.
