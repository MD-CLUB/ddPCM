\documentclass[aip,jcp,a4paper,11pt]{revtex4-1}
%\documentclass[draft]{revtex4-1}
\usepackage[utf8x]{inputenc}
\usepackage{amsmath}
\usepackage{amsfonts}
\usepackage{amssymb}
\usepackage{bm}
\usepackage{color}
\usepackage{graphicx}
\usepackage{booktabs}
\usepackage{array}
%\usepackage{subfigure}
\usepackage{epsfig,subfig,graphicx}
\usepackage{booktabs}
\usepackage{multirow}

\newcommand{\two}{0.475\textwidth}
\newcommand{\three}{0.31\textwidth}

\newlength\tbspace
\setlength\tbspace{5mm}
\newcolumntype{R}{r<{\hspace{\tbspace}}}
\newcolumntype{A}{r<{\hspace{2.5mm}}}
\newcommand{\bn}{\mathbf{n}}
\newcommand{\br}{\mathbf{r}}
\newcommand{\bs}{\mathbf{s}}
\newcommand{\bv}{\mathbf{v}}
\newcommand{\bx}{\mathbf{x}}
\newcommand{\by}{\mathbf{y}}
\newcommand{\bR}{\mathbf{R}}
\newcommand{\bP}{\mathbf{P}}
\newcommand{\bh}{\mathbf{h}}
\newcommand{\bG}{\mathbf{G}}
\newcommand{\cA}{\mathcal{A}}
\newcommand{\cN}{\mathcal{N}}
\newcommand{\cI}{\mathcal{I}}
\newcommand{\cS}{\mathcal{S}}
\newcommand{\cD}{\mathcal{D}}
\newcommand{\cR}{\mathcal{R}}
\newcommand{\sR}{\mathbb{R}}
\newcommand{\bTh}{\bm{\Theta}}
\newcommand{\bXi}{\bm{\Xi}}

\newcommand{\App}{{\widehat{\Phi}_{j,\varepsilon}}}
\newcommand{\GApp}{{\widehat{\Phi}_{\varepsilon}}}
\newcommand{\FApp}{{\widehat{f}_{j,\varepsilon}}}
\newcommand{\GFApp}{{\widehat{f}_{\varepsilon}}}
\newcommand{\AppZ}{{\widehat{\Phi}^0_{j}}}
\newcommand{\GAppZ}{{\widehat{\Phi}^0}}
\newcommand{\red}{\color{red}}
\newcommand{\black}{\color{black}}
\newcommand{\nablai}{\nabla_{\!i}\,}
\newcommand{\Di}{D_i}
\newcommand{\nablaj}{\nabla_{\!j}\,}
\newcommand{\nablak}{\nabla_{\!k}\,}
\newcommand{\Dj}{D_j}

\newcommand{\Best}[1]{\color{red}{\tt #1}\color{black}}
%opening

\begin{document}
\title{Computation of Forces arising from the Polarizable Continuum Model within the Domain-Decomposition Paradigm}

\author{Paolo Gatto}
\affiliation{Mathematics Division, Center for Computational Engineering Science, RWTH Aachen University, Aachen, Germany}

\author{Filippo Lipparini}
\affiliation{Dipartimento di Chimica e Chimica Industriale, Universit\`a di Pisa, Via G. Moruzzi 13, 56124 Pisa, Italy}

\author{Benjamin Stamm}
\affiliation{Mathematics Division, Center for Computational Engineering Science, RWTH Aachen University, Aachen, Germany}



\begin{abstract}
The domain decomposition (dd) paradigm, originally introduced for the Conductor-like Screening Model (COSMO), has been recently extended to the dielectric Polarizable Continuum Model (PCM), resulting in the ddPCM method. We present here a complete derivation of the analytical derivatives of the ddPCM energy with respect to the positions of the solute's atoms, and discuss their efficient implementation. As it is the case for the energy, we observe a quadratic scaling, which is discussed and demonstrated with numerical tests.  
%Within implicit solvation models, the domain-decomposition strategy for the computation of the electrostatic energy due to the solvent based on the Polarizable %Continuum Model (PCM) has recently been developed. 
%The methodological development started with the so-called ddCOSMO method and has recently be generalized to the PCM equation resulting in the ddPCM-method [Stamm {\it et al.}, J. Chem. Phys. 144, 054101 (2016)] for which derive the forces within this article. 
%We show the derivation of the forces and derive an efficient implementation followed by numerical tests.
%In our previous work we developed a Schwarz's domain decomposition method for the Polarizable Continuum Model (PCM) equation. The discretization is systematically improvable and fully consistent with the Conductor-like Screening Model (COSMO), in the sense that it reduces to COSMO for large dielectric constants. In this work, we build on top of this framework and introduce the computation of analytical forces.

\end{abstract}

\maketitle
 
\section{Introduction}\label{sec:intro}
Solvation plays a crucial role in many processes in Chemistry and Biochemistry. As a consequence, including the effects of the solvent in the description of a chemical system is of paramount importance in computational Chemistry. In principle, this can be achieved by including sufficiently many solvent molecules in the computational representation of the system. Unfortunately, solute-solvent interactions are often dominated by long-range electrostatic contributions, which would require the inclusion of an extended number of solvent molecules to be properly described. When an accurate, quantum mechanical level of theory is employed, the presence of a large number of solvent molecules can easily make the computation very expensive, or even completely unfeasible. Furthermore, the system would not be correctly represented by a single configuration, thus requiring  to repeat the computation many times in order to take care of statistical sampling. 

Such a brute force strategy is not only undesirable, but also unnecessary when one's aim is to describe the solute and its properties in the presence of a solvent, rather than the solvent itself. In such case, a \textit{focused model} can be used, where the solute is described accurately, while the solvent is treated with an approximated level of theory which is, however, sufficient to introduce the desired solute-solvent interactions. 
A popular choice to describe solvation is to use QM/MM models\cite{Warshel_JMB_QMMM,Warshel_JACS_QMMM,Gao_Science_QMMM,Bakowies_JPC_QMMM,Truhlar_TCA_QMMMReview,Thiel_ACIE_QMMMReview,
Barone_Libro_QMMM,Morokuma_CR_ONIOM,Cappelli_IJQC_FQRev}, where the solute is treated at a quantum mechanical level of theory, while the solvent is described with a force field, and solute-solvent interactions are introduced via electrostatic interactions, possibly including mutual polarization\cite{Curutchet_JCTC_MMPol,Kongsted_JCTC_MMPExc,Christiansen_JCTC_MMPExcCCDFT,Steindal_PCCP_MMPExc,Caprasecca_JCTC_FMM,Lipparini_JCP_FQMag,Lipparini_JCTC_FQTD,Boulanger_JCTC_Drude,Boulanger_JCTC_QMMMPolPCM,Loco_JCTC_QMAMOEBA}. However, QM/MM methods still require to take care of the statistical sampling, which, when interested in certain properties, can be particularly challenging\cite{Lipparini_JCTC_ORMoxy,Giovannini_JCTC_FQVCD,Giovannini_JCTC_FQROA}.

Polarizable Continuum Solvation Models\cite{MST,ReviewPCM_1994,ReviewPCM_2005,Orozco_CR_Solvent00,Klamt:2011we,Mennucci:2012ct,honig1995cla,Roux:1999vp} (PCSM's) replace the solvent with a uniform, infinite, and polarizable continuum, which interacts with the solute's density of charge in a mutually polarizable fashion. PCSM's do not require statistical sampling, as such an effect is implicitly included in the use of a macroscopic property, namely the bulk dielectric permittivity $\varepsilon_s$, to model the solvent response. The interaction between the solute and the solvent is modeled via electrostatics. For a given value of the solute's density of charge, the electrostatics equations are solved in the presence of the polarizable continuum and a \emph{reaction potential} is then added to the solute's Hamiltonian. A new density of charge is then computed and the process is iterated until mutual polarization has been achieved\cite{ReviewPCM_2005,Lipparini_JCP_VPCM,Lipparini_JCTC_VPCMSCF,Lipparini_JCP_Perspective}.
Over the last decades, PCSM's have been extended to introduce solvent effects on molecular structures and properties\cite{Tomasi_PCCP_PCMProps,Mennucci_Chir_PCMProps,Mennucci_JPCL_PCM}, and are nowadays available in most quantum Chemistry codes, to the extent that have become a standard tool among computational chemists. 

More recently, PCSM's and QM/MM methods have been combined\cite{Pedone_CPC_QMMMPCM,Rega_JACS_QMMMPCM,Vreven_JCP_OniomPCM,Bandyopadhyay2002,Lipparini_JCTC_FQPCM,Caprasecca_JCTC_FMM,Steindal_JCPA_QMMMPCM,Boulanger_JCTC_QMMMPolPCM,Caprasecca_JCTC_QMMMPolPCM} in order to retain the atomistic description provided by QM/MM models, which is needed in order to describe specific solute-solvent interactions, and the inexpensive handling of long-range electrostatic interactions of PCSM's.

Standard implementations of PCSM's usually employ the Boundary Element Method\cite{MST,ReviewPCM_1994,ReviewPCM_2005,Scalmani_JCP_CSC,York_JPCA_CSC,
Herbert_JCP_ISWIG} (BEM) to numerically solve a discretized integral equation. This requires the solution of a linear system whose size scales linearly, although with a large proportionality constant, with respect to the number of atoms. This task is usually carried out through standard dense linear algebra techniques, such as the LU decomposition\cite{Cammi_JCC_Inversion}, which require a computational effort of cubic complexity, with respect to the size of the system. Consequently, the solution step can rapidly become demanding when dealing with systems as large as those treated via QM/MM methods, and one should seek other strategies. An alternative is provided by iterative techniques, which reduce the computational cost to that of several matrix-vector multiplications\cite{Scalmani_TCA_Iterative}, which is, in  the general case, quadratic. When fast summation techniques, e.g., the Fast Multipole Method (FMM)\cite{FMM}, can be employed, the computational complexity can be further reduced. Thus, it is possible to solve the PCSM linear equations in a number of floating point operations which is linear with respect to the number of atoms. Nevertheless, the solution of the PCSM equations can still represent a formidable bottleneck for large systems\cite{Lipparini_JPCL_ddCOSMO}, especially when repeated computations are required for statistical sampling purposes or time-dependent simulations.

In recent years, we have proposed an alternative approach for PCSM's which is based on  domain-decomposition. We introduced a novel strategy\cite{Cances_JCP_ddCOSMO}, referred to as ddCOSMO\cite{Cances_JCP_ddCOSMO,Lipparini_JCTC_ddCOSMO,Lipparini_JPCL_ddCOSMO,Lipparini_JCP_ddCOSMO-QM}, to solve the PCSM equation for the Conductor-like Screening Model\cite{Klamt_JCS_Cosmo} (COSMO) in combination with Van der Waals molecular cavities. 
As a first step, the COSMO equation in its differential form is rewritten as a system of coupled linear differential equations on each sphere, where the coupling occurs only between overlapping spheres. Secondly, each differential equation is recast as an integral equation, so that it can be efficiently solved by using a (truncated) expansion of spherical harmonics\cite{Cances_JCP_ddCOSMO}. The discretization produces a block-sparse linear system\cite{Lipparini_JCTC_ddCOSMO}, where only the blocks corresponding to overlapping spheres are nonzero. This structure allows for a computational cost that scales linearly with respect to the number of atoms, and is overall very small as compared to competing techniques. Indeed, as shown in \cite{Lipparini_JPCL_ddCOSMO}, as many as two or three orders of magnitude are gained by the ddCOSMO approach.

More recently, the method has been generalized to the Integral Equation Formalism Polarizable Continuum Model (IEFPCM) equation\cite{Mennucci_JCP_IEF1,Mennucci_JMC_IEF2,Mennucci_JPCB_IEF3}, which assumes a finite permittivity of the solvent. This resulted in the ddPCM method\cite{Stamm_JCP_DDPCM}, which is based on the same domain-decomposition approach. 
While the method has been developed in \cite{Stamm_JCP_DDPCM} to compute the electrostatic contribution to the solvation energy, the aim of this article is to present the derivation of analytical forces of the ddPCM solvation energy.

This paper is organized as follows. Section \ref{sec:review} reviews the ddPCM and ddCOSMO methods that we have previously developed. In Section \ref{sec:forces} we describe the derivation of the ddPCM forces and discuss their efficient implementation. Section \ref{sec:experiments} is devoted to numerical experiments. Finally, in Section \ref{sec:conclusions} we draw conclusions from the presented work and point to possible future directions of research.
%\newpage
%\section{A Brief Review of the ddPCM Strategy}\label{sec:review}

%\subsection{The Polarizable Continuum Model}
The foundation of Polarizable Continuum Solvation Models (PCSM's) is the assumption that the solvent in a solute-solvent system can be treated as either a dielectric, or a conducting continuum medium on the outside of the molecular cavity $\Omega$ of the solute. We follow the customary approach of taking the cavity to be the so-called Van der Waals cavity\cite{ReviewPCM_2005}, i.e., the union of spheres centered at each atom with radii coinciding with the van der Waals radii.
Within this approach, the topologically similar Solvent Accessible Surface (SAS) cavity can be treated as well. 
Models based on the Solvent Excluded Surface (SES) have also been proposed\cite{Quan_DDPCM_SES,C5CP03410H,Harbrecht2011,JCC:JCC21431}, but will not be considered here.

The electrostatic part of the solute-solvent interaction is given by $E_s = \tfrac{1}{2}\, f(\varepsilon)\,\int_\Omega \rho(x) W(x) \, dx$, where $f(\varepsilon)$ is an empirical scaling that depends on the dielectric constant of the solvent (and which is only applied in the case of the COSMO), $\rho$ is the charge density of the solute, and $W$ is the polarization potential of the solvent. The quantities $W$ and $E_s$ are usually referred to, respectively, as the reaction potential and the electrostatic contribution to the solvation energy. 

The reaction potential is defined as $W = \varphi - \Phi$, where $\varphi$ is the total electrostatic potential of the solute-solvent system and $\Phi$ is the potential of the solute \emph{in vacuo}. In the case of the PCM, the total potential $\varphi$ satisfies a (generalized) Poisson equation with suitable interface conditions\cite{Mennucci_JCP_IEF1,Mennucci_JMC_IEF2}. Indeed, if $\varepsilon_s$ is the macroscopic, zero-frequency relative dielectric permittivity of the solvent, and define $\varepsilon(x) = 1$ when $x \in \Omega$ and $\varepsilon(x) = \varepsilon_s$ otherwise, the reaction potential fulfills 
\begin{equation} 
\label{eq:pcmpde}
\left \{ 
\begin{alignedat}{4}
\Delta  W &= 0  &&\mbox{in } \sR^3\setminus \Gamma  \\
 \ [W] &= 0  &&\mbox{on } \Gamma\\
\  [\varepsilon \, \partial_\nu W] &= (\varepsilon_s-1) \partial_\nu \Phi &\qquad& \mbox{on } \Gamma
\end{alignedat} 
\right.
\end{equation}
Here $\Gamma=\partial\Omega$ is the boundary of the cavity, $\partial_\nu$ is the normal derivative on $\Gamma$, and $[\,\cdot\,]$ is the jump operator (inside minus outside) on $\Gamma$.

Recalling potential theory, $W$ can be represented as $W(x) = (\tilde{\mathcal{S}}\sigma)(x)$ when $x \in \sR^3 \setminus \Gamma$, or $W(s) = (\mathcal{S}\sigma)(s)$ when $s \in \Gamma$. 
The surface density $\sigma$ defined on $\Gamma$ is the so-called apparent surface charge,  $\tilde{\cS}$ is the single layer potential and $\cS$ is the single layer operator,  which is invertible \cite{Calderon}. 
Note that both $\tilde{\cS}$ and $\cS$ are based on the surface $\Gamma$.
It can be shown that $\sigma$ satisfies the equation $\sigma = 1/4\pi \, [ \partial_\nu W]$, so that it is possible to recast the PCM problem \eqref{eq:pcmpde} as a single integral equation for $\sigma$. In fact, if we define the operators 
\begin{equation}
 \label{eq:Reps}
 \cR_\varepsilon = 2\pi \frac{\varepsilon+1}{\varepsilon-1} \, \cI - \cD \qquad, \qquad \cR_\infty = 2\pi \, \cI - \cD
\end{equation}
where $\cI$ is the identity and $\cD$ is the double layer boundary operator (also based on $\Gamma$).
It can be shown\cite{ReviewPCM_2005} that the apparent surface charge satisfies
\begin{equation}
\label{eq:IEFPCM}
\cR_\varepsilon \, \cS \, \sigma = - \cR_\infty \, \Phi \qquad \text{on }\Gamma
\end{equation}
which is known as the IEF-PCM equation. It involves operators $\cR_\infty$ and $\cR_\varepsilon$, which are both invertible. Furthermore, when the dielectric constant $\varepsilon_s$ approaches infinity, the IEF-PCM equation simplifies to $\cS \, \sigma = - \Phi$ on $\Gamma$, which is the Integral Equation Formulation of the Conductor-like Screening Model (COSMO)\cite{Lipparini_JCP_VPCM}.

%\subsection{A domain decomposition approach}
%\label{ssec:AddApproach}
%The domain decomposition (dd) method is a methodology first introduced by Schwarz\cite{DD_Method} in the nineteenth century to solve partial differential equations defined on complex domains which can be decomposed as the union of simple, overlapping ones. Domain decomposition is an iterative strategy that consists in recasting the global (i.e., defined on the whole domain) problem in a collection of local problems (i.e., defined on each simple unit that constitutes the domain) with modified boundary conditions. 
%For each local domain, the boundary conditions depend on the global ones and on the local solutions defined on the neighboring domains: such conditions are updated at each iteration and convergence is reached when a stationary point is found. The main advantage of the dd-paradigm is that only overlapping domains exchange information, which, in turn, means that the system of local problems can be very sparse. 
%
%Recently, we have proposed a domain decomposition based strategy to solve the COSMO equations\cite{Cances_JCP_ddCOSMO,Lipparini_JCTC_ddCOSMO,Lipparini_JPCL_ddCOSMO,Lipparini_JCP_ddCOSMO-QM}; we called such a strategy ddCOSMO. ddCOSMO performs significantly better than any other discretization, exhibiting linear scaling in computational cost and a overall very low computational cost, which makes it two to three orders of magnitude faster than previous implementations. 
%Domain decomposition is a particularly viable approach for the COSMO problem, as its domain is the molecular cavity, which naturally decomposes into the atom-centered balls of which it is the union. Furthermore, the solution of the local problem in each ball is trivial and, as for large molecules only a small number of balls with respect to the total number will overlap, the resulting (linear) system of local equations is very sparse (as a rough estimate, each ball overlaps with at most 20-25 other balls). 
%We recapitulate here the ddCOSMO algorithm; all the details of its derivation, implementation, numerical performances and coupling with various methodologies to describe the solute can be found elsewhere.
%
%The COSMO equation \eqref{eq:COSMO} is recast as a system of linear PDEs defined in each ball:
%\begin{equation}
%\label{eq:ddcosmo1}
%\forall j=1,\ldots M \quad \left \{
%\begin{array}{rcll}
%\Delta W_j(\bx) &=& 0 &\qquad \bx \in \Omega_j, \\
%W_j(\bs) &=& h_j(\bs) &\qquad \bs \in \Gamma_j,
%\end{array}
%\right .
%\end{equation}
%where $\Gamma_j = \partial \Omega_j$ is the boundary of $\Omega_j$.
%Let $\Gamma_j^{\rm e}$ be the portion of the sphere $\Gamma_j $ which is exposed to the solvent, i.e., $\Gamma_j^{\rm e} = \Gamma \cap \Gamma_j$ and $\Gamma_j^{\rm i}$ the internal part. The boundary condition $h_j$ is defined as follows:
%\begin{equation}
%\label{eq:ddboundary}
% h_j(\bs) =
%\begin{cases}
% -\Phi(\bs), & \bs \in \Gamma_j^{\rm e}, \\
% \sum_{k\in\cN_j(\bs)} \frac{1}{|\cN_j(\bs)|}W_k(\bs) & \bs \in \Gamma_j^{\rm i},
%\end{cases} 
%\end{equation}
%where $\cN_j(\bs)$ is the list of balls that overlap $\Omega_j$ at $\bs$ and $ |\cN_j(\bs)|$ the number of such balls.
%We can now exploit the representation formula \eqref{eq:SL1}, as the local functions $W_j$ are local single-layer potentials 
%%As in the case of the global representation, the local harmonic function $W_j$ can be represented by a (now local) single layer potential
%\begin{equation}
%\label{eq:SL1-local}
%\forall \bx \in \Omega_j,\quad W_j(\bx) = (\tilde{\mathcal{S}_j}\sigma)(\bx) := \int_{\Gamma_j} \frac{\sigma_j(\by)}{|\bx - \by|} d\by,
%\end{equation}
%with $\sigma_j(\bs)$ being the local ASC associated with $W_j$. 
%We particularly emphasize that $W_j$ only coincides with $W$ in $\Omega_j$. 
%We can now rewrite the ddCOSMO boundary condition in a more compact fashion. Let $\chi_j$ the characteristic function of $\Omega_j$, i.e., 
%\begin{equation}
%	\label{eq:CharFct}
% \chi_j(\br) =
% \begin{cases}
%  1 & \br \in \Omega_j, \\
%  0 & \mbox{otherwise}.
% \end{cases}
%\end{equation}
%and let 
%\begin{equation}
%	\label{eq:LocCharFct}
%	\forall \bs \in \Gamma_j\quad \omega_{jk}(\bs) = \frac{\chi_k(\bs)}{|\cN_j(\bs)|}, \quad U_j(\bs) = 1 - \sum_{k\in \cN_j(\bs)} \omega_{jk}(\bs).
%\end{equation}
%Eq. \eqref{eq:ddboundary} can be rewritten as
%\begin{equation}
%\label{eq:ddbound2}
% h_j(\bs) = -U_j(\bs)\Phi(\bs) + \sum_{k\in\cN_j(\bs)} \omega_{jk}(\bs) (\tilde{\cS}_k\sigma_k)(\bs),
%\end{equation}
%where the subscript $k$ to the operator $\tilde{\cS}$ refers to the fact that the integral is to be performed on $\Gamma_k$ as introduced in  \eqref{eq:ddcosmo1}.
%Finally, by using the representation formula \eqref{eq:SL2} for the local problem, i.e. using 
%\begin{equation}
%	\label{eq:SL2-local}
%	\forall \bs \in  \Gamma_j,\quad W_j(\bs) = (\mathcal{S}_j\sigma_j)(\bs) := \int_{\Gamma_j} \frac{\sigma_j(\bs')}{|\bs - \bs'|} d\bs',
%\end{equation}
%we get the ddCOSMO system of coupled integral equations:
%\begin{equation}
% \label{eq:ddCOSMO}
% \forall \bs \in \Gamma_j \quad (\cS_j\sigma_j)(\bs) = -U_j(\bs)\Phi(\bs) + \sum_{k\in \cN_j(\bs)} \omega_{jk}(\bs)(\tilde{\cS}_{k}\sigma_k)(\bs).
%\end{equation}
%The last step to obtain the ddCOSMO algorithm is to discretize the local ASCs and the integral operators in a suitable basis of functions. A truncated set of real spherical harmonics is the natural choice: by expanding $\sigma_j$ as
%\[
%\sigma_j(\bR_j + r_j\by) = \sum_{l=0}^N\sum_{m=-l}^l [X_j]_l^m Y_l^m(\by),
%\]
%where $\by$ is a unit vector and we write a generic point $\bs \in \Gamma_j$ as $\bs = \bR_j + r_j\by$, we obtain
%\begin{equation}
%\label{eq:ddCOSMO-disc}
% \left (
%\begin{array}{ccc}
% L_{11} & \ldots & L_{1M} \\
% \vdots & \ddots & \vdots \\
% L_{M1} & \ldots & L_{MM} \\
%\end{array}
% \right )
% \left (
%\begin{array}{c}
%X_1\\
%\vdots\\
%X_M
%\end{array}
% \right )
%=
% \left (
%\begin{array}{c}
%g_1\\
%\vdots\\
%g_M
%\end{array}
% \right ),
%\end{equation}
%where 
%\begin{equation}
%	\label{eq:BraKet}
%	[ L_{jj}]_{ll'}^{mm'} =  \langle lm |\cS_j| l'm'\rangle,  \quad 	[L_{jk}]_{ll'}^{mm'} = - \langle lm|\omega_{jk}\tilde{\cS}_{k}|l'm'\rangle
%\end{equation}
%and
%\begin{equation}
%	\label{eq:ddCOSOMrhs}
%	[g_j]_l^m = -\langle lm| U_j \Phi\rangle.
%\end{equation}
%The expression of the matrix elements can be found in ref. \citenum{Cances_JCP_ddCOSMO}. Notice that $\omega_{jk}(\bs)$ vanishes if $\Omega_k$ is not overlapping with $\Omega_j$, which makes the $L_{jk}$-blocks vanish if they do not correspond to overlapping spheres: the ddCOSMO matrix is therefore sparse\cite{Cances_JCP_ddCOSMO}.
%
%\subsection{The PCM within the ddCOSMO paradigm}
%The aim of this article is to propose an efficient computational strategy for the Polarizable Continuum Model (PCM) based on a domain decomposition approach. Before discussing such a strategy, let us recapitulate what are, according to us, the requirements for a good discretization. Such a discretization must be
%\begin{enumerate}
%\item simple, i.e., controlled by a small number of parameters;
%\item systematically improvable, i.e., the numerical solution converges to the exact solution as the parameters are refined;
%\item consistent with ddCOSMO, i.e., it provides the same results as ddCOSMO for infinite dielectric constants independent of the discretization parameters used;
%\item robust, i.e., free from numerical instabilities;
%\item suitable for geometry optimization and molecular dynamics simulations, i.e., the solvation energy should be a smooth function of the nuclear positions;
%\item efficient and linear scaling in computational cost and memory requirements with respect to the number of atoms of the solute.
%\end{enumerate}
%The scheme that we propose, which we call ddPCM, satisfies all requirements but the linear scaling one; the latter appears to be feasible but requires substantial new developments and will be subject to further investigations.

%A straightforward application of Schwarz's domain decomposition to the PCM problem is, unfortunately, not possible. A qualitative argument to understand why this is the case is the following: as the PCM problem is defined in the whole space, and not only inside the cavity, the domain would decompose in the spheres that make the cavity plus the space occupied by the dielectric. The latter subdomain, which is not present in the COSMO, touches all the spheres which are exposed to the solvent, coupling them and destroying the sparsity of the resulting system of equations, making the overall strategy not effective. However, 

%\subsection{The ddPCM-method}


Let us recall how to solve equation \eqref{eq:IEFPCM} within the domain-decomposition paradigm. The first step is to write the IEF-PCM integral equation \eqref{eq:IEFPCM} as a succession of two integral equations, one of which is equivalent to the COSMO equation\cite{Cances_Librone_PCM}. Indeed, if we define $\Phi_\varepsilon = \cS \, \sigma$, equation \eqref{eq:IEFPCM} becomes
\begin{alignat}{3}
\cR_\varepsilon \, \Phi_\varepsilon & = \cR_\infty \, \Phi \qquad && \text{on }\Gamma  \label{eq:ddPCM-1} \\
\cS \, \sigma & = -\Phi_\varepsilon  && \text{on }\Gamma \label{eq:ddPCM-2} 
\end{alignat}
The ddPCM strategy is an extension of ddCOSMO in the following sense: first, equation \eqref{eq:ddPCM-1} is solved in order to compute the right-hand side $-\Phi_\varepsilon$ of equation \eqref{eq:ddPCM-2}; secondly, ddCOSMO is employed to solve equation \eqref{eq:ddPCM-2} with the modified potential $-\Phi_\varepsilon$, and compute the solvation energy $E_s$.

In order to discuss the domain-decomposition approach employed for both steps, let us introduce some notation. As anticipated, we take the cavity $\Omega$ be the union of $M$ spheres $\Omega_j = B(x_j, r_j)$ with boundaries $\Gamma_j$. Let $U_j: \Gamma_j \to \mathbb{R}$ be the characteristic function of $\Gamma_j^\text{ext}:= \Gamma_j \cap \Gamma$, and define extensions $\Phi_j , \Phi_{\varepsilon,j} : \Gamma_j \to \mathbb{R}$ as  $\Phi_j (s)= U_j(s) \, \widetilde{\Phi}(s)$ and $\Phi_{\varepsilon,j}(s)= U_j(s) \, \widetilde{\Phi}_{\varepsilon}(s)$ for $s \in \Gamma_j$, where $\tilde{(\cdot)}$ indicates the trivial extension to $\overline{\Omega}$.


{\bf Step 1.} 
%\[
%\Phi_j =
%\begin{cases}
%\Phi & \text{on } \Gamma_j^\text{ext} := \Gamma_j \cap \Gamma \\
%0 & \text{otherwise}
%\end{cases}
%\qquad , \qquad 
%\Phi_{\varepsilon,j} =
%\begin{cases}
%\Phi_\varepsilon & \text{on } \Gamma_j^\text{ext} := \Gamma_j \cap \Gamma \\
%0 & \text{otherwise}
%\end{cases}
%\]
% Let $\chi_k$ be the characteristic function of $\Omega_k$, and define coefficients $\omega_{jk}(s) = \chi_k(s) / |N_j(s)|$, where $N_j(s)$ is the set of all subdomains that contain $s \in \overline{\Omega}_j$, and $|N_j(s)|$ is its cardinality. Notice that $\sum_{k \in N_j(\,\cdot\,)} \omega_{jk}(\,\cdot\,)$ is the characteristic function of $\Gamma_j^\text{int} := \Gamma_j \setminus \Gamma_j^\text{ext}$, while $U_j(\,\cdot\,) = 1 - \sum_{k \in N_j(\,\cdot\,)} \omega_{jk}(\,\cdot\,)$ is the characteristic function of $ \Gamma_j^\text{ext}$. Thus, the extensions can be compactly defined as 
We enforce the integral equation \eqref{eq:ddPCM-1} on $\Gamma_j^\text{ext}$
%, i.e.
%\[
%	\cR_\varepsilon \, \Phi_\varepsilon  = \cR_\infty \, \Phi \qquad  \text{on }\Gamma_j^\text{ext}
%\]
along with the condition $\Phi_{\varepsilon,j}=0$ on $\Gamma_j^\text{int}$ combined within one equation
\begin{equation}
	\label{eq:ddPCM_aux1}
	\alpha(1 - U_j)\,\Phi_{\varepsilon,j} + U_j \, {\mathcal{R}}_{\varepsilon} \, \Phi_{\varepsilon} = U_j \, {\mathcal{R}}_{\infty} \, \Phi \qquad \text{on }\Gamma_j
\end{equation}
where $\alpha$ is an arbitrary nonzero scalar, and ${\mathcal{R}}_{\varepsilon} \, \Phi_{\varepsilon}$ and $ {\mathcal{R}}_{\infty} \, \Phi$ should be understood as their trivial extensions to $\Gamma_j$. Employing the extensions $\Phi_j , \Phi_{\varepsilon,j}$, the double layer operator $\cD$ can be decomposed as
\[
(\mathcal{D} \, \Phi ) (s) = ( \mathcal{D}_j \, \Phi_j )(s) + \sum_{k \ne j} (\tilde{\mathcal{D}}_k \, \Phi_k )(s) \qquad ; \qquad s \in \Gamma_j^\text{ext} \quad, \quad  j = 1 , \ldots , M
\]
where $\cD_j$ and $\tilde{\cD}_j$ are, respectively, the local double layer operator and the local double layer potential on $\Gamma_j$. 
We refer to\cite{Stamm_JCP_DDPCM} for concise details for these operators.
Thus, we can also decompose ${\mathcal{R}}_{\varepsilon}$ in \eqref{eq:ddPCM_aux1} as
\begin{equation}\label{eq:PCM_aux2}
({\mathcal{R}}_{\varepsilon} \, \Phi_{\varepsilon})(s)
=
({\mathcal{R}}_{\varepsilon,j} \, \Phi_{\varepsilon,j})(s) + \sum_{k \ne j} (\tilde{\mathcal{R}}_{\varepsilon,k} \, \Phi_{\varepsilon,k})(s)
%= {\mathcal{R}}_{\infty,j} \, \Phi_{j} + \sum_{k \ne j} \tilde{\mathcal{R}}_{\infty,k} \, \Phi_{k} 
 \qquad ; \qquad s \in \Gamma_j^\text{ext} \quad, \quad  j = 1 , \ldots , M
\end{equation}
%\[
%2 \pi \, \frac{\varepsilon + 1}{\varepsilon - 1} \, \Phi_{\varepsilon,j}(s) - ( \mathcal{D}_j \, \Phi_{\varepsilon,j} )(s) - \sum_{k \ne j} (\tilde{\mathcal{D}}_k \, \Phi_{\varepsilon,k})(s) 
%= 2 \pi  \, \Phi_j(s) -  (\mathcal{D}_j \, \Phi_j)(s) - \sum_{k \ne j} (\tilde{\mathcal{D}}_k \, \Phi_k)(s)
%\]
where the local operators ${\mathcal{R}}_{\varepsilon,j}$ and $\tilde{\mathcal{R}}_{\varepsilon,j}$ are defined as
\[
{\mathcal{R}}_{\varepsilon,j} = 2 \pi \frac{\varepsilon + 1}{\varepsilon - 1} \, \cI - {\mathcal{D}}_j  \qquad, \qquad
\tilde{\mathcal{R}}_{\varepsilon,j} =  - \tilde{\mathcal{D}}_j
\]
with the obvious extension to the case $\varepsilon = \infty$. 
%The integral equation \eqref{eq:PCMloc}, together with the homogenous condition $\Phi_{\varepsilon,j} = 0$ on $\Gamma_j^\text{int}$, constitute the local problem for $\Phi_{\varepsilon,j}$.
%
%An integral equation defined on the whole $\Gamma_j$, as opposed to $\Gamma_j^\text{ext}$, is obtained by means of the characteristic function of $\Gamma_j^\text{int}$, namely $1 - U_j$ as
%\[
%\alpha(1 - U_j)\Phi_{\varepsilon,j} + U_j \bigg( {\mathcal{R}}_{\varepsilon,j} \, \Phi_{\varepsilon,j} + \sum_{k \ne j} \tilde{\mathcal{R}}_{\varepsilon,k} \, \Phi_{\varepsilon,k}\bigg) = U_j \bigg( {\mathcal{R}}_{\infty,j} \, \Phi_{j} - \sum_{k \ne j} \tilde{\mathcal{R}}_{\infty,k} \, \Phi_{k} \bigg)\qquad \text{on }\Gamma_j
%\]
%where $\alpha$ in an arbitrary scalar. 
If we insert \eqref{eq:PCM_aux2} into \eqref{eq:ddPCM_aux1} and select $\alpha = 2\pi(\varepsilon + 1)/(\varepsilon - 1)$, we obtain a convenient form in terms of the local double layer potentials and double layer operators:
\begin{multline}\label{eq:1}
2\pi \, \frac{\varepsilon + 1}{\varepsilon - 1} \, \Phi_{\varepsilon,j} - U_j \bigg( {\mathcal{D}}_j \Phi_{\varepsilon,j} + \sum_{k \ne j} \tilde{\mathcal{D}}_{k} \, \Phi_{\varepsilon,k}  \bigg) = \\ 2 \pi \, U_j \Phi_j - U_j \bigg( {\mathcal{D}}_j \Phi_{j} + \sum_{k \ne j} \tilde{\mathcal{D}}_{k} \, \Phi_{k}  \bigg) \qquad \text{on }\Gamma_j
\end{multline}
This constitutes our domain-decomposition strategy for equation \eqref{eq:ddPCM-1}. It is important to remark that, because of the summation, every subdomain $\Omega_j$ interacts with all other subdomains. We anticipate that this contrasts with the ddCOSMO strategy for equation \eqref{eq:ddPCM-2}.

{\bf Step 2.} The restriction $W_j := W |_{\overline{\Omega}_j}$ is harmonic over the subdomain $\Omega_j$, thus it can be represented as 
\begin{equation}\label{eq:COSMOloc}
W_j(x) = (\tilde{\mathcal{S}}_j \,  \sigma_j) (x) \quad , \quad x \in \Omega_j \qquad ; \qquad
W_j(s) = (\mathcal{S}_j \,  \sigma_j) (s) \quad , \quad s \in \Gamma_j
\end{equation}
where $\sigma_j$ is an unknown surface charge, and $\cS_j$ and $\tilde{\cS}_j$ are, respectively, the single layer potential and the single layer operator on $\Gamma_j$. The local problems \eqref{eq:COSMOloc}, are coupled together by decomposing $W_j$ as
\begin{equation}\label{eq:5}
W_j(s) = - U_j(s) \, \Phi_{\varepsilon,j}(s) +  \big (1 - U_j(s)\big ) \, n_j(s) \, \sum_{k \in N_j} {W}_k(s) \qquad s \in \Gamma_j
\end{equation}
{\bf (replace by following, since characteristic functions are redundant)}
\begin{equation}
W_j(s) = - \Phi_{\varepsilon,j}(s) +  n_j(s) \, \sum_{k \in N_j} {W}_k(s) \qquad ; \qquad s \in \Gamma_j \quad , \quad j = 1, \ldots , M
\end{equation}
where $N_j$ is the set of all neighboring subdomains of $\Omega_j$, $W_k$ is understood as its trivial extension to $\Omega$, and $n_j$ is a normalization factor. If $s$ does not belong to any neighbor of $\Omega_j$, then $n_j(s)$ vanishes. Otherwise, $n_j(s)$ is the reciprocal of the number of neighbors. When we substitute the local problems \eqref{eq:COSMOloc} into the decomposition \eqref{eq:5}, and define $(\tilde{\cS}_{jk} \, \sigma_k) (s) = n_j(s) \, (\tilde{\cS}_k \, \sigma_k)(s)$, we obtain
\begin{equation}\label{eq:2}
\mathcal{S}_j \, \sigma_j  = -U_j \, \Phi_{\varepsilon,j} + \big (1 - U_j(s)\big ) \, \sum_{k \in N_j} \tilde{\mathcal{S}}_{jk} \, \sigma_k \qquad \text{on } \Gamma_j
\end{equation}
{\bf (replace by following, since characteristic functions are redundant)}
\begin{equation}\label{eq:2}
\mathcal{S}_j \, \sigma_j  = - \Phi_{\varepsilon,j} +  \sum_{k \in N_j} \tilde{\mathcal{S}}_{jk} \, \sigma_k \qquad \text{on } \Gamma_j
\end{equation}
As opposed to the local problem \eqref{eq:1} which features a global interaction of all subdomains, the ddCOSMO step \eqref{eq:2} is characterized by the interaction of subdomain $\Omega_j$ with only its neighbors. This results in a sparse, rather than dense, discrete operator.

We discretize equation \eqref{eq:1} and \eqref{eq:2} by expanding $\Phi_j$, $\Phi_{\varepsilon,j}$ and $\sigma_j$ as truncated series of spherical harmonics. If $Y_\ell^m$ indicates the spherical harmonic of degree $\ell$ and order $m$ on the unit sphere $\mathbb{S}$, we approximate the surface charge $\sigma_j$ as
\[
\sigma_j(s) = \sigma_j(x_j + r_j y) = \frac{1}{r_j}\sum_{\ell=0}^{L_\text{max}} \sum_{m = -\ell}^\ell [X_j]_\ell^m \, Y_\ell^m(y)
\]
for some unknown coefficients $X = [X_j]_\ell^m$ and a prescribed integer parameter ${L_\text{max}}$. Here $y$ is the variable on $\mathbb{S}$. We approximate $\Phi_{\varepsilon,j}$ and $\Phi_j$ in the same fashion, namely
\[
\Phi_{\varepsilon,j} = - \sum_{\ell=0}^{L_\text{max}} \sum_{m = -\ell}^\ell [G_j]_\ell^m \, Y_\ell^m \qquad , \qquad \Phi_j = -\sum_{\ell=0}^{L_\text{max}} \sum_{m = -\ell}^\ell [F_j]_\ell^m \, Y_\ell^m
\]
where $G = [G_j]_\ell^m$ and $F = [F_j]_\ell^m$ are the coefficients of the expansions, and the minus signs have been introduced for convenience. In the following, we shall use the condensed notation $\sum_{\ell ,m}$ to indicate the double sum. We interpret each local problem (\ref{eq:1}) and (\ref{eq:2}) in a variational setting that uses spherical harmonics as test functions, see Appendix \ref{app:mats}. We employ orthogonality conditions of the spherical harmonics, along with Lebedev grids to perform numerical quadrature to derive discretizations of the global problems \eqref{eq:ddPCM-1} and \eqref{eq:ddPCM-2}. Respectively, we obtain
\begin{equation}\label{eq:6}
A_\varepsilon \, G = A_\infty \, F \qquad , \qquad  L \, X = G
\end{equation}
and the expressions for the entries of the discrete operator $A_\varepsilon$ are given in \eqref{eq:ajj} and \eqref{eq:ajk}.

%\newpage
\section{A Brief Review of the ddPCM Strategy}\label{sec:review}

%\subsection{The Polarizable Continuum Model}
The foundation of Polarizable Continuum Solvation Models (PCSM's) is the assumption that the solvent in a solute-solvent system can be treated as either a dielectric, or a conducting continuum medium on the outside of the molecular cavity $\Omega$ of the solute. We follow the customary approach of taking the cavity to be the so-called Van der Waals cavity\cite{ReviewPCM_2005}, i.e., the union of spheres centered at each atom with radii coinciding with the van der Waals radii.
Within this approach, the topologically similar Solvent Accessible Surface (SAS) cavity can be treated as well. 
Models based on the Solvent Excluded Surface (SES) have also been proposed\cite{Quan_DDPCM_SES,C5CP03410H,Harbrecht2011,JCC:JCC21431}, but will not be considered here.

The electrostatic part of the solute-solvent interaction is given by $E_s = \tfrac{1}{2}\, f(\varepsilon)\,\int_\Omega \rho(x) W(x) \, dx$, where $f(\varepsilon)$ is an empirical scaling that depends on the dielectric constant of the solvent (and which is only applied in the case of the COSMO), $\rho$ is the charge density of the solute, and $W$ is the polarization potential of the solvent. The quantities $W$ and $E_s$ are usually referred to, respectively, as the reaction potential and the electrostatic contribution to the solvation energy. 

The reaction potential is defined as $W = \varphi - \Phi$, where $\varphi$ is the total electrostatic potential of the solute-solvent system and $\Phi$ is the potential of the solute \emph{in vacuo}. In the case of the PCM, the total potential $\varphi$ satisfies a (generalized) Poisson equation with suitable interface conditions\cite{Mennucci_JCP_IEF1,Mennucci_JMC_IEF2}. Indeed, if $\varepsilon_s$ is the macroscopic, zero-frequency relative dielectric permittivity of the solvent, and define $\varepsilon(x) = 1$ when $x \in \Omega$ and $\varepsilon(x) = \varepsilon_s$ otherwise, the reaction potential fulfills 
\begin{equation} 
\label{eq:pcmpde}
\left \{ 
\begin{alignedat}{4}
\Delta  W &= 0  &&\mbox{in } \sR^3\setminus \Gamma  \\
 \ [W] &= 0  &&\mbox{on } \Gamma\\
\  [\varepsilon \, \partial_\nu W] &= (\varepsilon_s-1) \partial_\nu \Phi &\qquad& \mbox{on } \Gamma
\end{alignedat} 
\right.
\end{equation}
Here $\Gamma=\partial\Omega$ is the boundary of the cavity, $\partial_\nu$ is the normal derivative on $\Gamma$, and $[\,\cdot\,]$ is the jump operator (inside minus outside) on $\Gamma$.

Recalling potential theory, $W$ can be represented as $W(x) = (\tilde{\mathcal{S}}\sigma)(x)$ when $x \in \sR^3 \setminus \Gamma$, or $W(s) = (\mathcal{S}\sigma)(s)$ when $s \in \Gamma$. 
The surface density $\sigma$ defined on $\Gamma$ is the so-called apparent surface charge,  $\tilde{\cS}$ is the single layer potential and $\cS$ is the single layer operator,  which is invertible \cite{Calderon}. 
Note that both $\tilde{\cS}$ and $\cS$ are based on the surface $\Gamma$.
It can be shown that $\sigma$ satisfies the equation $\sigma = 1/4\pi \, [ \partial_\nu W]$, so that it is possible to recast the PCM problem \eqref{eq:pcmpde} as a single integral equation for $\sigma$. In fact, if we define the operators 
\begin{equation}
 \label{eq:Reps}
 \cR_\varepsilon = 2\pi \frac{\varepsilon+1}{\varepsilon-1} \, \cI - \cD \qquad, \qquad \cR_\infty = 2\pi \, \cI - \cD
\end{equation}
where $\cI$ is the identity and $\cD$ is the double layer boundary operator (also based on $\Gamma$).
It can be shown\cite{ReviewPCM_2005} that the apparent surface charge satisfies
\begin{equation}
\label{eq:IEFPCM}
\cR_\varepsilon \, \cS \, \sigma = - \cR_\infty \, \Phi \qquad \text{on }\Gamma
\end{equation}
which is known as the IEF-PCM equation. It involves operators $\cR_\infty$ and $\cR_\varepsilon$, which are both invertible. Furthermore, when the dielectric constant $\varepsilon_s$ approaches infinity, the IEF-PCM equation simplifies to $\cS \, \sigma = - \Phi$ on $\Gamma$, which is the Integral Equation Formulation of the Conductor-like Screening Model (COSMO)\cite{Lipparini_JCP_VPCM}.

%\subsection{A domain decomposition approach}
%\label{ssec:AddApproach}
%The domain decomposition (dd) method is a methodology first introduced by Schwarz\cite{DD_Method} in the nineteenth century to solve partial differential equations defined on complex domains which can be decomposed as the union of simple, overlapping ones. Domain decomposition is an iterative strategy that consists in recasting the global (i.e., defined on the whole domain) problem in a collection of local problems (i.e., defined on each simple unit that constitutes the domain) with modified boundary conditions. 
%For each local domain, the boundary conditions depend on the global ones and on the local solutions defined on the neighboring domains: such conditions are updated at each iteration and convergence is reached when a stationary point is found. The main advantage of the dd-paradigm is that only overlapping domains exchange information, which, in turn, means that the system of local problems can be very sparse. 
%
%Recently, we have proposed a domain decomposition based strategy to solve the COSMO equations\cite{Cances_JCP_ddCOSMO,Lipparini_JCTC_ddCOSMO,Lipparini_JPCL_ddCOSMO,Lipparini_JCP_ddCOSMO-QM}; we called such a strategy ddCOSMO. ddCOSMO performs significantly better than any other discretization, exhibiting linear scaling in computational cost and a overall very low computational cost, which makes it two to three orders of magnitude faster than previous implementations. 
%Domain decomposition is a particularly viable approach for the COSMO problem, as its domain is the molecular cavity, which naturally decomposes into the atom-centered balls of which it is the union. Furthermore, the solution of the local problem in each ball is trivial and, as for large molecules only a small number of balls with respect to the total number will overlap, the resulting (linear) system of local equations is very sparse (as a rough estimate, each ball overlaps with at most 20-25 other balls). 
%We recapitulate here the ddCOSMO algorithm; all the details of its derivation, implementation, numerical performances and coupling with various methodologies to describe the solute can be found elsewhere.
%
%The COSMO equation \eqref{eq:COSMO} is recast as a system of linear PDEs defined in each ball:
%\begin{equation}
%\label{eq:ddcosmo1}
%\forall j=1,\ldots M \quad \left \{
%\begin{array}{rcll}
%\Delta W_j(\bx) &=& 0 &\qquad \bx \in \Omega_j, \\
%W_j(\bs) &=& h_j(\bs) &\qquad \bs \in \Gamma_j,
%\end{array}
%\right .
%\end{equation}
%where $\Gamma_j = \partial \Omega_j$ is the boundary of $\Omega_j$.
%Let $\Gamma_j^{\rm e}$ be the portion of the sphere $\Gamma_j $ which is exposed to the solvent, i.e., $\Gamma_j^{\rm e} = \Gamma \cap \Gamma_j$ and $\Gamma_j^{\rm i}$ the internal part. The boundary condition $h_j$ is defined as follows:
%\begin{equation}
%\label{eq:ddboundary}
% h_j(\bs) =
%\begin{cases}
% -\Phi(\bs), & \bs \in \Gamma_j^{\rm e}, \\
% \sum_{k\in\cN_j(\bs)} \frac{1}{|\cN_j(\bs)|}W_k(\bs) & \bs \in \Gamma_j^{\rm i},
%\end{cases} 
%\end{equation}
%where $\cN_j(\bs)$ is the list of balls that overlap $\Omega_j$ at $\bs$ and $ |\cN_j(\bs)|$ the number of such balls.
%We can now exploit the representation formula \eqref{eq:SL1}, as the local functions $W_j$ are local single-layer potentials 
%%As in the case of the global representation, the local harmonic function $W_j$ can be represented by a (now local) single layer potential
%\begin{equation}
%\label{eq:SL1-local}
%\forall \bx \in \Omega_j,\quad W_j(\bx) = (\tilde{\mathcal{S}_j}\sigma)(\bx) := \int_{\Gamma_j} \frac{\sigma_j(\by)}{|\bx - \by|} d\by,
%\end{equation}
%with $\sigma_j(\bs)$ being the local ASC associated with $W_j$. 
%We particularly emphasize that $W_j$ only coincides with $W$ in $\Omega_j$. 
%We can now rewrite the ddCOSMO boundary condition in a more compact fashion. Let $\chi_j$ the characteristic function of $\Omega_j$, i.e., 
%\begin{equation}
%	\label{eq:CharFct}
% \chi_j(\br) =
% \begin{cases}
%  1 & \br \in \Omega_j, \\
%  0 & \mbox{otherwise}.
% \end{cases}
%\end{equation}
%and let 
%\begin{equation}
%	\label{eq:LocCharFct}
%	\forall \bs \in \Gamma_j\quad \omega_{jk}(\bs) = \frac{\chi_k(\bs)}{|\cN_j(\bs)|}, \quad U_j(\bs) = 1 - \sum_{k\in \cN_j(\bs)} \omega_{jk}(\bs).
%\end{equation}
%Eq. \eqref{eq:ddboundary} can be rewritten as
%\begin{equation}
%\label{eq:ddbound2}
% h_j(\bs) = -U_j(\bs)\Phi(\bs) + \sum_{k\in\cN_j(\bs)} \omega_{jk}(\bs) (\tilde{\cS}_k\sigma_k)(\bs),
%\end{equation}
%where the subscript $k$ to the operator $\tilde{\cS}$ refers to the fact that the integral is to be performed on $\Gamma_k$ as introduced in  \eqref{eq:ddcosmo1}.
%Finally, by using the representation formula \eqref{eq:SL2} for the local problem, i.e. using 
%\begin{equation}
%	\label{eq:SL2-local}
%	\forall \bs \in  \Gamma_j,\quad W_j(\bs) = (\mathcal{S}_j\sigma_j)(\bs) := \int_{\Gamma_j} \frac{\sigma_j(\bs')}{|\bs - \bs'|} d\bs',
%\end{equation}
%we get the ddCOSMO system of coupled integral equations:
%\begin{equation}
% \label{eq:ddCOSMO}
% \forall \bs \in \Gamma_j \quad (\cS_j\sigma_j)(\bs) = -U_j(\bs)\Phi(\bs) + \sum_{k\in \cN_j(\bs)} \omega_{jk}(\bs)(\tilde{\cS}_{k}\sigma_k)(\bs).
%\end{equation}
%The last step to obtain the ddCOSMO algorithm is to discretize the local ASCs and the integral operators in a suitable basis of functions. A truncated set of real spherical harmonics is the natural choice: by expanding $\sigma_j$ as
%\[
%\sigma_j(\bR_j + r_j\by) = \sum_{l=0}^N\sum_{m=-l}^l [X_j]_l^m Y_l^m(\by),
%\]
%where $\by$ is a unit vector and we write a generic point $\bs \in \Gamma_j$ as $\bs = \bR_j + r_j\by$, we obtain
%\begin{equation}
%\label{eq:ddCOSMO-disc}
% \left (
%\begin{array}{ccc}
% L_{11} & \ldots & L_{1M} \\
% \vdots & \ddots & \vdots \\
% L_{M1} & \ldots & L_{MM} \\
%\end{array}
% \right )
% \left (
%\begin{array}{c}
%X_1\\
%\vdots\\
%X_M
%\end{array}
% \right )
%=
% \left (
%\begin{array}{c}
%g_1\\
%\vdots\\
%g_M
%\end{array}
% \right ),
%\end{equation}
%where 
%\begin{equation}
%	\label{eq:BraKet}
%	[ L_{jj}]_{ll'}^{mm'} =  \langle lm |\cS_j| l'm'\rangle,  \quad 	[L_{jk}]_{ll'}^{mm'} = - \langle lm|\omega_{jk}\tilde{\cS}_{k}|l'm'\rangle
%\end{equation}
%and
%\begin{equation}
%	\label{eq:ddCOSOMrhs}
%	[g_j]_l^m = -\langle lm| U_j \Phi\rangle.
%\end{equation}
%The expression of the matrix elements can be found in ref. \citenum{Cances_JCP_ddCOSMO}. Notice that $\omega_{jk}(\bs)$ vanishes if $\Omega_k$ is not overlapping with $\Omega_j$, which makes the $L_{jk}$-blocks vanish if they do not correspond to overlapping spheres: the ddCOSMO matrix is therefore sparse\cite{Cances_JCP_ddCOSMO}.
%
%\subsection{The PCM within the ddCOSMO paradigm}
%The aim of this article is to propose an efficient computational strategy for the Polarizable Continuum Model (PCM) based on a domain decomposition approach. Before discussing such a strategy, let us recapitulate what are, according to us, the requirements for a good discretization. Such a discretization must be
%\begin{enumerate}
%\item simple, i.e., controlled by a small number of parameters;
%\item systematically improvable, i.e., the numerical solution converges to the exact solution as the parameters are refined;
%\item consistent with ddCOSMO, i.e., it provides the same results as ddCOSMO for infinite dielectric constants independent of the discretization parameters used;
%\item robust, i.e., free from numerical instabilities;
%\item suitable for geometry optimization and molecular dynamics simulations, i.e., the solvation energy should be a smooth function of the nuclear positions;
%\item efficient and linear scaling in computational cost and memory requirements with respect to the number of atoms of the solute.
%\end{enumerate}
%The scheme that we propose, which we call ddPCM, satisfies all requirements but the linear scaling one; the latter appears to be feasible but requires substantial new developments and will be subject to further investigations.

%A straightforward application of Schwarz's domain decomposition to the PCM problem is, unfortunately, not possible. A qualitative argument to understand why this is the case is the following: as the PCM problem is defined in the whole space, and not only inside the cavity, the domain would decompose in the spheres that make the cavity plus the space occupied by the dielectric. The latter subdomain, which is not present in the COSMO, touches all the spheres which are exposed to the solvent, coupling them and destroying the sparsity of the resulting system of equations, making the overall strategy not effective. However, 

%\subsection{The ddPCM-method}


Let us recall how to solve equation \eqref{eq:IEFPCM} within the domain-decomposition paradigm. The first step is to write the IEF-PCM integral equation \eqref{eq:IEFPCM} as a succession of two integral equations, one of which is equivalent to the COSMO equation\cite{Cances_Librone_PCM}. Indeed, if we define $\Phi_\varepsilon = \cS \, \sigma$, equation \eqref{eq:IEFPCM} becomes
\begin{alignat}{3}
\cR_\varepsilon \, \Phi_\varepsilon & = \cR_\infty \, \Phi \qquad && \text{on }\Gamma  \label{eq:ddPCM-1} \\
\cS \, \sigma & = -\Phi_\varepsilon  && \text{on }\Gamma \label{eq:ddPCM-2} 
\end{alignat}
The ddPCM strategy is an extension of ddCOSMO in the following sense: first, equation \eqref{eq:ddPCM-1} is solved in order to compute the right-hand side $-\Phi_\varepsilon$ of equation \eqref{eq:ddPCM-2}; secondly, ddCOSMO is employed to solve equation \eqref{eq:ddPCM-2} with the modified potential $-\Phi_\varepsilon$, and compute the solvation energy $E_s$.

In order to discuss the domain-decomposition approach employed for both steps, let us introduce some notation. As anticipated, we take the cavity $\Omega$ be the union of $M$ spheres $\Omega_j = B(x_j, r_j)$ with boundaries $\Gamma_j$. Let $U_j: \Gamma_j \to \mathbb{R}$ be the characteristic function of $\Gamma_j^\text{ext}:= \Gamma_j \cap \Gamma$, and define extensions $\Phi_j , \Phi_{\varepsilon,j} : \Gamma_j \to \mathbb{R}$ as  $\Phi_j (s)= U_j(s) \, \widetilde{\Phi}(s)$ and $\Phi_{\varepsilon,j}(s)= U_j(s) \, \widetilde{\Phi}_{\varepsilon}(s)$ for $s \in \Gamma_j$, where $\tilde{(\cdot)}$ indicates the trivial extension to $\overline{\Omega}$.


{\bf Step 1.} 
%\[
%\Phi_j =
%\begin{cases}
%\Phi & \text{on } \Gamma_j^\text{ext} := \Gamma_j \cap \Gamma \\
%0 & \text{otherwise}
%\end{cases}
%\qquad , \qquad 
%\Phi_{\varepsilon,j} =
%\begin{cases}
%\Phi_\varepsilon & \text{on } \Gamma_j^\text{ext} := \Gamma_j \cap \Gamma \\
%0 & \text{otherwise}
%\end{cases}
%\]
% Let $\chi_k$ be the characteristic function of $\Omega_k$, and define coefficients $\omega_{jk}(s) = \chi_k(s) / |N_j(s)|$, where $N_j(s)$ is the set of all subdomains that contain $s \in \overline{\Omega}_j$, and $|N_j(s)|$ is its cardinality. Notice that $\sum_{k \in N_j(\,\cdot\,)} \omega_{jk}(\,\cdot\,)$ is the characteristic function of $\Gamma_j^\text{int} := \Gamma_j \setminus \Gamma_j^\text{ext}$, while $U_j(\,\cdot\,) = 1 - \sum_{k \in N_j(\,\cdot\,)} \omega_{jk}(\,\cdot\,)$ is the characteristic function of $ \Gamma_j^\text{ext}$. Thus, the extensions can be compactly defined as 
We enforce the integral equation \eqref{eq:ddPCM-1} on $\Gamma_j^\text{ext}$
%, i.e.
%\[
%	\cR_\varepsilon \, \Phi_\varepsilon  = \cR_\infty \, \Phi \qquad  \text{on }\Gamma_j^\text{ext}
%\]
along with the condition $\Phi_{\varepsilon,j}=0$ on $\Gamma_j^\text{int}$ combined within one equation
\begin{equation}
	\label{eq:ddPCM_aux1}
	\alpha(1 - U_j)\,\Phi_{\varepsilon,j} + U_j \, {\mathcal{R}}_{\varepsilon} \, \Phi_{\varepsilon} = U_j \, {\mathcal{R}}_{\infty} \, \Phi \qquad \text{on }\Gamma_j
\end{equation}
where $\alpha$ is an arbitrary nonzero scalar, and ${\mathcal{R}}_{\varepsilon} \, \Phi_{\varepsilon}$ and $ {\mathcal{R}}_{\infty} \, \Phi$ should be understood as their trivial extensions to $\Gamma_j$. Employing the extensions $\Phi_j , \Phi_{\varepsilon,j}$, the double layer operator $\cD$ can be decomposed as
\[
(\mathcal{D} \, \Phi ) (s) = ( \mathcal{D}_j \, \Phi_j )(s) + \sum_{k \ne j} (\tilde{\mathcal{D}}_k \, \Phi_k )(s) \qquad ; \qquad s \in \Gamma_j^\text{ext} \quad, \quad  j = 1 , \ldots , M
\]
where $\cD_j$ and $\tilde{\cD}_j$ are, respectively, the local double layer operator and the local double layer potential on $\Gamma_j$. 
We refer to\cite{Stamm_JCP_DDPCM} for concise details for these operators.
Thus, we can also decompose ${\mathcal{R}}_{\varepsilon}$ in \eqref{eq:ddPCM_aux1} as
\begin{equation}\label{eq:PCM_aux2}
({\mathcal{R}}_{\varepsilon} \, \Phi_{\varepsilon})(s)
=
({\mathcal{R}}_{\varepsilon,j} \, \Phi_{\varepsilon,j})(s) + \sum_{k \ne j} (\tilde{\mathcal{R}}_{\varepsilon,k} \, \Phi_{\varepsilon,k})(s)
%= {\mathcal{R}}_{\infty,j} \, \Phi_{j} + \sum_{k \ne j} \tilde{\mathcal{R}}_{\infty,k} \, \Phi_{k} 
 \qquad ; \qquad s \in \Gamma_j^\text{ext} \quad, \quad  j = 1 , \ldots , M
\end{equation}
%\[
%2 \pi \, \frac{\varepsilon + 1}{\varepsilon - 1} \, \Phi_{\varepsilon,j}(s) - ( \mathcal{D}_j \, \Phi_{\varepsilon,j} )(s) - \sum_{k \ne j} (\tilde{\mathcal{D}}_k \, \Phi_{\varepsilon,k})(s) 
%= 2 \pi  \, \Phi_j(s) -  (\mathcal{D}_j \, \Phi_j)(s) - \sum_{k \ne j} (\tilde{\mathcal{D}}_k \, \Phi_k)(s)
%\]
where the local operators ${\mathcal{R}}_{\varepsilon,j}$ and $\tilde{\mathcal{R}}_{\varepsilon,j}$ are defined as
\[
{\mathcal{R}}_{\varepsilon,j} = 2 \pi \frac{\varepsilon + 1}{\varepsilon - 1} \, \cI - {\mathcal{D}}_j  \qquad, \qquad
\tilde{\mathcal{R}}_{\varepsilon,j} =  - \tilde{\mathcal{D}}_j
\]
with the obvious extension to the case $\varepsilon = \infty$. 
%The integral equation \eqref{eq:PCMloc}, together with the homogenous condition $\Phi_{\varepsilon,j} = 0$ on $\Gamma_j^\text{int}$, constitute the local problem for $\Phi_{\varepsilon,j}$.
%
%An integral equation defined on the whole $\Gamma_j$, as opposed to $\Gamma_j^\text{ext}$, is obtained by means of the characteristic function of $\Gamma_j^\text{int}$, namely $1 - U_j$ as
%\[
%\alpha(1 - U_j)\Phi_{\varepsilon,j} + U_j \bigg( {\mathcal{R}}_{\varepsilon,j} \, \Phi_{\varepsilon,j} + \sum_{k \ne j} \tilde{\mathcal{R}}_{\varepsilon,k} \, \Phi_{\varepsilon,k}\bigg) = U_j \bigg( {\mathcal{R}}_{\infty,j} \, \Phi_{j} - \sum_{k \ne j} \tilde{\mathcal{R}}_{\infty,k} \, \Phi_{k} \bigg)\qquad \text{on }\Gamma_j
%\]
%where $\alpha$ in an arbitrary scalar. 
If we insert \eqref{eq:PCM_aux2} into \eqref{eq:ddPCM_aux1} and select $\alpha = 2\pi(\varepsilon + 1)/(\varepsilon - 1)$, we obtain a convenient form in terms of the local double layer potentials and double layer operators:
\begin{multline}\label{eq:1}
2\pi \, \frac{\varepsilon + 1}{\varepsilon - 1} \, \Phi_{\varepsilon,j} - U_j \bigg( {\mathcal{D}}_j \Phi_{\varepsilon,j} + \sum_{k \ne j} \tilde{\mathcal{D}}_{k} \, \Phi_{\varepsilon,k}  \bigg) = \\ 2 \pi \, U_j \Phi_j - U_j \bigg( {\mathcal{D}}_j \Phi_{j} + \sum_{k \ne j} \tilde{\mathcal{D}}_{k} \, \Phi_{k}  \bigg) \qquad \text{on }\Gamma_j
\end{multline}
This constitutes our domain-decomposition strategy for equation \eqref{eq:ddPCM-1}. It is important to remark that, because of the summation, every subdomain $\Omega_j$ interacts with all other subdomains. We anticipate that this contrasts with the ddCOSMO strategy for equation \eqref{eq:ddPCM-2}.

{\bf Step 2.} The restriction $W_j := W |_{\overline{\Omega}_j}$ is harmonic over the subdomain $\Omega_j$, thus it can be represented as 
\begin{equation}\label{eq:COSMOloc}
W_j(x) = (\tilde{\mathcal{S}}_j \,  \sigma_j) (x) \quad , \quad x \in \Omega_j \qquad ; \qquad
W_j(s) = (\mathcal{S}_j \,  \sigma_j) (s) \quad , \quad s \in \Gamma_j
\end{equation}
where $\sigma_j$ is an unknown surface charge, and $\cS_j$ and $\tilde{\cS}_j$ are, respectively, the single layer potential and the single layer operator on $\Gamma_j$. The local problems \eqref{eq:COSMOloc}, are coupled together by decomposing $W_j$ as
\begin{equation}\label{eq:5}
W_j(s) = - U_j(s) \, \Phi_{\varepsilon,j}(s) +  \big (1 - U_j(s)\big ) \, n_j(s) \, \sum_{k \in N_j} {W}_k(s) \qquad s \in \Gamma_j
\end{equation}
{\bf (replace by following, since characteristic functions are redundant)}
\begin{equation}
W_j(s) = - \Phi_{\varepsilon,j}(s) +  n_j(s) \, \sum_{k \in N_j} {W}_k(s) \qquad ; \qquad s \in \Gamma_j \quad , \quad j = 1, \ldots , M
\end{equation}
where $N_j$ is the set of all neighboring subdomains of $\Omega_j$, $W_k$ is understood as its trivial extension to $\Omega$, and $n_j$ is a normalization factor. If $s$ does not belong to any neighbor of $\Omega_j$, then $n_j(s)$ vanishes. Otherwise, $n_j(s)$ is the reciprocal of the number of neighbors. When we substitute the local problems \eqref{eq:COSMOloc} into the decomposition \eqref{eq:5}, and define $(\tilde{\cS}_{jk} \, \sigma_k) (s) = n_j(s) \, (\tilde{\cS}_k \, \sigma_k)(s)$, we obtain
\begin{equation}\label{eq:2}
\mathcal{S}_j \, \sigma_j  = -U_j \, \Phi_{\varepsilon,j} + \big (1 - U_j(s)\big ) \, \sum_{k \in N_j} \tilde{\mathcal{S}}_{jk} \, \sigma_k \qquad \text{on } \Gamma_j
\end{equation}
{\bf (replace by following, since characteristic functions are redundant)}
\begin{equation}\label{eq:2}
\mathcal{S}_j \, \sigma_j  = - \Phi_{\varepsilon,j} +  \sum_{k \in N_j} \tilde{\mathcal{S}}_{jk} \, \sigma_k \qquad \text{on } \Gamma_j
\end{equation}
As opposed to the local problem \eqref{eq:1} which features a global interaction of all subdomains, the ddCOSMO step \eqref{eq:2} is characterized by the interaction of subdomain $\Omega_j$ with only its neighbors. This results in a sparse, rather than dense, discrete operator.

We discretize equation \eqref{eq:1} and \eqref{eq:2} by expanding $\Phi_j$, $\Phi_{\varepsilon,j}$ and $\sigma_j$ as truncated series of spherical harmonics. If $Y_\ell^m$ indicates the spherical harmonic of degree $\ell$ and order $m$ on the unit sphere $\mathbb{S}$, we approximate the surface charge $\sigma_j$ as
\[
\sigma_j(s) = \sigma_j(x_j + r_j y) = \frac{1}{r_j}\sum_{\ell=0}^{L_\text{max}} \sum_{m = -\ell}^\ell [X_j]_\ell^m \, Y_\ell^m(y)
\]
for some unknown coefficients $X = [X_j]_\ell^m$ and a prescribed integer parameter ${L_\text{max}}$. Here $y$ is the variable on $\mathbb{S}$. We approximate $\Phi_{\varepsilon,j}$ and $\Phi_j$ in the same fashion, namely
\[
\Phi_{\varepsilon,j} = - \sum_{\ell=0}^{L_\text{max}} \sum_{m = -\ell}^\ell [G_j]_\ell^m \, Y_\ell^m \qquad , \qquad \Phi_j = -\sum_{\ell=0}^{L_\text{max}} \sum_{m = -\ell}^\ell [F_j]_\ell^m \, Y_\ell^m
\]
where $G = [G_j]_\ell^m$ and $F = [F_j]_\ell^m$ are the coefficients of the expansions, and the minus signs have been introduced for convenience. In the following, we shall use the condensed notation $\sum_{\ell ,m}$ to indicate the double sum. We interpret each local problem (\ref{eq:1}) and (\ref{eq:2}) in a variational setting that uses spherical harmonics as test functions, see Appendix \ref{app:mats}. We employ orthogonality conditions of the spherical harmonics, along with Lebedev grids to perform numerical quadrature to derive discretizations of the global problems \eqref{eq:ddPCM-1} and \eqref{eq:ddPCM-2}. Respectively, we obtain
\begin{equation}\label{eq:6}
A_\varepsilon \, G = A_\infty \, F \qquad , \qquad  L \, X = G
\end{equation}
and the expressions for the entries of the discrete operator $A_\varepsilon$ are given in \eqref{eq:ajj} and \eqref{eq:ajk}.
 
\section{Computation of the ddPCM-Forces}\label{sec:forces}

\subsection{Theory}
%In the case of a classical solute, the form of the charge density $\rho$ allows to compute 
The solvation energy can be written as a sum of subdomain contributions, which perfectly fits the domain-decomposition paradigm. 

First, for a classical solute's charge distribution of the form of $\rho=\sum_j q_j \delta_{x_j}$, we can develop
\[
	E_s 
	= \frac{1}{2} \,  \int_{\Omega} \rho(x) W(x) \, dx
	= \frac{1}{2} \, \sum_j q_j  W(x_j)
	= 2 \pi \, \sum_j q_j [X_j]_0^0    \, Y_0^0
	= {\sqrt{\pi}}\, \sum_j q_j [X_j]_0^0
\]
This concept easily generalizes to point multipolar charge distributions. 
The evaluation of the energy for charge distributions is a bit more involved as it requires a three-dimensional integration and we refer to \cite{Lipparini_JCP_ddCOSMO-QM} for more details. In all cases however, the energy can be written as
\[
E_s = \tfrac{1}{2}
% \,f(\varepsilon)
 \, \sum_j \sum_{\ell,m} [\Psi_j]_\ell^m [X_j]_\ell^m
  =: \tfrac{1}{2} 
  %\,f(\varepsilon) 
  \,\langle \Psi, X \rangle
\]
where the angular brackets indicate the double scalar product over $j$ and $\ell,m$.
For example, for the classical charge as illustrated above, we have 
\[
	[\Psi_j]_\ell^m = {\sqrt{\pi}}\, q_j \, \delta_{\ell,0} \delta_{m,0}.
\]

% THE FOLLOWING IS NOT ENTIRELY CORRECT SINCE THE OVERLAP IN THE INTEGRALS SHOULD BE SUBSTRACTED
%Indeed, the spherical harmonics addition theorem implies that
%\[
%W_j(x) = (\tilde{\cS}_j \, \sigma_j)(x) = \sum_{\ell,m} [X_j]_\ell^m (\tilde{\cS}_j \, Y_\ell^m)(x) = \sum_{\ell,m} [X_j]_\ell^m \, \frac{4\pi}{2\ell+1} \,\frac{r^\ell}{r_j^{\ell+1}} \, Y_\ell^m(y)
%\]
%where $x = x_j + r\, y$, so that the solvation energy can be determined as
%\[
%E_s = \tfrac{1}{2} 
%%\,f(\varepsilon) 
%\, \sum_{j=1}^M \int_{\Omega_j} \rho(x) W_j(x) \, dx = \tfrac{1}{2}
%%\, f(\varepsilon) 
%\, \sum_{j=1}^M \sum_{\ell,m} [X_j]_\ell^m \, \frac{4\pi}{2\ell+1} \,\frac{1}{r_j^{\ell+1}} 
%\int_{\Omega_j} \rho(x) \, r^\ell \, Y_\ell^m(y) \, dx
%\]
%If we define
%\[
%[\Psi_j]_\ell^m = \frac{4\pi}{2\ell+1}\, \frac{1}{r_j^{\ell+1}}\int_{\Omega_j} \rho(x) \, r^\ell \, Y_\ell^m(y) \, dx
%\]
%we can compactly write the energy as
%\[
%E_s = \tfrac{1}{2}
%% \,f(\varepsilon)
% \, \sum_j \sum_{\ell,m} [\Psi_j]_\ell^m [X_j]_\ell^m
%  =: \tfrac{1}{2} 
%  %\,f(\varepsilon) 
%  \,\langle \Psi, X \rangle
%\]
%where the angular brackets indicate the double scalar product over $j$ and $\ell,m$.

The force acting on the $i$-th atom is then given by
\[
\mathcal{F}_i = -\nablai E_s = - \tfrac{1}{2} 
%\,f(\varepsilon) 
\,  \langle \Psi, \nablai X \rangle 
% = - \langle \Psi, \nabla_j\sigma \rangle
\]
where the gradient $\nablai$ is understood with respect to $x_i$ and $X$ denotes the solution to the ddPCM-equations~\eqref{eq:6}. 
Here we used the fact that the entries of the vector $\Psi$ are independent of the atomic positions. On the other hand, since both the matrices $A_\varepsilon$, $A_\infty$ and the right-hand side $F$ depend on the nuclear positions $x_1 , \ldots, x_M$, so does the unknown $X$ of~\eqref{eq:6}, which can compactly be written as $A_\varepsilon \, L \, X = A_\infty \, F$.
The idea adjoint differentiation is to consider the adjoint problem $(A_\varepsilon \, L)^* s = \Psi$ and compute the quantity $\langle \Psi, \nablai X \rangle = \langle s ,  A_\varepsilon \, L \, \nablai X\rangle$ by using the definition of the adjoint matrix.

%If combine equations \eqref{eq:6}, the fully discretized problem becomes $A_\varepsilon \, L \, X = A_\infty \, F$, and 
Applying the Leibnitz differentiation rule to $A_\varepsilon \, L \, X = A_\infty \, F$ allows to move the derivative from $X$ onto the other terms, namely
\[
A_\varepsilon \, L \, \nablai X = \nablai A_\infty \, F +  A_\infty \, \nablai F - \nablai A_\varepsilon \, L \, X -  A_\varepsilon \, \nablai L \, X=: h_i
\]
Thus, once the solution $s$ of the adjoint problem and vector $h_i$ have been determined, the forces can be computed as
\[
\mathcal{F}_i =  - \tfrac{1}{2} 
%\,f(\varepsilon)
 \,  \langle s, h_i \rangle
\]
The derivatives $\nablai L $ of the ddCOSMO discretization were discussed in \cite{Lipparini_JCTC_ddCOSMO} and we now focus on the new parts due to the ddPCM-method. We remark that the quantity $\nablai F$ is \emph{a priori} nonzero since $F_j$ does depend upon the nuclear positions through the characteristic function $U_j$, recall \eqref{eq:25}.

Let $\{ s_n\}$ be the $N_\text{grid}$ Lebedev integration points with associated weights $\{ w_n \}$ and define the following quantities
\[
t_n^{jk} = \frac{|x_j + r_j s_n -x_k|}{r_k} \quad , \quad s_n^{jk} = \frac{x_j + r_j s_n -x_k}{|x_j + r_j s_n -x_k|}\quad , \quad U_j^n = U_j(x_j + r_j s_n)
\]
The entries of $F_j$ are evaluated through numerical quadrature as
\[
[F_j]_\ell^m = - \sum_{n = 1} ^{N_\text{grid}} w_n \, U_j^n \, \Phi(x_j + r_js_n) \, Y_\ell^m(s_n)
\]
so that we immediately obtain
\[
[\nablai F_j]_\ell^m = - \sum_{n = 1} ^{N_\text{grid}} w_n \, \nablai U_j^n \, \Phi(x_j + r_js_n) \, Y_\ell^m(s_n)
\]
In the remainder of this section we discuss the derivatives of the ddPCM matrix $A_\varepsilon$.

The blocks $A_{jk}^\varepsilon$ of the ddPCM matrix $A_\varepsilon$, see \eqref{eq:ajj} and \eqref{eq:ajk}, have the form
\begin{alignat*}{3}
{[A_{jj}^\varepsilon]}_{\ell \ell'}^{mm'}& = 2\pi \, \frac{\varepsilon + 1}{\varepsilon - 1}\, \delta_{\ell \ell'} \delta_{m m'}&& + \frac{2\pi}{2 \ell' + 1} \,\sum_{n= 1}^{N_\text{grid}} w_n \, U_j^n  \,Y_\ell^m(s_n) \,  Y_{\ell'}^{m'}(s_n) \\
{[A_{jk}^\varepsilon]}_{\ell \ell'}^{mm'}& =&& -  \frac{4 \pi \ell'}{2 \ell'+1} \, \sum_{n= 1}^{N_\text{grid}} w_n\, U_j^n  \, Y_\ell^m(s_n) \, \big( t_n^{jk}\big)^{-(\ell'+1)} \, Y_{\ell'}^{m'} (s_n^{jk})
\end{alignat*}
and, since the derivatives are independent of $\varepsilon$, we drop the $\varepsilon$-dependency for ease of notation.

The case of the diagonal blocks yields
\[
{[\nablai A_{jj}]}_{\ell \ell'}^{mm'} = \frac{2\pi}{2 \ell' + 1} \,\sum_{n} w_n \, \nablai U_j^n  \,Y_\ell^m(s_n) \,  Y_{\ell'}^{m'}(s_n)
\]
so that it only requires the derivatives of the characteristic function. The function $U_j$ is, in practice, a smoothed version of the (discontinuous!)~characteristic function, and is defined as 
\[
U_j(x_j + r_j y) =
\begin{cases}
1 - f_j(y) 	&\quad f_j(y) \le 1\\
0		&\quad \text{otherwise}
\end{cases}
\qquad , \qquad 
f_j(y) = \sum_{k \in N_j} \chi \bigg(\frac{|x_j + r_j y - x_k|}{r_k}\bigg)
\]
where $y$ varies on $\mathbb{S}$ and $\chi$ is a regularized characteristic function of $[0,1]$. We conclude that $\nablai U_j$ and, consequently, $\nablai A_{jj}$ are \emph{a priori} nonzero only when $i \in N_j$ or $i = j$.

The case of the off-diagonal blocks ${[ A_{jk}]}_{\ell \ell'}^{mm'}$ with $j \not=k$ is more involved since it includes the gradient of the product of three functions, namely
\begin{equation}\label{eq:7}
{[\nablai A_{jk}]}_{\ell \ell'}^{mm'} = -  \frac{4 \pi \ell'}{2 \ell'+1} \, \sum_{n} w_n\, Y_\ell^m(s_n) \,\nablai \Big[ U_j^n  \,  \big( t_n^{jk}\big)^{-(\ell'+1)} \, Y_{\ell'}^{m'} (s_n^{jk}) \Big]
\end{equation}
%\cdots\Big[ \nabla U_j^n  \,  \big( t_n^{jk}\big)^{-(\ell'+1)} \, Y_{\ell'}^{m'} (s_n^{jk}) + U_j^n  \, \nabla \Big( \big( t_n^{jk}\big)^{-(\ell'+1)} \Big)\, Y_{\ell'}^{m'} (s_n^{jk}) + U_j^n  \,  \big( t_n^{jk}\big)^{-(\ell'+1)} \, \nabla Y_{\ell'}^{m'} (s_n^{jk}) \Big]
%\end{multline*}
However, since $t_n^{jk}$ and $s_n^{jk}$ depend only upon $x_j$ and $x_k$, if we assume $i \not=j$ and $i\not=k$, we obtain
\begin{equation}\label{eq:8}
{[\nablai A_{jk}]}_{\ell \ell'}^{mm'} = -  \frac{4 \pi \ell'}{2 \ell'+1} \, \sum_{n} w_n\, Y_\ell^m(s_n) \,\nablai U_j^n  \,  \big( t_n^{jk}\big)^{-(\ell'+1)} \, Y_{\ell'}^{m'} (s_n^{jk})
\end{equation}
%Such relation holds, in particular, for $i \in N_j$ and $i \not= k$.
Thus, since $U_j^n$ depends only upon $x_i$ if $i \in N_j$, we conclude that $\nablai A_{jk}$ vanishes whenever $i \not= j$ and $i \not=k$ and $i \not\in N_j$. In order to discuss the opposite case, i.e., $i = j$ or $i =k$ or $i \in N_j$, notice that the events $(i = j)$ and $(i = k)$ are mutually exclusive, as are $(i = j)$ and $(i \in N_j)$. We obtain the three subcases $i = j$, and $i = k$, and $i \in N_j \, , \, i \not= k$, which will be addressed individually.

Standard differentiation implies that
\begin{multline}\label{eq:9}
\nablai \Big[ U_j^n  \,  \big( t_n^{jk}\big)^{-(\ell'+1)} \, Y_{\ell'}^{m'} (s_n^{jk}) \Big] = \nablai U_j^n  \,  \big( t_n^{jk}\big)^{-(\ell'+1)} \, Y_{\ell'}^{m'} (s_n^{jk}) \\
- U_j^n  \, (\ell' + 1)  \big( t_n^{jk}\big)^{-(\ell'+2)} \, \nablai t_n^{jk} \, Y_{\ell'}^{m'} (s_n^{jk}) + U_j^n  \,  \big( t_n^{jk}\big)^{-(\ell'+1)} \, \big( \Di\, s_n^{ji} \big)^T\, \nablai Y_{\ell'}^{m'} (s_n^{jk})
\end{multline}
where $\Di$ emphasizes that the gradient of the vector quantity $s_j^{jk}$ is indeed its Jacobian matrix and where the extra subscripts refer to the variables with respect to which differentiation is taken. We proceed to evaluate $\nablai t_n^{jk}$ and $ \Di \, s_n^{jk}$. When $i = j$, differentiation implies
\[
\nablaj t_n^{jk} = \frac{s_n^{jk}}{r_k} \qquad , \qquad \Dj \, s_n^{jk} = \frac{I - s_n^{jk} \otimes s_n^{jk} }{|x_j + r_j s_n - x_k|^3}
\]
where $I$ is the identity matrix and $\otimes$ indicates the outer product. We remark that the Jacobian matrix $\Dj \, s_n^{jk}$ is symmetric, so that the transpose in \eqref{eq:9} is redundant. 
Due to the particular relation between $x_j$ and $x_k$, we obtain $\nabla_{\! j} \, t_n^{jk} = - \nabla_{\! k} \, t_n^{jk}$ and $D_j \, s_n^{jk} = - D_k \, s_n^{jk}$.
We can therefore analogously derive the case $i = k$ and those relationships imply
\begin{multline*}
\nabla_{\! j} \Big[ U_j^n  \,  \big( t_n^{jk}\big)^{-(\ell'+1)} \, Y_{\ell'}^{m'} (s_n^{jk}) \Big] + \nabla_{\! k} \Big[ U_j^n  \,  \big( t_n^{jk}\big)^{-(\ell'+1)} \, Y_{\ell'}^{m'} (s_n^{jk}) \Big] = \\
  \Big[ \nabla_{\! j} \, U_j^n + \nabla_{\! k} \, U_j^n  \Big]  \big( t_n^{jk}\big)^{-(\ell'+1)} \, Y_{\ell'}^{m'} (s_n^{jk})
\end{multline*}
which provide a convenient way of evaluating $[\nabla_{\! k} \, A_{jk}]_{\ell \ell'}^{m m'}$ from $[\nabla_{\! j} \, A_{jk}]_{\ell \ell'}^{m m'}$. In fact, we obtain the quasi-skew-symmetric relation
\[
[ \nabla_{\! j} \, A_{jk}]_{\ell \ell'}^{m m'} + [\nabla_{\! k} \, A_{jk}]_{\ell \ell'}^{m m'} = -  \frac{4 \pi \ell'}{2 \ell'+1} \, \sum_{n} w_n\, Y_\ell^m(s_n) \Big[ \nabla_{\! j} \, U_j^n + \nabla_{\! k} \, U_j^n  \Big]  \big( t_n^{jk}\big)^{-(\ell'+1)} \, Y_{\ell'}^{m'} (s_n^{jk})
\]
for $\nablai A_{jk}$. Finally, the case $i \in N_j \, , \, i \not= k$ reduces to \eqref{eq:8}.

\subsection{Implementation}
We solve the adjoint problem $(A_\varepsilon \, L )^* s = \Psi$ through the two steps ${L}^* \, y = \Psi$ and ${A_\varepsilon}^* \, s = y$, and apply Leibnitz differentiation rule to the PCM problem \eqref{eq:6}, so that
\[
\nablai A_\varepsilon \, G + A_\varepsilon \, \nablai G = \nablai A_\infty \, F + A_\infty \, \nablai F  \qquad , \qquad \nablai L \, X + L \, \nablai X = \nablai G
\]
Since $\nablai A_\infty=\nablai A_\varepsilon =: \nablai A$, we obtain
\begin{align*}
\langle \Psi , \nablai X\rangle &  = \langle s ,  A_\varepsilon \, L \, \nablai X\rangle \\
&  = \langle s ,  A_\varepsilon \,( \nablai G - \nablai L \,  X ) \rangle \\ 
& = \langle s ,  A_\varepsilon \, \nablai G \rangle - \langle s , A_\varepsilon \, \nablai L \,  X \rangle \\
& = \langle s , \nablai  A ( G - F) + A_\infty \, \nablai F \rangle - \langle s , A_\varepsilon \, \nablai L \,  X \rangle \\
& = \langle s , \nablai  A ( G - F) \rangle + \langle {A_\infty}^* \, s , \nablai F \rangle - \langle {A_\varepsilon}^* \, s , \nablai L \,  X \rangle \\
& = \langle s , \nablai  A ( G - F) \rangle + \langle {A_\infty}^* \, s , \nablai F \rangle - \langle y , \nablai L \,  X \rangle
\end{align*}
Notice that, when $\varepsilon = \infty$, then $G = F$ and ${A_\infty}^* \, s =y$, so that we recover
\[
\langle \Psi , \nablai X\rangle =  \langle y , \nablai F \rangle - \langle y , \nablai L \,  X \rangle
\]
which is the expression of the force for COSMO. Because of the sparsity of $L$, the computation of the action $\nablai L \, X$ is of complexity $O(N_\text{grid} \times M)$, with a prefactor that depends, al least, upon $N_i$ and $L_\text{max}$. Since the contraction product $\langle \cdot \, , \cdot \rangle$ also has complexity $O(M)$, with a prefactor that depends on $L_\text{max}$, we conclude that the cost is $O(N_\text{gird} \times M)$. In the case of PCM, the bottleneck is the action $ \nablai  A ( G - F)$...

1. Use that $\nablai A_\infty=\nablai A_\varepsilon$ and that $L \, X=G$:
\begin{alignat*}{1}
h_i &= \nablai A_\infty \, F +  A_\infty \, \nablai F - \nablai A_\varepsilon \, L \, X -  A_\varepsilon \, \nablai L \, X\\
&= \nablai A \, (F-  G) +  A_\infty \, \nablai F  -  A_\varepsilon \, \nablai L \, X
\end{alignat*}

2. Explain how to do the following operations efficiently:
\begin{alignat*}{3}
\langle s, \nablai A \, (F-  G) \rangle &= \langle (\nablai A)^* \, s,  F-  G \rangle ? \\
\langle s, A_\infty \, \nablai F \rangle &= \langle A_\infty \,^* \, s,  \nablai F \rangle \\
\langle s, A_\varepsilon \, \nablai L \, X \rangle &= \langle (A_\varepsilon\, \nablai L)^* \, s,  X \rangle ? \\
\end{alignat*}
I don't see how the first and third term are efficiently done in practice?

3. Use the skew-symmetric relationship from above.
 
\section{Numerical Experiments}\label{sec:experiments}
\subsection{Convergence tests}
We first verify the implementation of the computation of the forces.
Let us denote by $	\mathcal{F}_{i,\alpha}$ the $\alpha$-component of the force  $	\mathcal{F}_{i}$ for $\alpha=1,2,3$. Let $e_\alpha$ be the canonical unit vectors of $\mathbb R^3$.
Then, for any atomic position $x_i$ we consider a sequence $\delta_1,\delta_2,\ldots$ and consider the approximate force obtained by finite difference
\[
	\mathcal{F}_{i,\alpha}
	\approx
	\mathcal{F}_{i,\alpha}(\delta_n)
	= 
	\frac{E_s(x_1,\ldots,x_i + \delta_n e_\alpha,\ldots,x_M) - E_s(x_1,\ldots,x_i,\ldots,x_M)}{\delta_n},
\]
where we have made explicit the dependency of the solvation energy on the nuclear positions.

In table \ref{tab:}, we illustrate the RMS of the approximate forces with $\delta_n = x^{-n}$ 
for caffeine.

\ldots


\subsection{Timings}
Timings for different molecular structures depending on the number of atoms (i.e. alanine chains, hemoglobin, etc). 
\section{Conclusions}\label{sec:conclusions}
We presented analytical gradients of the Polarizable Continuum Model electrostatic solvation energy computed using the domain decomposition strategy. The complete derivation is offered in this paper, and a correctly scaling implementation is discussed. Numerical experiments prove not only the correctness of the implementation via the comparison of analytical and numerical derivatives, but also the predicted scaling of the computational cost with respect to the number of atoms of our implementation.
The pilot implementation presented in this work is a starting point for future work. While the overall timings here presented are acceptable and compare well with the ones obtained with different discretizations, the computational effort required by the solution of the ddPCM equations and the computation of the ddPCM forces is still large, and makes routine applications on large and very large systems not pursuable. In particular, while a more efficient implementation can be achieved and a better parallelization strategy implemented, the quadratic scaling of the computation represents a barrier that cannot be circumvented. For this reason, a linear scaling implementation based on the use of the Fast Multipole Method is currently under investigation. Furthermore, we have presented results for classical solutes only, i.e., solutes described by a classical force field. An implementation of ddPCM in the framework of quantum chemical methods is particularly attractive. ddPCM shares with ddCOSMO the rigorous foundations, variational discretization and overall simplicity and limited number of parameters. Furthermore, even the quadratic scaling code has a smaller constant than BEM-based discretization (which scale quadratically in the number of total grid points, i.e., like $\mathcal{O}(M^2N^2_g)$, to be compared with $\mathcal{O}(M^2N_g)$ for ddPCM) and is expected to be competitive. As the systems that can be treated with quantum mechanical methods are well within the sizes accessible by ddPCM, we expect the method to be already fully applicable in the framework of quantum chemistry. 
 


\section*{Acknowledgments}
P.G.~and B.S.~acknowledge funding from the German Academic Exchange Service (DAAD) from funds of the ``Bundesministeriums f\"ur Bildung und Forschung'' (BMBF) for the project Aa-Par-T (Project-ID 57317909).
\appendix

\section{ddPCM discretization\label{app:pcm}}
The derivation ddPCM discrete operator $A^\varepsilon_{jk}$ rests upon the fact that the spherical harmonics $Y_\ell^m$ are eigenfunctions of the double layer operator on $\mathbb{S}$, i.e., $\cD \, Y_\ell^m =  -2\pi/ (2\ell+1) \,  Y_\ell^m$, along with the following jump relation for the double layer potential
\begin{equation}\label{eq:jump}
	\lim_{\delta \to + 0} \big(\tilde{\cD} \, Y_\ell^m\big)(y \pm \delta \nu) =  \pm 2\pi \, Y_\ell^m(y)+ ( \cD \, Y_\ell^m )(y)
\end{equation}
where $\nu$ denotes the outward normal at $y \in \mathbb{S}$. We shall employ the invariance by translation and scaling $(\mathcal{D}_j \, \Phi_{\varepsilon,j})(s) = (\mathcal{D} \, \hat{\Phi}_{\varepsilon,j})(y)$, where $y = (s - x_j)/r_j$ and $\hat{\Phi}_{\varepsilon,j}$ is defined on $\mathbb{S}$ through the push-forward-like transformation $\hat{\Phi}_{\varepsilon,j}(y) = \Phi_{\varepsilon,j}(s)$. We begin by discussing the diagonal term $2\pi f_\varepsilon \, \Phi_{\varepsilon,j} - U_j \, \cD_j \, \Phi_{\varepsilon,j}$ of \eqref{eq:1} where, for brevity, we set $f_\varepsilon = (\varepsilon + 1)/(\varepsilon - 1)$. As customary, in order to obtain a numerical discretization, we multiply by a test function $\varphi$ and integrate over $\Gamma_j$.  The change of variable $y = (s- x_j)/r_j$, yields an integral over $\mathbb{S}$ which involves the hatted quantities, namely
\[
\int_{\Gamma_j} \big(2\pi f_\varepsilon \, \Phi_{\varepsilon,j}  - U_j \, \cD_j \, \Phi_{\varepsilon,j} \big) \, \varphi = 2 \pi  \, f_\varepsilon \, r_j^2 \,\int_\mathbb{S} \hat{\Phi}_{\varepsilon,j}  \, \hat{\varphi} - r_j^2 \, \int_\mathbb{S} \hat{U}_j \, \cD \, \hat{\Phi}_{\varepsilon,j}  \, \hat{\varphi}
\]
We proceed to expand $\hat{\Phi}_{\varepsilon,j}$ as a series of spherical harmonics with coefficients $-[G_j]_{\ell'}^{m'}$, and select as a test function $\hat{\varphi}$  the spherical harmonic $Y_{\ell}^{m}$. The orthogonality of the spherical harmonics, together with the fact that they are eigenfunctions of the double layer potential, yield
\begin{multline*}
\int_{\Gamma_j} \big(2\pi f_\varepsilon \, \Phi_{\varepsilon,j}  - U_j \, \cD_j \, \Phi_{\varepsilon,j} \big) \, \varphi = \\
= - 2 \pi  \, f_\varepsilon \, r_j^2 \,\sum_{\ell',m'}  \, [G_j]_{\ell'}^{m'} \, \delta_{\ell \ell'} \delta_{mm'}  - r_j^2 \, \sum_{\ell',m'} \, [G_j]_{\ell'}^{m'} \, \frac{2\pi}{2\ell'+1} \int_{\mathbb{S}} \hat{U}_j \,  Y_{\ell'}^{m'}\, Y_\ell^m
\end{multline*}
The last step to obtain the diagonal block $A_{jj}^\varepsilon$ is to approximate the integral through a suitable quadrature formula with weights $\{w_n\}$ and nodes $\{ s_n\}$. Once the numerical quadrature is carried out and the spherical harmonics expansion is truncated at $\ell' = L_\text{max}$, we derive the final expression
\begin{equation}\label{eq:ajj}
[A_{jj}^\varepsilon]_{\ell \ell'}^{mm'} = 2\pi \, f_\varepsilon %\frac{\varepsilon+1}{\varepsilon-1}
 \,  \delta_{\ell \ell'} \delta_{mm'} + \frac{2\pi}{2\ell'+1} \sum_{n=1}^{N_\text{grid}} \, w_n \, \hat{U}_j(s_n)  \, Y_{\ell'}^{m'}(s_n)\,  Y_\ell^m(s_n)
\end{equation}
where we have divided by a factor $r_j^2$, since we anticipate that we can do the same on the remaining terms. The computation of the off-diagonal term $-U_j \, \tilde{\cD}_k \, \Phi_{\varepsilon,k}$ employs again the fact that the double layer operator is invariant under translation and scaling, i.e., $(\tilde{\cD}_k \, \Phi_{\varepsilon,k} )(x) = (\tilde{\cD} \, \hat{\Phi}_{\varepsilon,k})(u)$ where $x \in \mathbb{R}^3 \setminus \overline{\Omega}_k$ and $u = (x -x_k)/ r_k$. In particular, when $x \in \Gamma_j$, i.e., $x = s = x_j + r_j y$ for some $y \in \mathbb{S}$, then $u = u(y) = (x_j + r_j y -x_k)/r_k$. As the quantity $U_j \, \tilde{\cD}_k \, \Phi_{\varepsilon,k}$ in indeed well-defined on the whole $\Gamma_j$, we can proceed as before and obtain
\begin{multline*}
\int_{\Gamma_j}U_j(s) \, (\tilde{\cD}_k \, \Phi_{\varepsilon,k})(s) \, \varphi(s) \, ds
 = r_j^2  \int_{\mathbb{S}} \hat{U}_j(y) \, (\tilde{\cD} \, \hat{\Phi}_{\varepsilon,k})(u(y)) \, Y_{\ell}^{m}(y) \, dy = \\ %= -\sum_{\ell ,m} \, [G_j]_\ell^m \,  \int_\mathbb{S} \hat{U}_j \, \tilde{\cD} \, \hat{\Phi}_{\varepsilon,k} \, Y_{\ell'}^{m'}  %\simeq \sum_{n=1}^{N_\text{grid}} \, w_n \, \hat{U}_j(s_n) \,   \big( \tilde{\cD}_k \, \hat{\Phi}_{\varepsilon,k}\big)(s_n) \, Y_{\ell'}^{m'}(s_n)
= - r_j^2 \, \sum_{\ell',m'} \, [G_k]_{\ell'}^{m'} \,  \int_{\mathbb{S}} \hat{U}_j(y) \, (\tilde{\cD} \, Y_{\ell'}^{m'})(u(y)) \, Y_{\ell}^{m}(y) \, dy 
\end{multline*}
where $-[G_k]_{\ell'}^{m'}$ are the coefficients of the expansion of $ \hat{\Phi}_{\varepsilon,k}$ as a series of spherical harmonics. The function $\tilde{\cD} \, Y_{\ell'}^{m'}$ is harmonic on $\mathbb{R}^3 \setminus \overline{B(0,1)}$, so that is has to coincide with the unique harmonic extension of its boundary value. The jump relation \eqref{eq:jump}, along with the eigenfunction property, provide the boundary value
\[
\lim_{\delta \; \to \; + 0} \big(\tilde{\cD} \, Y_{\ell'}^{m'}\big)(y + \delta \nu) =  2\pi \,Y_{\ell'}^{m'}(y)+ \big( \cD \, Y_{\ell'}^{m'} \big)(y) = \frac{4 \pi \ell'}{2 \ell +' 1} \, Y_{\ell'}^{m'}(y)
\]
and, by elementary notions on harmonic functions, we conclude
\[
(\tilde{\cD} \, Y_{\ell'}^{m'})(u) = \frac{4 \pi \ell'}{2 \ell' + 1} \, \frac{1}{|u|^{\ell' + 1}} \,  Y_{\ell'}^{m'}(u/|u|)
\]
After truncation the series expansion and performing numerical integration we obtain the final result
\begin{equation}\label{eq:ajk}
[A_{jk}]_{\ell \ell'}^{m m'} =-  \frac{4 \pi \ell'}{2 \ell' + 1}  \sum_{n=1}^{N_\text{grid}} \, w_n  \,  \hat{U}_j(s_n) \, \frac{1}{|u(s_n)|^{\ell' + 1}} \, Y_{\ell'}^{m'}\left(\frac{u(s_n)}{|u(s_n)|}\right) \, Y_{\ell}^{m}(s_n)
\end{equation}
where again we have divided by factor $r_j^2$. The discretization of the right-hand-side of the ddPCM equation yields $A_\infty \, F$, where $A_\infty$ is almost the same as $A_\varepsilon$. In fact, the term $f_\varepsilon$ appearing in the diagonal term \eqref{eq:ajj} is replaced by $f_\infty = 1$. This justifies the fact that we can divide through by the $r_j^2$ factor.
\section{ddCOSMO discretization (Paolo)\label{app:mats}}
We multiply equation \eqref{eq:2} by a test function $\varphi$, integrate over $\Gamma_j$, namely
\begin{equation}\label{eq:50}
\int_{\Gamma_j} \cS_j \, \sigma_j \, \varphi - \sum_{k \in N_j} \int_{\Gamma_j} \tilde{\mathcal{S}}_{jk} \, \sigma_k \, \varphi = -\int_{\Gamma_j} U_j \, \Phi_{\varepsilon,j} \, \varphi
\end{equation}
and proceed to map quantities to the unit sphere $\mathbb{S}$. In order to do so, we employ the translation-invariant properties $(\cS_j \, \sigma_j)(s) = r_j( \cS \, \hat{\sigma}_j)(z)$ and $(\tilde{\cS}_j \, \sigma_j)(x) = r_j( \tilde{\cS} \, \hat{\sigma}_j)(u)$, where $\cS$ and $\tilde{\cS}$ are, respectively, the single layer operator and potential on the unit sphere, $y = (s - x_j)/r_j$, $u = (x - x_j)/r_j$, and $\hat{\sigma}_j(y) = \sigma_j(s)$. The workhorse for the objects $\cS \, \hat{\sigma}_j$ and $\tilde{\cS} \, \hat{\sigma}_j$ is the Addition Theorem for spherical harmonics.

We expand $\hat{\sigma}_j$ through spherical harmonics $Y_\ell^m$ as
\[
\hat{\sigma}_j(y) = \frac{1}{r_j} \, \sum_{\ell'= 0}^{\infty} \sum_{m' = -\ell'}^{\ell'} \,  [X_j]_{\ell'}^{m'} \, Y_{\ell'}^{m'}(y)
\]
for some (unknown!)~coefficients $X_j = [X_j]_{\ell'}^{m'}$, so that the orthogonality of the spherical harmonics implies
\[
(\cS \, \hat{\sigma}_j)(y) = \frac{1}{r_j} \, \sum_{\ell'= 0}^{\infty} \sum_{m' = -\ell'}^{\ell'} \, \frac{4\pi}{(2\ell' + 1)}\, [X_j]_{\ell'}^{m'} \, Y_{\ell'}^{m'}(y)
\]
Thus, it is natural to select as a test functions the spherical harmonics, so that
\begin{equation}\label{eq:51}
\int_{\Gamma_j} \cS_j \, \sigma_j \, \varphi =  r_j^3 \int_\mathbb{S} \cS \, \hat{\sigma}_j \, Y_\ell^m = r_j^2 \, \sum_{\ell'= 0}^{\infty} \sum_{m' = -\ell'}^{\ell'} \, \frac{4\pi}{(2\ell' + 1)} \, \delta_{\ell \ell'} \delta_{m m'}\, [X_j]_{\ell'}^{m'} 
\end{equation}
due to the orthogonality of the spherical harmonics. A numerical discretization is obtained by truncation the expansion at $\ell'=L_\text{max}$, thus providing the discrete action $L_{jj} \, X_j$, where $L_{jj} = [L_{jj}]_{\ell \ell'}^{m m'}$ is a diagonal matrix.

The discretization of $\int_{\Gamma_j} \tilde{\cS}_{jk} \, \sigma_k \, \varphi $ follows the same strategy. As a preliminary step, we manipulate the integral as
\[
\int_{\Gamma_j} \tilde{\cS}_{jk} \, \sigma_k \, \varphi =\int_{\Gamma_j} n_j \, \tilde{\cS}_{k} \, \sigma_k \, \varphi = \int_{\Gamma_j} (1 - U_j ) \,n_j \, \tilde{\cS}_{k} \, \sigma_k \, \varphi = \int_{\Gamma_j} V_j \, \tilde{\cS}_{k} \, \sigma_k \, \varphi
\]
where we inserted the characteristic function $U_j$ for convenience, and defined the rescaled characteristic function $V_j : \Gamma_j \to \mathbb{R}$ as $V(s) = (1 - U_j(s)) \, n_j(s)$. We remark that the quantity $V_j \, \tilde{\cS}_{k} \, \sigma_k $ is indeed well defined on the whole $\Gamma_j$, so that the integral is legitimate. The reason for introducing the characteristic function is that we can modify it, e.g., replace it by a smooth counterpart, to improve robustness of the algorithm. When $x \in \Gamma_j$, i.e., $x = s = x_j + r_j y$, then $u = u(y) = (x_j + r_j y -x_k)/r_k$ and $|u| < 1$ because of the assumption $k \in N_j$. Analogously to the previous case, the Addition Theorem implies
\[
(\tilde{\cS} \, \hat{\sigma}_k)(u) = \frac{1}{r_k} \, \sum_{\ell'= 0}^{\infty} \sum_{m' = -\ell'}^{\ell'} \, \frac{4\pi}{(2\ell' + 1)}\, [X_k]_{\ell'}^{m'} \, |u|^{\ell'} \, Y_{\ell'}^{m'}(u/|u|)
\]
so that we obtain
\begin{multline*}
\int_{\Gamma_j} V_j(s) \, (\tilde{\cS}_{k} \, \sigma_k)(s) \, \varphi(s) \, ds =  r_j^2 \,r_k \int_\mathbb{S} \hat{V}_j(y) \, (\tilde{\cS} \, \hat{\sigma}_k)(u(y)) \, Y_\ell^m(y) \, dy = \\
= r_j^2 \, \sum_{\ell'= 0}^{\infty} \sum_{m' = -\ell'}^{\ell'} \, \frac{4\pi}{(2\ell' + 1)}\, [X_k]_{\ell'}^{m'} \int_\mathbb{S} \hat{V}_j(y) \, |u(y)|^{\ell'} \, Y_{\ell'}^{m'}(u(y)/|u(y)|) \, Y_\ell^m(y) \, dy
\end{multline*}
As opposed to the diagonal term, the integration has to be carried out numerically. We employ a Lebedev grid with $N_\text{grid}$ nodes $\{s_n\}$ and weights $\{ w_n \}$ to perform numerical quadrature, so that
\begin{equation}\label{eq:52}
L_{jk} \, X_k = r_j^2 \, \sum_{\ell'= 0}^{L_\text{max}}  \sum_{m' = -\ell'}^{\ell'} \, \frac{4\pi}{(2\ell' + 1)}\, \sum_{n = 1}^{N_\text{grid}} w_n \, \hat{V}_j(s_n) \, |u(s_n)|^{\ell'} \, Y_{\ell'}^{m'}(u(s_n)/|u(y)|) \, Y_\ell^m(s_n)\, [X_k]_{\ell'}^{m'}  
\end{equation}

Finally, we discuss the right-hand-side of \eqref{eq:50} when $\Phi_{\varepsilon,j}$ is provided by the ddPCM step as a truncated series of spherical harmonics with coefficients $-[G_j]_{\ell'}^{m'}$. Indeed, we obtain
\begin{equation}\label{eq:53}
-\int_{\Gamma_j} U_j \, \Phi_{\varepsilon,j} \, \varphi =  - r_j^2 \int_\mathbb{S} \hat{U}_j \, \hat{\Phi}_{\varepsilon,j} \, Y_\ell^m = r_j^2 \,  \sum_{\ell'= 0}^{L_\text{max}} \sum_{m' = -\ell'}^{\ell'} \, [G_j]_{\ell'}^{m'} \,  \int_{\mathbb{S}} \hat{U}_j \, Y_{\ell'}^{m'} \, Y_\ell^m 
\end{equation}
and we have to resort to numerical quadrature to evaluate the integral.  Notice that the factor $r_j^2$ appears in \eqref{eq:51}, \eqref{eq:52} and \eqref{eq:53}, so that we can effectively cross it out.
\section{Derivatives of ddPCM discretization}\label{app:pcm_der}

The function $U_j$ is, in practice, a smoothed version of the characteristic function. We use the following construction
\[
U_j(x_j + r_j y) =
\begin{cases}
1 - f_j(y) 	&\quad f_j(y) \le 1\\
0		&\quad \text{otherwise}
\end{cases}
\qquad , \qquad 
f_j(y) = \sum_{i \in N_j} \chi \bigg(\frac{|x_j + r_j y - x_i|}{r_i}\bigg)
\]
where $\chi$ is a regularized characteristic function of $[0,1]$. As previously, the variable $y$ varies on the unit sphere $\mathbb{S}$. This definition implies that $U_j$ depends upon the nuclear positions $x_j$ and $x_i$ such that $i \in N_j$.



Let $\{ s_n\}$ be $N_\text{grid}$ integration points, with associated weights $\{ w_n \}$, and define the following quantities
\[
t_n^{jk} = \frac{|x_j + r_j s_n -x_k|}{r_k} \qquad , \qquad s_n^{jk} = \frac{x_j + r_j s_n -x_k}{|x_j + r_j s_n -x_k|}%\quad , \quad U_j^n = U_j(x_j + r_j s_n)
\]
It is evident that $t_n^{jk}$ depends only upon the nuclear positions $x_j$ and $x_k$, as does $s_n^{jk}$. Standard differentiation yields
\begin{equation}\label{eq:75}
\nablaj t_n^{jk} = \frac{s_n^{jk}}{r_k} \qquad , \qquad \Dj \, s_n^{jk} = \frac{I - s_n^{jk} \otimes s_n^{jk} }{|x_j + r_j s_n - x_k|^3}
\end{equation}
where $I$ is the identity matrix and $\otimes$ indicates the outer product. We remark that the Jacobian matrix $\Dj \, s_n^{jk}$ is symmetric, and the ``twin'' relations $\nabla_{\! j} \, t_n^{jk} = - \nabla_{\! k} \, t_n^{jk}$ and $D_j \, s_n^{jk} = - D_k \, s_n^{jk}$ hold.

If we employ this notation and recall \eqref{eq:ajj} and \eqref{eq:ajk}, the blocks of the ddPCM operator can be written as
\begin{alignat*}{3}
{[A_{jj}^\varepsilon]}_{\ell \ell'}^{mm'}& = 2\pi \, g(\varepsilon_s) \, \delta_{\ell \ell'} \delta_{m m'}&& + \frac{2\pi}{2 \ell' + 1} \,\sum_{n= 1}^{N_\text{grid}} w_n \, \hat{U}_j(s_n)  \,Y_\ell^m(s_n) \,  Y_{\ell'}^{m'}(s_n) \\
{[A_{jk}^\varepsilon]}_{\ell \ell'}^{mm'}& =&& -  \frac{4 \pi \ell'}{2 \ell'+1} \, \sum_{n= 1}^{N_\text{grid}} w_n\, \hat{U}_j(s_n) \, Y_\ell^m(s_n) \, \big( t_n^{jk}\big)^{-(\ell'+1)} \, Y_{\ell'}^{m'} (s_n^{jk})
\end{alignat*}
The dependency of the diagonal block $A_{jj}^\varepsilon$ upon the nuclear positions occurs only through the characteristic function $U_j$. On the other hand, the off-diagonal block $A_{jk}^\varepsilon$ interacts with the nuclear positions through the characteristic function $U_j$, as well as $t_n^{jk}$ and $s_n^{jk}$. This implies that $\nablai A_{jj}$ is \emph{a priori} nonzero only when $i = j$ or $i \in N_j$. Similarly, $\nablai A_{jk}$ is \emph{a priori} nonzero only when $i = j$ or $i = k$ or $i \in N_j$.


The case of the diagonal blocks yields
\[
{[\nablai A_{jj}]}_{\ell \ell'}^{mm'} = \frac{2\pi}{2 \ell' + 1} \,\sum_{n=1}^{N_\text{grid}} w_n \, \nablai \hat{U}_j(s_n)  \,Y_\ell^m(s_n) \,  Y_{\ell'}^{m'}(s_n)
\]
so that only the derivatives of the characteristic function are required. The case of the off-diagonal blocks is significantly more involved, since it involves the gradient of the product of three functions, namely
\begin{equation}\label{eq:7}
{[\nablai A_{jk}]}_{\ell \ell'}^{mm'} = -  \frac{4 \pi \ell'}{2 \ell'+1} \, \sum_{n = 1}^{N_\text{grid}} w_n\, Y_\ell^m(s_n) \,\nablai \Big[ \hat{U}_j(s_n)  \,  \big( t_n^{jk}\big)^{-(\ell'+1)} \, Y_{\ell'}^{m'} (s_n^{jk}) \Big]
\end{equation}
We proceed to analyze the gradient of the triple product when $i = j$, or $i = k$, or $i \in N_j \, , \, i \not= k$. Those three cases are mutually exclusive.

The case $i = j$ yields no simplification, and the gradient of the triple product has the three standard contributions, namely
\begin{multline}\label{eq:9}
\nablaj \big[ \cdots \big] = \nablaj \hat{U}_j(s_n)  \,  \big( t_n^{jk}\big)^{-(\ell'+1)} \, Y_{\ell'}^{m'} (s_n^{jk}) \\
+ \hat{U}_j(s_n)  \, \nablaj \Big(  \big( t_n^{jk}\big)^{-(\ell'+1)} \Big) \, Y_{\ell'}^{m'} (s_n^{jk}) +  \hat{U}_j(s_n)  \,  \big( t_n^{jk}\big)^{-(\ell'+1)} \, \nablaj Y_{\ell'}^{m'} (s_n^{jk})
%\nablaj \Big[ \hat{U}_j(s_n)  \,  \big( t_n^{jk}\big)^{-(\ell'+1)} \, Y_{\ell'}^{m'} (s_n^{jk}) \Big] = \nablaj \hat{U}_j(s_n)  \,  \big( t_n^{jk}\big)^{-(\ell'+1)} \, Y_{\ell'}^{m'} (s_n^{jk}) \\
%- \hat{U}_j(s_n)  \, (\ell' + 1)  \big( t_n^{jk}\big)^{-(\ell'+2)} \, \nablaj t_n^{jk} \, Y_{\ell'}^{m'} (s_n^{jk}) + \hat{U}_j(s_n)  \,  \big( t_n^{jk}\big)^{-(\ell'+1)} \, \big( \Dj\, s_n^{jk} \big)^T\, \nablaj Y_{\ell'}^{m'} (s_n^{jk})
\end{multline}
In the implementation the differentiation is fully carried out through the chain rule, which employs the formulas \eqref{eq:75} for the derivatives of $t_n^{jk}$ and $s_n^{jk}$.

The case $i =k$ needs to be split into the subcases $k \in N_j$ and $k \not \in N_j$. The first subcase does not yield any simplification and reduces to \eqref{eq:9}. On the other hand, when $k \not\in N_j$, the first term on the right-hand-side of \eqref{eq:9} drops out, and we obtain
\[
\nablak \big[ \cdots \big] = 
 \hat{U}_j(s_n)  \, \nablak \Big(  \big( t_n^{jk}\big)^{-(\ell'+1)} \Big) \, Y_{\ell'}^{m'} (s_n^{jk}) +  \hat{U}_j(s_n)  \,  \big( t_n^{jk}\big)^{-(\ell'+1)} \, \nablak Y_{\ell'}^{m'} (s_n^{jk})
\]

Finally, when $i \in N_j$ and $i \not=k$ the second and third term on the right-hand-side of \eqref{eq:9} vanish, and we obtain 
\[
\nablai \big[ \cdots \big] = \nablaj \hat{U}_j(s_n)  \,  \big( t_n^{jk}\big)^{-(\ell'+1)} \, Y_{\ell'}^{m'} (s_n^{jk})
\]

This conclude our discussion on the derivative of the PCM matrix, we remark that $\nablai A_{jk}$ is \emph{a priori} nonzero only when $i \in N_j \cup \{ j,k\}$. This implies that its action can be computed within linear, as oppose to quadratic, complexity.


%%\cdots\Big[ \nabla U_j^n  \,  \big( t_n^{jk}\big)^{-(\ell'+1)} \, Y_{\ell'}^{m'} (s_n^{jk}) + U_j^n  \, \nabla \Big( \big( t_n^{jk}\big)^{-(\ell'+1)} \Big)\, Y_{\ell'}^{m'} (s_n^{jk}) + U_j^n  \,  \big( t_n^{jk}\big)^{-(\ell'+1)} \, \nabla Y_{\ell'}^{m'} (s_n^{jk}) \Big]
%%\end{multline*}
%However, since $t_n^{jk}$ and $s_n^{jk}$ depend only upon $x_j$ and $x_k$, if we assume $i \not=j$ and $i\not=k$, we obtain
%\begin{equation}\label{eq:8}
%{[\nablai A_{jk}]}_{\ell \ell'}^{mm'} = -  \frac{4 \pi \ell'}{2 \ell'+1} \, \sum_{n} w_n\, Y_\ell^m(s_n) \,\nablai U_j^n  \,  \big( t_n^{jk}\big)^{-(\ell'+1)} \, Y_{\ell'}^{m'} (s_n^{jk})
%\end{equation}
%%Such relation holds, in particular, for $i \in N_j$ and $i \not= k$.
%Thus, since $U_j^n$ depends only upon $x_i$ if $i \in N_j$ or $i=j$, we conclude that $\nablai A_{jk}$ vanishes whenever $i \not= j$ and $i \not=k$ and $i \not\in N_j$. In order to discuss the opposite case, i.e., $i = j$ or $i =k$ or $i \in N_j$, notice that the events $(i = j)$ and $(i = k)$ are mutually exclusive, as are $(i = j)$ and $(i \in N_j)$. We obtain the three subcases $i = j$, and $i = k$, and $i \in N_j \, , \, i \not= k$, which will be addressed individually.
%
%Standard differentiation implies that
%\begin{multline}\label{eq:9}
%\nablai \Big[ U_j^n  \,  \big( t_n^{jk}\big)^{-(\ell'+1)} \, Y_{\ell'}^{m'} (s_n^{jk}) \Big] = \nablai U_j^n  \,  \big( t_n^{jk}\big)^{-(\ell'+1)} \, Y_{\ell'}^{m'} (s_n^{jk}) \\
%- U_j^n  \, (\ell' + 1)  \big( t_n^{jk}\big)^{-(\ell'+2)} \, \nablai t_n^{jk} \, Y_{\ell'}^{m'} (s_n^{jk}) + U_j^n  \,  \big( t_n^{jk}\big)^{-(\ell'+1)} \, \big( \Di\, s_n^{ji} \big)^T\, \nablai Y_{\ell'}^{m'} (s_n^{jk})
%\end{multline}
%where $\Di$ emphasizes that the gradient of the vector quantity $s_j^{jk}$ is indeed its Jacobian matrix and where the extra subscripts refer to the variables with respect to which differentiation is taken. We proceed to evaluate $\nablai t_n^{jk}$ and $ \Di \, s_n^{jk}$. When $i = j$, differentiation implies
%\[
%\nablaj t_n^{jk} = \frac{s_n^{jk}}{r_k} \qquad , \qquad \Dj \, s_n^{jk} = \frac{I - s_n^{jk} \otimes s_n^{jk} }{|x_j + r_j s_n - x_k|^3}
%\]
%where $I$ is the identity matrix and $\otimes$ indicates the outer product. We remark that the Jacobian matrix $\Dj \, s_n^{jk}$ is symmetric, so that the transpose in \eqref{eq:9} is redundant. 
%Due to the particular relation between $x_j$ and $x_k$, we obtain $\nabla_{\! j} \, t_n^{jk} = - \nabla_{\! k} \, t_n^{jk}$ and $D_j \, s_n^{jk} = - D_k \, s_n^{jk}$.
%We can therefore analogously derive the case $i = k$ and those relationships imply
%\begin{multline*}
%\nabla_{\! j} \Big[ U_j^n  \,  \big( t_n^{jk}\big)^{-(\ell'+1)} \, Y_{\ell'}^{m'} (s_n^{jk}) \Big] + \nabla_{\! k} \Big[ U_j^n  \,  \big( t_n^{jk}\big)^{-(\ell'+1)} \, Y_{\ell'}^{m'} (s_n^{jk}) \Big] = \\
%  \Big[ \nabla_{\! j} \, U_j^n + \nabla_{\! k} \, U_j^n  \Big]  \big( t_n^{jk}\big)^{-(\ell'+1)} \, Y_{\ell'}^{m'} (s_n^{jk})
%\end{multline*}
%which provide a convenient way of evaluating $[\nabla_{\! k} \, A_{jk}]_{\ell \ell'}^{m m'}$ from $[\nabla_{\! j} \, A_{jk}]_{\ell \ell'}^{m m'}$. In fact, we obtain the quasi-skew-symmetric relation
%\[
%[ \nabla_{\! j} \, A_{jk}]_{\ell \ell'}^{m m'} + [\nabla_{\! k} \, A_{jk}]_{\ell \ell'}^{m m'} = -  \frac{4 \pi \ell'}{2 \ell'+1} \, \sum_{n} w_n\, Y_\ell^m(s_n) \Big[ \nabla_{\! j} \, U_j^n + \nabla_{\! k} \, U_j^n  \Big]  \big( t_n^{jk}\big)^{-(\ell'+1)} \, Y_{\ell'}^{m'} (s_n^{jk})
%\]
%for $\nablai A_{jk}$. Finally, the case $i \in N_j \, , \, i \not= k$ reduces to \eqref{eq:8}.

% Ben's notes
%\noindent
{\bf A quick overview of sparsity of $\nablai A_{jk}$}: There is nothing new, but should clarify a bit the computational complexity of computing the forces.
\begin{itemize}
\item {\bf Diagonal terms: $j=k$:}
	\begin{itemize}
		\item If $i\in N_j=N_k$ or $i=j=k$, then $[ \nablai A_{jj}]_{\ell \ell'}^{m m'} \neq 0$. 
		\item If $i\not \in N_j=N_k$ and  $i\neq j=k$, then $[ \nablai A_{jk}]_{\ell \ell'}^{m m'} = 0$.
	\end{itemize}
\item {\bf Off-diagonal terms: $j\neq k$:}
	\begin{itemize}
		\item If $i=k$, then $[ \nablai A_{jk}]_{\ell \ell'}^{m m'} \neq 0$.
		\item If $i=j$, then $[ \nablai A_{jk}]_{\ell \ell'}^{m m'} \neq 0$.
		\item If $i\neq j$ and $i\neq k$
		\begin{itemize}
			\item If $i\in N_j$ and $i\neq k$, then $[ \nablai A_{jk}]_{\ell \ell'}^{m m'} \neq 0$.
			\item If $i\not\in N_j$ and $i\neq k$, then $[ \nablai A_{jk}]_{\ell \ell'}^{m m'} =0 0$.
		\end{itemize}
	\end{itemize}
\end{itemize}
From a different perspective, we want now to analyse the complexity of the matrix-vector product $\nablai A f$ for some vector $f$.
%

We see that 
\begin{align*}
	[(\nablai A f)_j]_\ell^m 
	&= \sum_k \sum_{\ell',m'} [\nablai A_{jk}]_{\ell \ell'}^{m m'} [f_k]_{\ell'}^{m'} \\
	&= \sum_{\ell',m'} [\nablai A_{jj}]_{\ell \ell'}^{m m'} [f_j]_{\ell'}^{m'} 
	+ \sum_{k\neq j} \sum_{\ell',m'} [\nablai A_{jk}]_{\ell \ell'}^{m m'} [f_k]_{\ell'}^{m'} 
\end{align*}
so that we split $\nablai A$ into the block-diagonal part $\nablai A_d$ and the off-diagonal part $\nablai A_o$.
\begin{align*}
	[(\nablai A_d f)_j]_\ell^m 
	&= \sum_{\ell',m'} [\nablai A_{jj}]_{\ell \ell'}^{m m'} [f_j]_{\ell'}^{m'} \\
	[(\nablai A_o f)_j]_\ell^m 
	&= \sum_{k\neq j} \sum_{\ell',m'} [\nablai A_{jk}]_{\ell \ell'}^{m m'} [f_k]_{\ell'}^{m'} 
\end{align*}
For the diagonal part, we have that
\[
	[(\nablai A_d f)_j]_\ell^m = 0
\]
if $i\not\in N_j$ and $i\neq j$. For a fixed $i$, the number of non-zero contributions (indexed by $j$) from the diagonal is independent of $M$. $\mathcal O(1)$

For the off-diagonal part, we distinguish three cases.

Case 1 ($i=j$ and necessarily $i\neq N_j$):
\[
	[(\nablai A_o f)_i]_\ell^m 
	= 	\sum_{k\neq i} \sum_{\ell',m'} [\nablai A_{ik}]_{\ell \ell'}^{m m'} [f_k]_{\ell'}^{m'} 
\]
Here, all terms $k\neq i$ must be considered. This operation is $\mathcal O(M)$ but this case happens for one $j$ only.

Case 2 ($i\neq j$ and $i\in N_j$):
\[
	[(\nablai A_o f)_j]_\ell^m 
	= \sum_{\ell',m'} [\nablai A_{ji}]_{\ell \ell'}^{m m'} [f_i]_{\ell'}^{m'} 
	+ \sum_{k\neq j, k\neq i} \sum_{\ell',m'} [\nablai A_{jk}]_{\ell \ell'}^{m m'} [f_k]_{\ell'}^{m'} 
\]
This operation is $\mathcal O(M)$, but this case happens for $\mathcal O(1)$ values of $j$ only.

Case 3 ($i\neq j$ and $i\not \in N_j$):
\[
	[(\nablai A_o f)_j]_\ell^m 
	= \sum_{\ell',m'} [\nablai A_{ji}]_{\ell \ell'}^{m m'} [f_i]_{\ell'}^{m'} 
\]
This operation is $\mathcal O(1)$ and this needs to be done for $\mathcal O(M)$ values of $j$.

Therefore, we see that computing all coefficients of the matrix-vector product $[(\nablai A f)_j]_\ell^m $ for one fixed $i$ requires $\mathcal O(M)$ of operations. 
Now, this needs to be done for each $i=1,\ldots,M$ such that the overall complexity to compute all coefficients $[(\nablai A f)_j]_\ell^m $ for all $i$ and $j$ is $\mathcal O(M^2)$.



%1. Use that $\nablai A_\infty=\nablai A_\varepsilon$ and that $L \, X=G$:
%\begin{alignat*}{1}
%h_i &= \nablai A_\infty \, F +  A_\infty \, \nablai F - \nablai A_\varepsilon \, L \, X -  A_\varepsilon \, \nablai L \, X\\
%&= \nablai A \, (F-  G) +  A_\infty \, \nablai F  -  A_\varepsilon \, \nablai L \, X
%\end{alignat*}
%
%2. Explain how to do the following operations efficiently:
%\begin{alignat*}{3}
%\langle s, \nablai A \, (F-  G) \rangle &= \langle (\nablai A)^* \, s,  F-  G \rangle ? \\
%\langle s, A_\infty \, \nablai F \rangle &= \langle A_\infty \,^* \, s,  \nablai F \rangle \\
%\langle s, A_\varepsilon \, \nablai L \, X \rangle &= \langle (A_\varepsilon\, \nablai L)^* \, s,  X \rangle ? \\
%\end{alignat*}
%I don't see how the first and third term are efficiently done in practice?
%
%3. Use the skew-symmetric relationship from above.



\bibliography{newbiblio}
\end{document}
