\section{Conclusions}\label{sec:conclusions}

In this paper we discussed analytical gradients of the electrostatic solvation energy arising from the Polarizable Continuum Model, discretized with a domain-decomposition strategy. The complete derivation is discussed in this paper, along with a quadratic complexity estimate, with respect to the number of atoms, that is supported by extensive numerical experiments.


The pilot implementation presented in this work is a stepping stone for future work. In fact, while the overall timings are acceptable and compare well to those obtained with different discretizations, the computational effort required by the solution of the ddPCM equations and the related forces is still large. This prevents to pursue applications on large and very large systems. In particular, while a more efficient implementation can be achieved and a better parallelization strategy implemented, the quadratic scaling of the computation cannot be circumvented. For this reason, a linear scaling implementation based on the use of the Fast Multipole Method is currently under investigation.

Finally, we have presented results for classical solutes only, i.e., solutes described by a classical force field. An implementation of ddPCM in the framework of quantum chemical methods is particularly attractive, since ddPCM shares with ddCOSMO the rigorous foundations, variational discretization and overall simplicity and limited number of parameters. Furthermore, the current quadratic scaling code already has a smaller constant than BEM-based discretizations. Indeed, those methods scale quadratically with respect to the number of total grid points, as opposed to ddPCM with scales linearly with respect to $N_\text{grid}$. As the systems that can be treated with quantum mechanical methods are well within reach of ddPCM, we expect the method to be already fully applicable in the framework of quantum Chemistry. 
