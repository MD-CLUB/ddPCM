\documentclass[12pt,letterpaper,oneside]{article}


%\usepackage{mathpazo}
%\usepackage{mathptmx}
%\usepackage{avant}
%\usepackage{fourier}

\usepackage{amsfonts}
\usepackage{stmaryrd}
\usepackage{layout}
%\usepackage[italian]{babel}
\usepackage{makeidx}
\usepackage{amsthm}
%\usepackage{mathabx}
\usepackage{amsmath}
\usepackage{amscd}
\usepackage{mathrsfs}
%\usepackage{bbold}
\usepackage{geometry} 
\geometry{letterpaper} 
\usepackage{setspace}
\usepackage{titlesec}
\usepackage{titletoc}
%\usepackage{xymatrix}
%\usepackage{pictexwd,dcpic}
\usepackage[parfill]{parskip}    % Activate to begin paragraphs with an empty line
\usepackage{graphicx}
\usepackage{amssymb}
\usepackage{subfigure}
%\usepackage{epstopdf}
%\usepackage[arrsy]{kuvio}
\usepackage[all]{xy}
\DeclareGraphicsRule{.tif}{png}{.png}{`convert #1 `dirname #1`/`basename #1 .tif`.png}
%\usepackage{diagrams}


%\renewcommand{\arraystretch}{0.7}
\swapnumbers


\newcommand{\blob}{
 \rule[.0ex]{1ex}{1ex}
}


\newtheoremstyle{plain}
{3pt}
{3pt}
{\itshape}
{}
{\bf}
{.}
{.7em}
{}

\newtheoremstyle{definition}
{5pt}
{5pt}
{}
{}
{\bf}
{.}
{.7em}
{}

\swapnumbers

\theoremstyle{definition}
\newtheorem{theorem}{Theorem}[section]
\newtheorem{axiom}[theorem]{Axiom}
\newtheorem{proposition}[theorem]{Proposition}
\newtheorem{lemma}[theorem]{Lemma}
\newtheorem{cor}[theorem]{Corollary}
\newtheorem{definition}[theorem]{Def\mbox{}inition}
\newtheorem{problem}[theorem]{Problem}
\newtheorem{ass}[theorem]{Assignment}


%\theoremstyle{definition}
\newtheorem{result}[theorem]{Result}
\newtheorem{ex}[theorem]{Example}
\newtheorem{remark}[theorem]{Remark}
%\newtheorem{oss}[theorem]{Osservazione}


\renewcommand{\deg}{\operatorname{\emph{deg}}}
\newcommand{\dive}{\operatorname{div}}
\newcommand{\alt}{\operatorname{Alt}}
\newcommand{\dist}{\operatorname{dist}}
\newcommand{\diag}{\operatorname{diag}}
\renewcommand{\dist}{\operatorname{dist}}
\newcommand{\Cof}{\operatorname{Cof}}
\newcommand{\id}{\operatorname{id}}
\newcommand{\sign}{\operatorname{sign}}
\newcommand{\Prod}{\operatorname{Prod}}
\newcommand{\Sum}{\operatorname{Sum}}
\newcommand{\supp}{\operatorname{supp}}
\newcommand{\vol}{\operatorname{vol}}
\newcommand{\area}{\operatorname{area}}
\newcommand{\matr}{\operatorname{Matr}}
\newcommand{\spa}{\operatorname{span}}
\newcommand{\sinc}{\operatorname{sinc}}
\newcommand{\rank}{\operatorname{rank}}
\newcommand{\curl}{\operatorname{curl}}
\newcommand{\inte}{\operatorname{int}}
\newcommand{\dom}{\operatorname{dom}}
\newcommand{\diam}{\operatorname{diam}}
\newcommand{\bcurl}{\operatorname{\mathbf{curl}}}
\newcommand{\F}{\boldsymbol{\mathrm{F}}}
\newcommand{\A}{\boldsymbol{\mathsf{A}}}
\newcommand{\B}{\boldsymbol{\mathsf{B}}}
\newcommand{\M}{\boldsymbol{\mathsf{M}}}
\newcommand{\N}{\boldsymbol{\mathsf{N}}}
\newcommand{\n}{\boldsymbol{\mathrm{n}}}
\newcommand{\tril}{\mathsf{Tril}(\hat{K})}
\renewcommand{\u}{\boldsymbol{u}}
\newcommand{\Hu}{\hat{\boldsymbol{u}}}
\renewcommand{\v}{\boldsymbol{v}}
\newcommand{\Hv}{\hat{\boldsymbol{v}}}
\newcommand{\Piolai}{\operatorname{\mathsf{P}_{\F_i}}}
\newcommand{\Piola}{\operatorname{\mathsf{P}_{\F}}}
\newcommand{\Piolaiinv}{\operatorname{\mathsf{P}_{\F_i}^{-1}}}
\newcommand{\Piolainv}{\operatorname{\mathsf{P}_{\F}^{-1}}}
\newcommand{\Pioladivi}{\operatorname{\mathsf{T}_{\F_i}}}
\newcommand{\Pioladiv}{\operatorname{\mathsf{T}_{\F}}}
\newcommand{\Pioladiviinv}{\operatorname{\mathsf{T}_{\F_i}^{-1}}}
\newcommand{\Pioladivinv}{\operatorname{\mathsf{T}_{\F}^{-1}}}
\newcommand{\Res}{\mathrm{Res}}
\renewcommand{\L}{L^1(\mathbb{R}^d)}
\newcommand{\Lp}{L^p(\mathbb{R}^d)}
\newcommand{\Lq}{L^q(\mathbb{R}^d)}
\newcommand{\Linf}{L^\infty(\mathbb{R}^d)}
\newcommand{\Ltwo}{L^2(\mathbb{R}^d)}
\newcommand{\range}{\operatorname{\mathrm{range}}}
\renewcommand{\graph}{\operatorname{\mathrm{graph}}}


%\renewcommand{\qed}{\hf\mbox{}ill \mbox{\raggedright \rule{.07in}{.1in}}}
\renewcommand{\qedsymbol}{\blob}

%\renewenvironment{proof}{\vspace{-0.5ex}\noindent{\sc Dimostrazione.} \hspace{0em}}
%	{\quad \qedsymbol \vspace{0.5ex}}

%\newcommand{\appsection}[1]{\let\oldthesection\thesection
%\renewcommand{\thesection}{Appendix \oldthesection}
%\section{#1}\let\thesection\oldthesection}


  
  
  

%\author{Paolo Gatto}
\title{Notes for ddCOSMO/ddPCM}


%\doublespacing


\begin{document}
\maketitle

\section{ddCOSMO}
The van der Waals molecular cavity $\Omega$ is defined as the union of $M$ spheres $\Omega_j$'s with centers $\{\boldsymbol{r}_j \}_{1\le j \le M}$ and radii $\{ \rho_j\}_{1\le j \le M}$. The solvation energy $E_s$ of the molecule is defined in terms of the solute's density of charge $\varrho$ and the reaction potential $W$ as:
\[
E_s = \frac{1}{2} f(\varepsilon_s) \int_\Omega \varrho(\boldsymbol{r}) \, W(\boldsymbol{r}) \, d\boldsymbol{r}
\]
Here $f(\varepsilon_s)$ is a constant depending on the solvent dielectric constant $\varepsilon_s$. If we assume a classical charge distribution, namely
\[
\varrho(\boldsymbol{r}) = \sum_{j=1}^M q_j \, \delta(\boldsymbol{r} -{\boldsymbol{r}_j})
\]
the solvation energy reduces to
\begin{equation}\label{eq:1}
E_s = \frac{1}{2} f(\varepsilon_s) \sum_{j=1}^M q_j \, W(\boldsymbol{r}_j)
\end{equation}
The reaction potential $W$ is the solution to the boundary value problem
\[
-\Delta W = 0 \quad \text{in }\Omega \qquad ; \qquad W(\boldsymbol{s}) = - \Phi(\boldsymbol{s}) := - \sum_{j = 1}^m \frac{q_j}{|\boldsymbol{s} -  \boldsymbol{r}_j |} \quad \text{on }\Gamma
\]
Since $W$ is harmonic over $\Omega$, it can be represented by means of an apparent surface charge $\sigma$ through the single layer potential $\tilde{\mathcal{S}}$:
\[
W(\boldsymbol{r}) = \int_\Gamma \frac{\sigma(\boldsymbol{s})}{| \boldsymbol{r} -  \boldsymbol{s} |} \, d \boldsymbol{s} =: \tilde{\mathcal{S}} \sigma(\boldsymbol{r}) \qquad , \qquad \forall \, \boldsymbol{r} \in \Omega
\]
The apparent surface charge $\sigma$ satisfies the integral equation:
\[
\mathcal{S} \sigma (\boldsymbol{s}) := \int_\Gamma \frac{\sigma(\boldsymbol{s'})}{| \boldsymbol{s} -  \boldsymbol{s}' |} \, d \boldsymbol{s}' = -\Phi(\boldsymbol{s}) \qquad , \qquad \forall \, \boldsymbol{s} \in \Gamma
\]
where $\mathcal{S}$ is the single layer operator. Let us turn to the domain-decomposition approach. Since the restriction $W_j := W |_{\overline{\Omega}_j}$ is harmonic over $\Omega_j$, it can be represented as 
\begin{gather*}
W_j(\boldsymbol{r}) = \int_{\Gamma_j} \frac{\sigma_j(\boldsymbol{s})}{| \boldsymbol{r} -  \boldsymbol{s} |} \, d \boldsymbol{s} =: \tilde{\mathcal{S}}_j \sigma_j (\boldsymbol{r}) \qquad , \qquad \forall \, \boldsymbol{r} \in \Omega_j \\
W_j(\boldsymbol{s}) = \int_{\Gamma_j} \frac{\sigma_j(\boldsymbol{s}')}{| \boldsymbol{s} -  \boldsymbol{s}' |} \, d \boldsymbol{s}' =: \mathcal{S}_j \sigma_j (\boldsymbol{s}) \qquad , \qquad \forall \, \boldsymbol{s} \in \Gamma_j
\end{gather*}
for some surface charge $\sigma_j$. Each restriction $W_j$ satisfies the following boundary condition:
\begin{multline*}
W_j(\boldsymbol{s}) = - \Phi(\boldsymbol{s}) \bigg( 1 - \frac{1}{|N_j(\boldsymbol{s}) |} \sum_{k \in N_j(\boldsymbol{s})} \chi_k(\boldsymbol{s}) \bigg) + \frac{1}{|N_j(\boldsymbol{s}) |} \sum_{k \in N_j(\boldsymbol{s})} \chi_k(\boldsymbol{s}) W_k(\boldsymbol{s}) \\
\forall \, \boldsymbol{s} \in \Gamma_j
\end{multline*}
where $\chi_k$ is the characteristic function of $\Omega_k$. If we employ single layer potentials and operators, and define coefficients
\[
\omega_{kj}(\boldsymbol{s}) =  \frac{\chi_k(\boldsymbol{s}) }{|N_j(\boldsymbol{s}) |}
\]
we can rewrite the previous equation in terms of surface charges $\sigma_j$'s as:
\[
\mathcal{S}_j \sigma_j (\boldsymbol{s}) = -\Phi(\boldsymbol{s}) \bigg( 1 - \sum_{k \in N_j(\boldsymbol{s})} \omega_{kj}(\boldsymbol{s}) \bigg) + \omega_{kj}(\boldsymbol{s}) \, \tilde{\mathcal{S}}_k \sigma_k (\boldsymbol{s}) \qquad , \qquad \boldsymbol{s} \in \Gamma_j
\]
If we now expand each local surface charge in series of spherical harmonics $Y_\ell^m$ as
\begin{equation}\label{eq:2}
\sigma_j (\boldsymbol{s})= \sum_{\ell,m} [X_j]_\ell^m \, Y_\ell^m(\boldsymbol{s})
\end{equation}
we can discretize the previous equation as:
\[
[L_{jj} ]_{\ell \ell'}^{m m'} [X_j]_{\ell'}^{m'} = [ g_j ]_\ell^m - [L_{jk} ]_{\ell \ell'}^{m m'} [X_k]_{\ell'}^{m'}  \qquad , \qquad \forall \, j
\]
where summation is understood over repeated indices. Equivalently, we can consider the following linear system
\[
\begin{pmatrix}
L_{11} 	& \cdots 	& L_{1M}  \\
\vdots 	& \ddots 	& \vdots \\
L_{M1} 	& \cdots 	& L_{MM}
\end{pmatrix}
\begin{pmatrix}
X_1 \\ \vdots \\ X_M
\end{pmatrix}
=
\begin{pmatrix}
g_1 \\ \vdots \\ g_M
\end{pmatrix}
\]
or $LX = g$ in short. Let us now return to the solvation energy in the presence of a classical charge distribution, i.e., equation (\ref{eq:1}). Since $W(\boldsymbol{r}_j) = W_j(\boldsymbol{r}_j)$, by means of single layer potentials we obtain that:
\[
E_s = \frac{1}{2} f(\varepsilon_s) \sum_{j=1}^M q_j \int_{\Gamma_j} \frac{\sigma_j(\boldsymbol{s})}{| \boldsymbol{r}_j -  \boldsymbol{s} |} \, d \boldsymbol{s} = \frac{1}{2} f(\varepsilon_s) \sum_{j=1}^M \frac{q_j}{\rho_j} \int_{\Gamma_j} \sigma_j(\boldsymbol{s}) \, d \boldsymbol{s}
\]
Without loss of generality, let us replace $\sigma_j (\boldsymbol{s}) \to \rho_j \sigma_j (\boldsymbol{s}) \to \rho_j \sigma_j (\boldsymbol{r}_j + \rho_j \boldsymbol{s})$, so that we can reduce the computation of the energy to a integrals on the unit sphere $\mathbb{S}^2$:
\[
E_s = \frac{1}{2} f(\varepsilon_s) \sum_{j=1}^M q_j \int_{\mathbb{S}^2} \sigma_j(\boldsymbol{s}) \, d \boldsymbol{s}
\]
Let us now expand each surface charge $\sigma_j$ is series of spherical harmonics as in equation (\ref{eq:2}). Since only the first mode has non-zero average, we obtain:
\[
E_s = \frac{1}{2} f(\varepsilon_s) \sum_{j=1}^M q_j [X_j]_0^0
\]
Thus, if we define
\[
[\Psi_j]_\ell^m =  \frac{1}{2} f(\varepsilon_s) q_j \delta_{\ell 0} \delta_{m 0}
\]
% introduce local operators
%\[
%\mathcal{S}_j \sigma(\boldsymbol{s}) := \int_{\Gamma_j} \frac{\sigma(\boldsymbol{s'})}{| \boldsymbol{s} -  \boldsymbol{s}' |} \, d \boldsymbol{s}' \quad \forall \, \boldsymbol{s} \in \Gamma_j  \qquad , \qquad \tilde{\mathcal{S}}_j \sigma (\boldsymbol{r}) = \int_{\Gamma_j} \frac{\sigma(\boldsymbol{s'})}{| \boldsymbol{r} -  \boldsymbol{s}' |} \, d \boldsymbol{s}' \quad \forall \, \boldsymbol{r} \in \Omega_j 
%\]
%and let 
%
%
%
%This integral equation is discretized as:
% The apparent surface charge is discretized as
%\[
%\sigma( \boldsymbol{s})= \bigcup_{j=1}^M \sigma_j( \boldsymbol{s}) = \bigcup_{j=1}^M \sigma_j( \boldsymbol{r}_j + \rho_j \boldsymbol{s}')\quad , \quad 
%\]
%where $\boldsymbol{s}'$ belongs to the unit sphere, and $Y_\ell^m$'s are spherical harmonics. Similarly, we can represent the reaction potential $W$
we can write the polarization energy as
\[
E_s =  \sum_j \sum_{\ell,m} [\Psi_j]_\ell^m [X_j]_\ell^m =: \langle \Psi, X \rangle
\]
Let us recall that $L = L(\boldsymbol{r}_1, \ldots , \boldsymbol{r}_M)$, and $g = g(\boldsymbol{r}_1, \ldots , \boldsymbol{r}_M)$, thus the solution vector $X = L^{-1} g$ does depend on $\boldsymbol{r}_1, \ldots , \boldsymbol{r}_M$ as well. The force acting on the $j$-th particle can be computed as:
\[
F_j = -\nabla_j \langle \Psi, \sigma \rangle % = - \langle \Psi, \nabla_j\sigma \rangle
\]
where $\nabla_j$ is the gradient with respect to the position of the $j$-th atom. In the following we should just refer to it with the symbol $\cdot\,'$. Let $s$ be the solution of the adjoint problem $L^*s = \Psi$. Then, in the case that $\Psi$ is independent of $\boldsymbol{r}_1, \ldots , \boldsymbol{r}_M$, we obtain
\[
F_j = - \langle \Psi, X' \rangle = - \langle L^*s, X' \rangle = - \langle s, L X' \rangle
\]
Recalling that $LX = g$, Leibnitz formula yields $L'X + L X' = g'$, hence:
\[
F_j = -\langle s , g' - L' X \rangle = - \langle s , h \rangle
\]
where we set $h = g' - L' X$. Thus, the computation of $F_j$ amounts to the adjoint solve $L^*s = \Psi$ contracted with $h = g' - L' X$.





\section{ddPCM}
Let $\phi$ be the total electrostatic potential of the solute/solvent system, and $\Phi$ be the potential of the solute in vacuum. The reaction potential is $W = \phi - \Phi$. Let us define:
\[
\varepsilon = 
\begin{cases}
1 & \text{in }\Omega \\
\varepsilon_s & \text{on }\mathbb{R}^3 \setminus \Omega
\end{cases}
\]
where $\varepsilon_s$ is the macroscopic dielectric permittivity of the solvent. Then the reaction potential satisfies the boundary value problem:
\[
-\Delta W = 0 \quad \text{in } \mathbb{R}^3 \setminus \Gamma \qquad ; \qquad \llbracket W \rrbracket = 0 \quad  , \quad \llbracket \varepsilon \, \partial_n W \rrbracket = (\varepsilon_s - 1) \partial_n \Phi \quad \text{on }\Gamma
\]
As previously, since $W$ is harmonic over $\mathbb{R}^3 \setminus \Omega$, we can write
\[
W(\boldsymbol{r}) = \int_\Gamma \frac{\sigma(\boldsymbol{s})}{| \boldsymbol{r} - \boldsymbol{s}|} \, d\boldsymbol{s} \qquad \forall \, \boldsymbol{r} \in \mathbb{R}^3 \setminus \Gamma
\]
for some apparent surface charge $\sigma$, defined on $\Gamma$, that satisfies:
\[
\sigma = \frac{1}{4\pi} \llbracket \partial_n W \rrbracket \qquad \text{on }\Gamma
\]
Then it can be shown that $\sigma$ satisfies the integral equation
\[
\mathcal{R}_\varepsilon \, \mathcal{S} \sigma = -\mathcal{R}_\infty \Phi \qquad \text{on }\Gamma
\]
where:
\begin{align*}
\mathcal{R}_\varepsilon & = 2 \pi \frac{\varepsilon + 1}{\varepsilon - 1} I - \mathcal{D} \\
\mathcal{R}_\infty & = 2 \pi I - \mathcal{D} \\
\mathcal{D} \sigma(\boldsymbol{s}) & = \int_\Gamma \nabla \frac{1}{| \boldsymbol{s} - \boldsymbol{s}' |} \cdot \boldsymbol{n} (\boldsymbol{s}') \,\sigma(\boldsymbol{s}') \, d\boldsymbol{s}' \qquad ,\qquad \forall \, \boldsymbol{s} \in \Gamma
\end{align*}
Here $\mathcal{D}$ is known as the double layer operator.
Thus, if we set $\mathcal{S} \sigma = -\Phi_\varepsilon$, namely a ddCOSMO solve, we obtain:
\begin{equation}\label{eq:3}
\mathcal{R}_\varepsilon \Phi_\varepsilon = \mathcal{R}_\infty \Phi \qquad \text{on }\Gamma
\end{equation}
%or, equivalentely
%\[
%2 \pi \frac{\varepsilon + 1}{\varepsilon - 1} \, \Phi_\varepsilon - \mathcal{D} \, \Phi_\varepsilon = 2 \pi \, \Phi - \mathcal{D} \, \Phi  \qquad \text{on }\Gamma
%\]
In order to treat this integral equation through a domain-decomposition approach, let us define the trivial extensions
\[
\Phi_j =
\begin{cases}
\Phi & \text{on }\Gamma_j \cap \Gamma \\
0 & \text{otherwise}
\end{cases}
\qquad , \qquad 
\Phi_{\varepsilon,j} =
\begin{cases}
\Phi_\varepsilon & \text{on }\Gamma_j \cap \Gamma \\
0 & \text{otherwise}
\end{cases}
\]
Those extensions can be more compactly written as
\[
\Phi_j = U_j \, \Phi \quad , \quad \Phi_{\varepsilon,j} = U_{\varepsilon,j} \, \Phi_\varepsilon  \qquad ; \qquad U_j(\,\cdot\,) = 1 - \sum_{k \in N_j(\,\cdot\,)} \omega_{jk}(\,\cdot\,)
\]
Notice that $U_j$ is the characteristic function of $\Gamma_j \cap \Gamma$, while $1 - U_j$ is the characteristic function of $\Gamma_j \setminus \Gamma_j \cap \Gamma$. If we define:
\begin{gather*}
\mathcal{D}_j \sigma(\boldsymbol{s})  = \int_{\Gamma_j} \nabla \frac{1}{| \boldsymbol{s} - \boldsymbol{s}' |} \cdot \boldsymbol{n} (\boldsymbol{s}') \,\sigma(\boldsymbol{s}') \, d\boldsymbol{s}' \qquad ,\qquad \forall \, \boldsymbol{s} \in \Gamma_j \\
\tilde{\mathcal{D}}_j \sigma(\boldsymbol{r})  = \int_{\Gamma_j} \nabla \frac{1}{| \boldsymbol{r} - \boldsymbol{s} |} \cdot \boldsymbol{n} (\boldsymbol{s}) \,\sigma(\boldsymbol{s}) \, d\boldsymbol{s} \qquad ,\qquad \forall \, \boldsymbol{r} \in \mathbb{R}^3 \setminus \Gamma_j %\\
%\tilde{\mathcal{R}}_{\varepsilon,j} = 2 \pi \frac{\varepsilon + 1}{\varepsilon - 1} I - \tilde{\mathcal{D}}_j \qquad , \qquad
%\tilde{\mathcal{R}}_{\infty,} = 2 \pi I - \tilde{\mathcal{D}}_j
\end{gather*}
then, for every $\boldsymbol{s} \in \Gamma_j \cap \Gamma$, we obtain
\begin{multline*}
\mathcal{D} \Phi (\boldsymbol{s}) =  \int_\Gamma \nabla \frac{1}{| \boldsymbol{s} - \boldsymbol{s}' |} \cdot \boldsymbol{n} (\boldsymbol{s}') \,\Phi(\boldsymbol{s}') \, d\boldsymbol{s}'  \\
= \sum_k \int_{\Gamma_k} \nabla \frac{1}{| \boldsymbol{s} - \boldsymbol{s}' |} \cdot \boldsymbol{n} (\boldsymbol{s}') \,\Phi_k(\boldsymbol{s}') \, d\boldsymbol{s}'  = \mathcal{D}_j \, \Phi_j(\boldsymbol{s}) + \sum_{k \ne j} \tilde{\mathcal{D}}_k \, \Phi_k(\boldsymbol{s})
\end{multline*}
and a similar result holds for $\Phi_\varepsilon$. Thus, for every $\boldsymbol{s} \in \Gamma_j \cap \Gamma$, we can write equation (\ref{eq:3}) as:
\begin{multline*}
2 \pi \frac{\varepsilon + 1}{\varepsilon - 1} \, \Phi_{\varepsilon,j}(\boldsymbol{s}) -  \mathcal{D}_j \, \Phi_{\varepsilon,j}(\boldsymbol{s}) - \sum_{k \ne j} \tilde{\mathcal{D}}_k \, \Phi_{\varepsilon,k}(\boldsymbol{s}) \\
= 2 \pi  \, \Phi_j(\boldsymbol{s}) -  \mathcal{D}_j \, \Phi_j(\boldsymbol{s}) - \sum_{k \ne j} \tilde{\mathcal{D}}_k \, \Phi_k(\boldsymbol{s})
\end{multline*}
By setting
\[
{\mathcal{R}}_{\varepsilon,j} = 2 \pi \frac{\varepsilon + 1}{\varepsilon - 1} I - {\mathcal{D}}_j \qquad , \qquad
\tilde{\mathcal{R}}_{\varepsilon,j} =  - \tilde{\mathcal{D}}_j
\]
with obvious extension to the case $\varepsilon = \infty$, we obtain:
\[
{\mathcal{R}}_{\varepsilon,j} \, \Phi_{\varepsilon,j} + \sum_{k \ne j} \tilde{\mathcal{R}}_{\varepsilon,j} \, \Phi_{\varepsilon,j} = {\mathcal{R}}_{\infty,j} \, \Phi_{j} + \sum_{k \ne j} \tilde{\mathcal{R}}_{\infty,j} \, \Phi_{j} \quad \text{on }\Gamma_j \cap \Gamma
\]
which, together with the condition $\Phi_{\varepsilon,j} = 0$ on $\Gamma_j \setminus \Gamma_j \cap \Gamma$ is the local problem. In order to lump those two equations into a single one, we resort to characteristic functions. In fact, we can equivalently write:
\begin{align*}
(1 - U_j)\Phi_{\varepsilon,j} & = 0 \quad \text{on }\Gamma_j \\
U_j \bigg( {\mathcal{R}}_{\varepsilon,j} \, \Phi_{\varepsilon,j} + \sum_{k \ne j} \tilde{\mathcal{R}}_{\varepsilon,j} \, \Phi_{\varepsilon,j} - {\mathcal{R}}_{\infty,j} \, \Phi_{j} - \sum_{k \ne j} \tilde{\mathcal{R}}_{\infty,j} \, \Phi_{j} \bigg) & = 0 \quad \text{on }\Gamma_j
\end{align*}
and, after premultiplying the first equation by a non-zero constant $\alpha$, side-wise addition yields:
\begin{multline*}
\alpha(1 - U_j)\Phi_{\varepsilon,j} + U_j \bigg( {\mathcal{R}}_{\varepsilon,j} \, \Phi_{\varepsilon,j} + \sum_{k \ne j} \tilde{\mathcal{R}}_{\varepsilon,j} \, \Phi_{\varepsilon,j}\bigg) = U_j \bigg( {\mathcal{R}}_{\infty,j} \, \Phi_{j} - \sum_{k \ne j} \tilde{\mathcal{R}}_{\infty,j} \, \Phi_{j} \bigg)\\ \text{on }\Gamma_j
\end{multline*}
If we choose $\alpha = 2\pi(\varepsilon + 1)/(\varepsilon - 1)$, we obtain:
\begin{multline}\label{eq:4}
2\pi \frac{\varepsilon + 1}{\varepsilon - 1}\Phi_{\varepsilon,j} - U_j \bigg( {\mathcal{D}}_j \Phi_{\varepsilon,j} + \sum_{k \ne j} \tilde{\mathcal{D}}_{\varepsilon,k} \, \Phi_{\varepsilon,k}  \bigg) = 2 \pi U_j \Phi_j - U_j \bigg( {\mathcal{D}}_j \Phi_{j} + \sum_{k \ne j} \tilde{\mathcal{D}}_{k} \, \Phi_{k}  \bigg) \\ \text{on }\Gamma_j
\end{multline}
Let us recall that the load $\Phi_j$ can be computed as $\Phi_j = U_j \Phi$. In order to define an approximation, we interpret equation (\ref{eq:4}) in a variational setting, with test functions given by spherical harmonics. We obtain the following linear system:
\[
\begin{pmatrix}
A_{11}^\varepsilon	& \cdots 	& A_{1M}^\varepsilon  \\
\vdots 			& \ddots 	& \vdots \\
A_{M1}^\varepsilon 	& \cdots 	& A_{MM}^\varepsilon
\end{pmatrix}
\begin{pmatrix}
\Phi_1^\varepsilon \\ \vdots \\ \Phi_M^\varepsilon
\end{pmatrix}
=
\begin{pmatrix}
A_{11}^\infty	& \cdots 	& A_{1M}^\infty  \\
\vdots 		& \ddots 	& \vdots \\
A_{M1}^\infty 	& \cdots 	& A_{MM}^\infty
\end{pmatrix}
\begin{pmatrix}
\Phi_1 \\ \vdots \\ \Phi_M
\end{pmatrix}
\]
or, in short
\[
A_\varepsilon \Phi_\varepsilon = A_\infty \Phi
\]
The entries are given by:
\begin{align*}
{[A_{jj}^\varepsilon]}_{\ell \ell'}^{mm'}& = 2\pi \frac{\varepsilon + 1}{\varepsilon - 1}\, \delta_{\ell \ell'} \delta_{m m'} - \frac{2\pi}{2 \ell' + 1} \sum_n w_n \,Y_\ell^m(\boldsymbol{s}_n) \, U_j^n \, Y_{\ell'}^{m'}(\boldsymbol{s}_n) \\
{[A_{jk}^\varepsilon]}_{\ell \ell'}^{mm'}& = -  \frac{4 \pi \ell'}{2 \ell'+1} \sum_n w_n \, Y_\ell^m(\boldsymbol{s}_n) \, U_j^n \, \bigg( \frac{\rho_k}{|\boldsymbol{r}_j + \rho_j \boldsymbol{s}_n - \boldsymbol{r}_k|} \bigg)^{\ell'+1} \, Y_{\ell'}^{m'} \bigg( \frac{\boldsymbol{r}_j + \rho_j \boldsymbol{s}_n - \boldsymbol{r}_k}{|\boldsymbol{r}_j + \rho_j \boldsymbol{s}_n - \boldsymbol{r}_k|} \bigg)
\end{align*}
where, for convenience, we set
\[
U_j^n = U_j(\boldsymbol{r}_j + \rho_j\boldsymbol{s}_n)
\]
Let us recall that $L\,X = -\Phi_\varepsilon$, so that the full discretization reads as $A_\varepsilon L \, X = - A_\infty \Phi$. We follow the same approach we used in the ddCOSMO case and set $(A_\varepsilon L)^* s = \Psi$. The force acting on the $j$-th particle is computed as:
\[
F_j = - \langle \Psi, X'\rangle = - \langle s , A_\varepsilon L \, X' \rangle
\]
and, using Leibnitz rule, we obtain that
\[
F_j = - \langle s , - A_\infty' \Phi - A_\infty \Phi' - A_\varepsilon' L \, X - A_\varepsilon L' \, X  \rangle = - \langle s , h \rangle
\]
where we set $h = -A_\infty' \Phi - A_\infty \Phi' + A_\varepsilon' \, \Phi_\varepsilon - A_\varepsilon L' \, X$. Thus we need to compute the analytical derivatives of $A_\varepsilon$ or, more specifically, their action. To this aim, let us drop the $\varepsilon$-dependency for ease of notation, and refer to the partial derivatives as $\partial_{i,\alpha} = \partial/\partial r_{i,\alpha}$.
%The action of $A$ is obtained as:
%\[
%[A \phi \,_j]_\ell^m = \sum_k \sum_{\ell',m'} [A_{jk}]_{\ell \ell'}^{m m'} \, [\phi_k]_{\ell'}^{m'}
%% = \sum_{k\ne j} \sum_{\ell',m'} [A_{jk}]_{\ell \ell'}^{m m'} \, [\phi_k]_{\ell'}^{m'} + \sum_{\ell',m'} [A_{jj}]_{\ell \ell'}^{m m'} \, [\phi_j]_{\ell'}^{m'}
%\]
%and its partial derivative is:
%\[
%\partial_{i,\alpha} [A\phi \,_j]_l^m = [ (\partial_{i,\alpha} A) \phi \,_j]_l^m + [ (A \, \partial_{i,\alpha} \phi) \,_j]_l^m
%\]
%Let us focus on the first term in the sum.
Since the entries of ${[A_{jj}]}_{\ell \ell'}^{mm'}$ are functions of $\boldsymbol{r}_j$ only, we obtain:
\begin{align*}
[(\partial_{i,\alpha} A) \phi \,_j]_\ell^m & = \sum_{k\ne j} \sum_{\ell',m'} \partial_{i,\alpha}[A_{jk}]_{\ell \ell'}^{m m'} \, [\phi_k]_{\ell'}^{m'}  & i  &\ne j \\
[(\partial_{j,\alpha} A) \phi \,_j]_\ell^m & = \sum_{k\ne j} \sum_{\ell',m'} \partial_{j,\alpha}[A_{jk}]_{\ell \ell'}^{m m'} \, [\phi_k]_{\ell'}^{m'} + \sum_{\ell',m'} \partial_{j,\alpha} [A_{jj}]_{\ell \ell'}^{m m'} \, [\phi_j]_{\ell'}^{m'} & &
\end{align*}
Then, in the case $i\ne j$, we have:
\begin{align*}
[(\partial_{i,\alpha} A) \phi\,_j]_\ell^m = & -\sum_{k\ne j}\sum_{\ell',m'}  \frac{4 \pi \ell'}{2 \ell'+1} \sum_n w_n \, Y_\ell^m(\boldsymbol{s}_n) \, U_j^n \times \\
& \qquad \quad \partial_{i,\alpha} \bigg[ \bigg( \frac{\rho_k}{| \boldsymbol{r}_j + \rho_j \boldsymbol{s}_n - \boldsymbol{r}_k |} \bigg)^{\ell'+1} \, Y_{\ell'}^{m'} \bigg( \frac{\boldsymbol{r}_j + \rho_j \boldsymbol{s}_n - \boldsymbol{r}_k}{| \boldsymbol{r}_j + \rho_j \boldsymbol{s}_n - \boldsymbol{r}_k |} \bigg) \bigg] \, [\phi_k]_{\ell '}^{m '} \\ \\
= & -\sum_{\ell',m'}  \frac{4 \pi \ell'}{2 \ell'+1} \sum_n w_n \, Y_\ell^m(\boldsymbol{s}_n) \, U_j^n \times \\
& \qquad \quad \sum_{k\ne j} \partial_{i,\alpha} \bigg[ \bigg( \frac{\rho_k}{| \boldsymbol{r}_j + \rho_j \boldsymbol{s}_n - \boldsymbol{r}_k |} \bigg)^{\ell'+1} \, Y_{\ell'}^{m'} \bigg( \frac{\boldsymbol{r}_j + \rho_j \boldsymbol{s}_n - \boldsymbol{r}_k}{| \boldsymbol{r}_j + \rho_j \boldsymbol{s}_n - \boldsymbol{r}_k |} \bigg) \bigg] \, [\phi_k]_{\ell '}^{m '} \\ \\
= & -\sum_{\ell',m'}  \frac{4 \pi \ell'}{2 \ell'+1} \sum_n w_n \, Y_\ell^m(\boldsymbol{s}_n) \, U_j^n \times \\
& \qquad \quad \partial_{i,\alpha} \bigg[ \bigg( \frac{\rho_i}{| \boldsymbol{r}_j + \rho_j \boldsymbol{s}_n - \boldsymbol{r}_i |} \bigg)^{\ell'+1} \, Y_{\ell'}^{m'} \bigg( \frac{\boldsymbol{r}_j + \rho_j \boldsymbol{s}_n - \boldsymbol{r}_i}{| \boldsymbol{r}_j + \rho_j \boldsymbol{s}_n - \boldsymbol{r}_i |} \bigg) \bigg] \, [\phi_i]_{\ell '}^{m '}
\end{align*}
Similarly, when $i = j$, we obtain:
\begin{align*}
[(\partial_{j,\alpha} A) \phi \,_j]_\ell^m  = & - \sum_{k\ne j} \sum_{\ell',m'}\frac{4 \pi \ell'}{2 \ell'+1} \sum_n w_n \, Y_\ell^m(\boldsymbol{s}_n) \times \\
& \qquad \quad \partial_{j,\alpha} \bigg[ U_j^n  \, \bigg( \frac{\rho_k}{|\boldsymbol{r}_j + \rho_j \boldsymbol{s}_n - \boldsymbol{r}_k|} \bigg)^{\ell'+1} \, Y_{\ell'}^{m'} \bigg( \frac{\boldsymbol{r}_j + \rho_j \boldsymbol{s}_n - \boldsymbol{r}_k}{|\boldsymbol{r}_j + \rho_j \boldsymbol{s}_n - \boldsymbol{r}_k |} \bigg) \bigg] \,  [\phi_k]_{\ell'}^{m'} +  \\  \\
& - \sum_{\ell',m'} \frac{2\pi}{2 \ell'+1}  \sum_n w_n \, Y_\ell^m(\boldsymbol{s}_n) \, \partial_{j,\alpha} U_j^n \, Y_{\ell'}^{m'}(\boldsymbol{s}_n) \, [\phi_j]_{\ell '}^{m '} 
\end{align*}
Notice that:
\begin{align*}
\partial_{i,\alpha} \bigg( \frac{\rho_i}{| \boldsymbol{r}_j + \rho_j \boldsymbol{s}_n - \boldsymbol{r}_i |} \bigg)^{\ell'+1} & =  (\ell'+1) \, (\: \cdot \: )^{\ell'} \rho_i \frac{ r_{j,\alpha} + \rho_j s_{n,\alpha} - r_{i,\alpha}}{| \boldsymbol{r}_j + \rho_j \boldsymbol{s}_n - \boldsymbol{r}_i |^3}  \\
\partial_{j,\alpha} \bigg( \frac{\rho_k}{| \boldsymbol{r}_j + \rho_j \boldsymbol{s}_n - \boldsymbol{r}_k |} \bigg)^{\ell'+1} & =  - (\ell'+1) \, (\: \cdot \: )^{\ell'} \rho_k \frac{ r_{j,\alpha} + \rho_j s_{n,\alpha} - r_{k,\alpha}}{| \boldsymbol{r}_j + \rho_j \boldsymbol{s}_n - \boldsymbol{r}_k |^3}
\end{align*}
Let $\boldsymbol{s} = \boldsymbol{r} / |\boldsymbol{r}| \in \mathbb{S}^2$. Using index notation we have that:
\[
\partial_{i,\alpha} Y= \frac{\partial Y}{\partial s_\beta} \frac{\partial s_\beta}{\partial r_{\gamma}} \frac{\partial r_\gamma}{\partial r_{i,\alpha}}
\]
and:
\[
\frac{\partial s_\beta}{ \partial r_\gamma} = \frac{\delta_{\beta \gamma}}{|\boldsymbol{r}|} - \frac{r_\beta r_\gamma}{|\boldsymbol{r}|^3}
\]
%Let us consider the two cases:
%\[
%r_\gamma = r_{j,\gamma} + \rho_j {s}_{n,\gamma} - {r}_{i,\gamma} \qquad , \qquad r_\gamma = r_{j,\gamma} + \rho_j {s}_{n,\gamma} - {r}_{k,\gamma}
%\]
Finally, we have that:
\begin{multline*}
\partial_{i,\alpha} \bigg[ Y_{\ell'}^{m'} \bigg( \frac{\boldsymbol{r}_j + \rho_j \boldsymbol{s}_n - \boldsymbol{r}_i}{| \boldsymbol{r}_j + \rho_j \boldsymbol{s}_n - \boldsymbol{r}_i |} \bigg) \bigg] = \frac{\partial Y_{\ell'}^{m'} }{\partial s_\beta} \bigg( \frac{\delta_{\beta \gamma}}{|\boldsymbol{r}|} - \frac{r_\beta r_\gamma}{|\boldsymbol{r}|^3} \bigg)(-\delta_{\gamma \alpha}) 
= \frac{\partial Y_{\ell'}^{m'} }{\partial s_\beta} \bigg( \frac{r_\beta r_\alpha}{|\boldsymbol{r}|^3} -\frac{\delta_{\beta \alpha}}{|\boldsymbol{r}|} \bigg) = \\ \\ =
\frac{\partial Y_{\ell'}^{m'} }{\partial s_\beta} \bigg( \frac{({r}_{j,\beta} + \rho_j {s}_{n,\beta} - {r}_{i,\beta})({r}_{j,\alpha} + \rho_j {s}_{n,\alpha} - {r}_{i,\alpha})}{| \boldsymbol{r}_j + \rho_j \boldsymbol{s}_n - \boldsymbol{r}_i |^3}  -\frac{\delta_{\alpha\beta}}{| \boldsymbol{r}_j + \rho_j \boldsymbol{s}_n - \boldsymbol{r}_i |}\bigg)
\end{multline*}
and
\begin{multline*}
\partial_{j,\alpha} \bigg[ Y_{\ell'}^{m'} \bigg( \frac{\boldsymbol{r}_j + \rho_j \boldsymbol{s}_n - \boldsymbol{r}_k}{| \boldsymbol{r}_j + \rho_j \boldsymbol{s}_n - \boldsymbol{r}_k |} \bigg) \bigg] = \frac{\partial Y_{\ell'}^{m'} }{\partial s_\beta} \bigg( \frac{\delta_{\beta \gamma}}{|\boldsymbol{r}|} - \frac{r_\beta r_\gamma}{|\boldsymbol{r}|^3} \bigg)\delta_{\gamma \alpha} = \frac{\partial Y_{\ell'}^{m'} }{\partial s_\beta} \bigg( \frac{\delta_{\beta \alpha}}{|\boldsymbol{r}|} - \frac{r_\beta r_\alpha}{|\boldsymbol{r}|^3} \bigg) = \\ \\
= \frac{\partial Y_{\ell'}^{m'} }{\partial s_\beta} \bigg( \frac{\delta_{\alpha\beta}}{| \boldsymbol{r}_j + \rho_j \boldsymbol{s}_n - \boldsymbol{r}_k |} - \frac{({r}_{j,\beta} + \rho_j {s}_{n,\beta} - {r}_{k,\beta})({r}_{j,\alpha} + \rho_j {s}_{n,\alpha} - {r}_{k,\alpha})}{| \boldsymbol{r}_j + \rho_j \boldsymbol{s}_n - \boldsymbol{r}_k |^3}\bigg)
\end{multline*}

Let us now discuss how to efficiently compute the contraction $\langle s, A' \phi \rangle$. To this aim let us split the computation as
\[
\langle s, A' \phi \rangle  = \sum_j \sum_{\ell,m} [s_j]_\ell^m [A' \phi \,_j]_\ell^m  = \sum_{j \ne i} \sum_{\ell,m} [s_j]_\ell^m [A' \phi \,_j]_\ell^m + \sum_{\ell,m} [s_i]_\ell^m [A' \phi \,_i]_\ell^m = I_1 + I_2
\]
We shall begin with the case $j \ne i$, and rearrange summations so that the computation is efficient. Let us recall that
\[
I_1 = -  \sum_{j \ne i} \sum_{\ell,m} \sum_{\ell',m'} \sum_n  \frac{4 \pi \ell'}{2 \ell'+1} \, w_n \, Y_\ell^m(\boldsymbol{s}_n) \, U_j^n \, f_1(j,n,\ell',m')\, [\phi_i]_{\ell '}^{m '} \, [s_j]_{\ell}^{m}
\]
where, for ease of notation, we set
\[
f_1(j,n,\ell',m') = \bigg[ \bigg( \frac{\rho_i}{| \boldsymbol{r}_j + \rho_j \boldsymbol{s}_n - \boldsymbol{r}_i |} \bigg)^{\ell'+1} \, Y_{\ell'}^{m'} \bigg( \frac{\boldsymbol{r}_j + \rho_j \boldsymbol{s}_n - \boldsymbol{r}_i}{| \boldsymbol{r}_j + \rho_j \boldsymbol{s}_n - \boldsymbol{r}_i |} \bigg) \bigg]'
\]
Let us remark that summation occurs over six indices, namely $j,\ell,m,\ell',m',n$. We can rearrange summations as follow
\[
I_1  =  - \sum_n w_n  \sum_{j \ne i}  \bigg( \sum_{\ell'}  \frac{4 \pi \ell'}{2 \ell'+1} \sum_{m'} [\phi_i]_{\ell '}^{m '} \, f_1(j,n,\ell',m') \bigg) \bigg( U_j^n \sum_{\ell,m}    Y_\ell^m(\boldsymbol{s}_n) \, [s_j]_{\ell}^{m} \bigg)
\]
If we further define
\begin{align*}
f_2(j,n) & = \sum_{\ell'} \frac{4 \pi \ell'}{2 \ell' + 1} \sum_{m'} [\phi_i]_{\ell'}^{m'} \,f_1(j,n,\ell',m') \\
f_3(j,n) & =  U_j^n\, \sum_{\ell,m}    Y_\ell^m(\boldsymbol{s}_n) \, [s_j]_{\ell}^{m}
\end{align*}
we can compactly write
\[
I_1 = - \sum_n w_n  \sum_{j \ne i} f_2(j,n) \, f_3(j,n)
\]
Let us recall the bounds on the summation indices:
\[
1 \le n \le N \quad , \quad 0 \le \ell, \ell' \le L  \quad , \quad -\ell \le m \le \ell \quad ,\quad 1 \le j \le M
\]
Then, provided we can precompute $f_1$, the cost of one evaluation of $f_2$ is $O(L^2)$ and similarly for $f_3$. Thus the total cost of evaluating $I_1$ is $O(M\,N\,L^2)$.

Let us now consider the case $j =i$:
\begin{multline*}
I_2 = -  \sum_{\ell,m} \Bigg[  \sum_{k\ne j} \sum_{\ell',m'}\frac{4 \pi \ell'}{2 \ell'+1} \sum_n w_n \, Y_\ell^m(\boldsymbol{s}_n)\, f_1(k,\ell',m',n) \,  [\phi_k]_{\ell'}^{m'} +  \\ 
 - \sum_{\ell',m'} \frac{2\pi}{2 \ell'+1}  \sum_n w_n \, Y_\ell^m(\boldsymbol{s}_n) \, \big( U_j^n \big)' \, Y_{\ell'}^{m'}(\boldsymbol{s}_n) \, [\phi_j]_{\ell '}^{m '} \Bigg] [s_j]_\ell^m
\end{multline*}
where, for ease of notation, we set
\[
f_1(k,\ell',m',n) =  \bigg[ U_j^n  \, \bigg( \frac{\rho_k}{|\boldsymbol{r}_j + \rho_j \boldsymbol{s}_n - \boldsymbol{r}_k|} \bigg)^{\ell'+1} \, Y_{\ell'}^{m'} \bigg( \frac{\boldsymbol{r}_j + \rho_j \boldsymbol{s}_n - \boldsymbol{r}_k}{|\boldsymbol{r}_j + \rho_j \boldsymbol{s}_n - \boldsymbol{r}_k |} \bigg) \bigg]'
\]
Let us break $I_2$ down into two contributions:
\begin{align*}
I_{2,1} & = -  \sum_{\ell,m}   \sum_{k\ne j} \sum_{\ell',m'}\frac{4 \pi \ell'}{2 \ell'+1} \sum_n w_n \, Y_\ell^m(\boldsymbol{s}_n) \, f_1(k,\ell',m',n)\,  [\phi_k]_{\ell'}^{m'}  [s_j]_\ell^m \\
I_{2,2}&  = \sum_{\ell,m}
\sum_{\ell',m'} \frac{2\pi}{2 \ell'+1}  \sum_n w_n \, Y_\ell^m(\boldsymbol{s}_n) \, \big( U_j^n \big)' \, Y_{\ell'}^{m'}(\boldsymbol{s}_n) \, [\phi_j]_{\ell '}^{m '} [s_j]_\ell^m
\end{align*}
In the case of $I_{2,1}$, the summation occurs over six indices, namely $\ell,m,k,\ell',m',n$. We can rearrange such summation as
\[
I_{2,1}  = - \sum_n w_n   \sum_{k\ne j} \bigg( \sum_{\ell'}  \frac{4 \pi \ell'}{2 \ell'+1}   \sum_{m'}[\phi_k]_{\ell'}^{m'} \, f_1(k,\ell',m',n) \bigg)  \bigg( \sum_{\ell,m}  Y_\ell^m(\boldsymbol{s}_n)\, [s_j]_\ell^m \bigg)
\]
and, by setting
\begin{align*}
f_2(k,n) & =\sum_{\ell'}  \frac{4 \pi \ell'}{2 \ell'+1}   \sum_{m'} [\phi_k]_{\ell'}^{m'} \,f_1(k,\ell',m',n) \\
f_3(k,n) & = \sum_{\ell,m}  Y_\ell^m(\boldsymbol{s}_n)  [s_j]_\ell^m
\end{align*}
we obtain
\[
I_{2,1} = - \sum_n w_n \sum_{k \ne j} f_2(k,n) \, f_3(k,n)
\]
As before, this quantity can be computed in $O(M \, N \, L^2)$ operations. Finally, for $I_{2,2}$, we can rearrange summations as
\[
I_{2,2}  = \sum_n w_n \bigg( \sum_{\ell'} \frac{2\pi}{2 \ell'+1} \sum_{m'} Y_{\ell'}^{m'}(\boldsymbol{s}_n) [\phi_j]_{\ell '}^{m '} \bigg) \bigg( \big( U_j^n \big)' \sum_{\ell,m} Y_\ell^m(\boldsymbol{s}_n) \, [s_j]_\ell^m \bigg)
\]
and, by setting
\begin{align*}
f_2(n) & = \sum_{\ell'} \frac{2\pi}{2 \ell'+1} \sum_{m'} Y_{\ell'}^{m'}(\boldsymbol{s}_n) [\phi_j]_{\ell '}^{m '}  \\
f_3(n) & = \big( U_j^n \big)' \sum_{\ell,m} Y_\ell^m(\boldsymbol{s}_n) \, [s_j]_\ell^m
\end{align*}
we obtain
\[
I_{2,2} = \sum_n w_n \, f_2(n) \, f_3(n)
\]
%\begin{multline*}
%\frac{\partial s_\beta}{\partial r_{i,\alpha}} = \frac{\partial}{\partial r_{i,\alpha}}  \bigg( \frac{{r}_{j,\beta} + \rho_j {s}_{n,\beta} - {r}_{i,\beta}}{| \boldsymbol{r}_j + \rho_j \boldsymbol{s}_n - \boldsymbol{r}_i |} \bigg) = \\
%-\frac{\delta_{\alpha\beta}}{| \boldsymbol{r}_j + \rho_j \boldsymbol{s}_n - \boldsymbol{r}_i |} + \frac{({r}_{j,\beta} + \rho_j {s}_{n,\beta} - {r}_{i,\beta})({r}_{j,\alpha} + \rho_j {s}_{n,\alpha} - {r}_{i,\alpha})}{| \boldsymbol{r}_j + \rho_j \boldsymbol{s}_n - \boldsymbol{r}_i |^3}
%\end{multline*}
%and:
%\begin{multline*}
%\frac{\partial s_\beta}{\partial r_{j,\alpha}} = \frac{\partial}{\partial r_{j,\alpha}}  \bigg( \frac{{r}_{j,\beta} + \rho_j {s}_{n,\beta} - {r}_{k,\beta}}{| \boldsymbol{r}_j + \rho_j \boldsymbol{s}_n - \boldsymbol{r}_k |} \bigg) = \\
%\frac{\delta_{\alpha\beta}}{| \boldsymbol{r}_j + \rho_j \boldsymbol{s}_n - \boldsymbol{r}_k |} - \frac{({r}_{j,\beta} + \rho_j {s}_{n,\beta} - {r}_{k,\beta})({r}_{j,\alpha} + \rho_j {s}_{n,\alpha} - {r}_{k,\alpha})}{| \boldsymbol{r}_j + \rho_j \boldsymbol{s}_n - \boldsymbol{r}_k |^3}
%\end{multline*}

%\partial_{i,\alpha}  \frac{\boldsymbol{r}_j + \rho_j \boldsymbol{s}_n - \boldsymbol{r}_i}{| \boldsymbol{r}_j + \rho_j \boldsymbol{s}_n - \boldsymbol{r}_i |} & = 0 \\
%\partial_{j,\alpha} \frac{\boldsymbol{r}_j + \rho_j \boldsymbol{s}_n - \boldsymbol{r}_k}{|\boldsymbol{r}_j + \rho_j \boldsymbol{s}_n - \boldsymbol{r}_k |} & = 0
%\end{align*}
%Then:
%\begin{align*}
%\partial_{i,\alpha} [A \phi\,_j]_\ell^m = & -\sum_{\ell',m'}  \frac{4 \pi \ell'}{2 \ell'+1} \sum_n w_n \, Y_\ell^m(\boldsymbol{s}_n) \, U_j^n \times \\
%& \qquad \quad \partial_{i,\alpha} \bigg[ \bigg( \frac{\rho_i}{| \boldsymbol{r}_j + \rho_j \boldsymbol{s}_n - \boldsymbol{r}_i |} \bigg)^{\ell'+1} \, Y_{\ell'}^{m'} \bigg( \frac{\boldsymbol{r}_j + \rho_j \boldsymbol{s}_n - \boldsymbol{r}_i}{| \boldsymbol{r}_j + \rho_j \boldsymbol{s}_n - \boldsymbol{r}_i |} \bigg) \bigg] \, [\phi_i]_{\ell '}^{m '}
%\end{align*}





%\begin{figure}[t]
%\begin{center}
%\subfigure[Optimal mesh ($\varepsilon = 0.0091$), $\# \, \text{dof} = 5817$.]{
%\includegraphics[scale=0.5]{figs/mesh_025.png}\label{fig30}
%}\\
%\vspace{1cm}
%\subfigure[Convergence of error and error indicator.]{
%\includegraphics[scale=0.45]{figs/conv_025}\label{fig31}
%}
%\caption{Numerical results for $s=0.25$. The colors of the optimal mesh represent the orders of approximation of the elements, see Figure \ref{fig1}.}\label{fig3}
%\end{center}
%\end{figure}

\appendix

%\section{Appendix}\label{app:1}





\end{document}